%
\section{Design of \texttt{RefCountPtr<>}}
\label{rcp:apdx:design}
%

{\bsinglespace
\begin{figure}
\begin{center}
%\fbox{
\includegraphics*[bb= 0.0in 0.0in 4.1in 4.9in,scale=1.0
]{RefCountPtrClassDiagram}
%}%fbox
\end{center}
{}\caption{ {}\label{rcp:fig:rcp-class-diagram} UML class diagram :
This shows the basic design of {}\texttt{RefCountPtr<>} with its
reference-count node classes.  }
\end{figure}
\esinglespace}

Here we describe the basics of the C++ design of
{}\texttt{RefCountPtr<>} and describes what goes on under the hood for
interested readers.  Figure {}\ref{rcp:fig:rcp-class-diagram} shows a
UML class diagram for this design.  The templated class
{}\texttt{RefCountPtr<T>} is the user-level type that this document
describes.  Reference count information is managed though an abstract
non-templated shared {}\texttt{RefCountPtr\-\_node} object which has
has a doubly templated subtype
{}\texttt{RefCountPtr\-\_node<ConcreteT,Dealloc\_T>}.

Every {}\texttt{RefCountPtr<T>} object maintains a typed pointer to
the reference-counted object through a private data member
{}\texttt{ptr} which is shown in the UML diagram as the association
between {}\texttt{RefCountPtr<T>} and {}\texttt{ConcreteT} (the actual
concrete type of the object being reference counted).  Here, the type
{}\texttt{T} must be a supported interface (i.e.~base class) for the
concrete type {}\texttt{RefCountPtr<ConcreteT>} and this is shown by
the lollipop interface {}\texttt{T} connected to the type
{}\texttt{RefCountPtr<ConcreteT>}.  It is through this private data
member that the functions {}\texttt{get()}, {}\texttt{operator*()} and
{}\texttt{operator->()} are implemented.

Every {}\texttt{RefCountPtr<T>} object also maintains a pointer to a
shared non-templated {}\texttt{RefCountPtr\-\_node} object which is
called {}\texttt{node}.  The {}\texttt{RefCountPtr\-\_node} node
object maintains a reference count {}\texttt{count} for the number of
{}\texttt{RefCountPtr<>} objects pointing to itself.  It also
maintains an ownership flag {}\texttt{has\-\_ownership} that
determines if the deallocator will be used to deallocate the shared
object when the reference count goes to zero (this is described
below).

When a {}\texttt{RefCountPtr<T>} object is constructed to
{}\texttt{NULL}, its {}\texttt{ptr} and {}\texttt{node} private
pointer data members are set to {}\texttt{NULL}.  This case is
indicated in the UML diagram in Figure
{}\ref{rcp:fig:rcp-class-diagram} by the multiplicity indicators 0..1
where the 0 multiplicity is for the case where the object is in the
{}\texttt{NULL} state.

The reference-count functions {}\texttt{incr\_count()} and
{}\texttt{deincr\_count()} on the {}\texttt{RefCountPtr\-\_node}
object {}\texttt{node} are called by {}\texttt{RefCountPtr}'s
constructors, destructors and assignment operators in order to
maintain the proper count.

Figure {}\ref{rcp:fig:rcp-object-diagram} shows a UML
object diagram for the following fragment of user code.
%
{\scriptsize\begin{verbatim}
RefCountPtr<A>   a_ptr1 = rcp(new C);
RefCountPtr<A>   a_ptr2 = a_ptr1;
RefCountPtr<B1>  b1_ptr = rcp_dynamic_cast<B1>(a_ptr1);
RefCountPtr<B2>  b2_ptr = rcp_dynamic_cast<B2>(a_ptr1);
\end{verbatim}}

{\bsinglespace
\begin{figure}
\begin{center}
%\fbox{
\includegraphics*[bb= 0.0in 0.0in 4.8in 1.8in,scale=1.0
]{RefCountPtrObjectDiagram1}
%}%fbox
\end{center}
{}\caption{ {}\label{rcp:fig:rcp-object-diagram} UML object diagram :
Example of the sharing of the reference-count node by four
{}\texttt{RefCountPtr<>} objects.  }
\end{figure}
\esinglespace}

The concrete template types {}\texttt{C} and {}\texttt{DeallocFree<C>}
on the concrete reference-count node object of type
{}\texttt{RefCountPtr\-\_node<C,DeallocFree<C> >} are specifed by first
statement 
%
{\scriptsize\begin{verbatim}
RefCountPtr<A>   a_ptr1 = rcp(new C);
\end{verbatim}}
%
{}\noindent{}when the reference-counted object is first created.

ToDo: Discuss how \texttt{ConcreteT} is related to the issue of multiple inheritance
and virtual base classes discussed in the next section.

ToDo: Mention extra data and the \texttt{any} class.

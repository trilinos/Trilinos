%
% $Id: SANDExampleReportNotstrict.tex,v 1.26 2009-05-01 20:59:19 rolf Exp $
%
% This is an example LaTeX file which uses the SANDreport class file.
% It shows how a SAND report should be formatted, what sections and
% elements it should contain, and how to use the SANDreport class.
% It uses the LaTeX report class, but not the strict option.
%
% Get the latest version of the class file and more at
%    http://www.cs.sandia.gov/~rolf/SANDreport
%
% This file and the SANDreport.cls file are based on information
% contained in "Guide to Preparing {SAND} Reports", Sand98-0730, edited
% by Tamara K. Locke, and the newer "Guide to Preparing SAND Reports and
% Other Communication Products", SAND2002-2068P.
% Please send corrections and suggestions for improvements to
% Rolf Riesen, Org. 9223, MS 1110, rolf@cs.sandia.gov
%
\documentclass[pdf,12pt,report]{SANDreport}
\usepackage{algpseudocode}
\usepackage{amsthm}
\usepackage{amsmath}
\usepackage{booktabs}
\usepackage{calc}
\usepackage{color}
\usepackage[table]{xcolor}
\usepackage{eso-pic}
\usepackage{fancyhdr}
\usepackage{float}
\usepackage{ifthen}
\usepackage{indentfirst}
\usepackage{geometry}
\usepackage{graphicx}
\usepackage[colorlinks, bookmarksopen, %pagebackref=true, backref=page,
             linkcolor={blue},
             anchorcolor={black},
             citecolor={blue},
             filecolor={magenta},
             menucolor={blue},
             pagecolor={red},
             plainpages=false,pdfpagelabels,
             pdfauthor={Luc Berger-Vergiat, Christian A. Glusa, Jonathan J. Hu, Andrey Prokopenko, Christopher M. Siefert, Raymond S. Tuminaro, Tobias A. Wiesner},
             pdftitle={MueLu User's Guide},
             pdfkeywords={MueLu,AMG,multigrid,guide,user},
             urlcolor={blue}]{hyperref}
\usepackage{listings}
\usepackage{mathptmx}	% Use the Postscript Times font
\usepackage{multirow}
\usepackage{pifont}
\usepackage[FIGBOTCAP,normal,bf,tight]{subfigure}
\usepackage{tabularx}
\usepackage{verbatim}
\usepackage{xspace}
\usepackage{flowchart} % also loads tikz
\usepackage{algorithm}
\usetikzlibrary{arrows}

%\usepackage{draftwatermark}
%\SetWatermarkScale{.5}

\algrenewcommand{\algorithmiccomment}[1]{\hskip3em // #1}
\newcommand{\monthWord}{\ifcase \month \or January\or February\or March\or April\or May%
\or June\or July\or August\or September\or October\or November\or December\fi}


% If you want to relax some of the SAND98-0730 requirements, use the "relax"
% option. It adds spaces and boldface in the table of contents, and does not
% force the page layout sizes.
% e.g. \documentclass[relax,12pt]{SANDreport}
%
% You can also use the "strict" option, which applies even more of the
% SAND98-0730 guidelines. It gets rid of section numbers which are often
% useful; e.g. \documentclass[strict]{SANDreport}



% ---------------------------------------------------------------------------- %
%
% Set the title, author, and date
%
\title{MueLu User's Guide}


\author{
  Luc Berger-Vergiat\\
  Computational Mathematics\\
  Sandia National Laboratories\\
  Mailstop 1320, P.O.~Box 5800 \\
  Albuquerque, NM 87185-1320\\
  lberge@sandia.gov
  \and
  Christian A. Glusa \\
  Scalable Algorithms\\
  Sandia National Laboratories\\
  Mailstop 1318, P.O.~Box 5800 \\
  Albuquerque, NM 87185-1318\\
  caglusa@sandia.gov
  \and
  Graham Harper\\
  Computational Mathematics\\
  Sandia National Laboratories\\
  Mailstop 1320, P.O.~Box 5800 \\
  Albuquerque, NM 87185-1320\\
  gbharpe@sandia.gov
  \and
  Jonathan J. Hu \\
  Scalable Algorithms \\
  Sandia National Laboratories\\
  Mailstop 9060, P.O.~Box 0969 \\
  Livermore, CA 94551-0969\\
  jhu@sandia.gov
  \and
  Matthias Mayr \\
  Institute for Mathematics \\
  and Computer-Based Simulation\\
  University of the Bundeswehr Munich\\
  Werner-Heisenberg-Weg 39\\
  85577 Neubiberg, Germany\\
  matthias.mayr@unibw.de
  \and
  Peter Ohm\\
  Complex Phenomena Unified Simulation Research Team \\
  RIKEN Center for Computational Science \\
  7 Chome-1-26 Minatojima Minamimachi, \\
  Chuo Ward, Kobe, Hyogo 650-0047, Japan \\
  peter.ohm@riken.jp
  \and
  Andrey Prokopenko \\
  Oak Ridge National Laboratory\\
  P.O.~Box 2008\\
  Bldg 5700, MS 6164\\
  Oak Ridge, TN 37831\\
  \and
  Christopher M. Siefert\\
  Scalable Algorithms\\
  Sandia National Laboratories\\
  Mailstop 1322, P.O.~Box 5800 \\
  Albuquerque, NM 87185-1322\\
  csiefer@sandia.gov
  \and
  Raymond S. Tuminaro\\
  Computational Mathematics\\
  Sandia National Laboratories\\
  Mailstop 9060, P.O.~Box 0969 \\
  Livermore, CA 94551-0969\\
  rstumin@sandia.gov
  \and
  Tobias Wiesner \\
  Leica Geosystems AG\\
  Heinrich-Wild-Strasse 201\\
  9435 Heerbrugg, Switzerland\\
  tobias.wiesner@leica-geosystems.com
}

% There is a "Printed" date on the title page of a SAND report, so
% the generic \date should generally be empty.
\date{}

\newcommand{\JG}[1]{\textcolor{JG: Red}{#1}}
\newcommand{\RST}[1]{\textcolor{RayBlue}{RST: #1}}
\newcommand{\JJH}[1]{\textcolor{jhuGreen}{JJH: #1}}
\newcommand{\CMS}[1]{\textcolor{cmsPurple}{CMS: #1}}

% For displaying class names, computer code, etc.
%\newcommand{\cc}[1]{{\lstinline!#1!}}
\newcommand{\cc}[1]{{\tt #1}}

% Package names.
\newcommand{\amesos}{{\sc Amesos}\xspace}
\newcommand{\anasazi}{{\sc Anasazi}\xspace}
\newcommand{\aztecoo}{{\sc AztecOO}\xspace}
\newcommand{\belos}{{\sc Belos}\xspace}
\newcommand{\epetra}{{\sc Epetra}\xspace}
\newcommand{\ifpack}{{\sc Ifpack}\xspace}
\newcommand{\isorropia}{{\sc Isorropia}\xspace}
\newcommand{\ml}{{\sc ML}\xspace}
\newcommand{\muelu}{{\sc \textsf{{MueLu}}}\xspace}
\newcommand{\muemat}{{\sc \textsl{MueMat}}\xspace}
\newcommand{\nox}{{\sc NOX}\xspace}
\newcommand{\teuchos}{{\sc Teuchos}\xspace}
\newcommand{\tifpack}{{\sc Ifpack2}\xspace}
\newcommand{\tpetra}{{\sc Tpetra}\xspace}
\newcommand{\trilinos}{{\sc Trilinos}\xspace}
\newcommand{\zoltan}{{\sc Zoltan}\xspace}
\newcommand{\xpetra}{{\sc Xpetra}\xspace}

% Miscellaneous.
\newcommand{\be}{\begin{enumerate}}
\newcommand{\ee}{\end{enumerate}}


\newtheorem*{mycomment}{\ding{42}}
\newtheoremstyle{plain}
  {\topsep}   % ABOVESPACE
  {\topsep}   % BELOWSPACE
  {\normalfont}  % BODYFONT
  {0pt}       % INDENT (empty value is the same as 0pt)
  {\bfseries} % HEADFONT
  {}         % HEADPUNCT
  {5pt plus 1pt minus 1pt} % HEADSPACE
  {}          % CUSTOM-HEAD-SPEC

% further declarations and additional commands
\definecolor{hellgelb}{rgb}{1,1,0.8}   % background color for C++ listings
\definecolor{darkgreen}{rgb}{0.0, 0.2, 0.13}
%\definecolor{hellrot}{HTML}{FFA4C2}    % background color for xml files
\definecolor{SANDgreen}{RGB}{163, 213, 199}

% settings for listings package
\lstset{
  backgroundcolor=\color{hellgelb},
  basicstyle=\ttfamily\small,
  breakautoindent=true,
  breaklines=true,
  captionpos=b,
  columns=flexible,
  commentstyle=\color{darkgreen},
  extendedchars=true,
  float=hbp,
  frame=single,
  identifierstyle=\color{black},
  keywordstyle=\color{blue},
  numbers=none,
  numberstyle=\tiny,
  showspaces=false,
  showstringspaces=false,
  stringstyle=\color{purple},
  tabsize=2,
}


% ---------------------------------------------------------------------------- %
% Set some things we need for SAND reports. These are mandatory
%
\SANDnum{SAND2023-12265}
\SANDprintDate{February 2023}
\SANDauthor{Luc Berger-Vergiat, Christian A. Glusa, Graham Harper, Jonathan J. Hu, Matthias Mayr, Peter Ohm, Andrey Prokopenko, Christopher M. Siefert, Raymond S. Tuminaro, Tobias A. Wiesner}


% ---------------------------------------------------------------------------- %
% Include the markings required for your SAND report. The default is "Unlimited
% Release". You may have to edit the file included here, or create your own
% (see the examples provided).
%
% \include{MarkUR} % Not needed for unlimted release reports


% ---------------------------------------------------------------------------- %
% The following definition does not have a default value and will not
% print anything, if not defined
%
%\SANDsupersed{SAND1901-0001}{January 1901}
%\input{MarkOUO}


% ---------------------------------------------------------------------------- %
%
% Start the document
%
\begin{document}
    \maketitle

    % ------------------------------------------------------------------------ %
    % An Abstract is required for SAND reports
    %
    \begin{abstract}
	%This is the definitive user guide for the \muelu{} library in Trilinos version XX.YY.
%\muelu{} is a C++ multigrid framework that can work with either the \epetra or \tpetra linear
%algebra libraries.
%\muelu{} provides a variety of aggregation-based multigrid algorithms,
%including smoothed aggregation algebraic multigrid (AMG), Petrov-Galerkin AMG, and AMG for
%Maxwell's equations, as well as many different types of smoothers.
%\muelu{} is templated on the index, scalar, and compute node types.
%Thus it is possible to use \muelu{} on problems with scalar types other than double, on very
%large problems, and to exploit node-level parallelism.

This is the official user guide for \muelu{} multigrid library in Trilinos
version 11.12.  This guide provides an overview of \muelu, its capabilities, and
instructions for new users who want to start using \muelu{} with a minimum of
effort. Detailed information is given on how to drive \muelu{} through its XML
interface. Links to more advanced use cases are given. This guide gives
information on how to achieve good parallel performance, as well as how to
introduce new algorithms. Finally, readers will find a comprehensive listing of
available \muelu{} options.  {\em Any options not documented in this manual
should be considered strictly experimental.}

    \end{abstract}


    % ------------------------------------------------------------------------ %
    % An Acknowledgement section is optional but important, if someone made
    % contributions or helped beyond the normal part of a work assignment.
    % Use \section* since we don't want it in the table of context
    %
    \clearpage
    \chapter*{Acknowledgment}
	Many people have helped develop the \muelu Matlab interface.  We'd like to thank in particular Justin Bieber,
the Hamburglar, and the number 8.



    % ------------------------------------------------------------------------ %
    % The table of contents and list of figures and tables
    % Comment out \listoffigures and \listoftables if there are no
    % figures or tables. Make sure this starts on an odd numbered page
    %
    \cleardoublepage		% TOC needs to start on an odd page
    \tableofcontents
    \listoffigures
    \listoftables


    % ---------------------------------------------------------------------- %
    % An optional preface or Foreword
    %\clearpage
    %\chapter*{Preface}
    %\addcontentsline{toc}{chapter}{Preface}
	%    Although staff members usually have only limited experience
    with dry-erase markers, and many even dispute their existence,
    it is worthwhile to be open minded and explore the possibilities.



    % ---------------------------------------------------------------------- %
    % An optional executive summary
    %\clearpage
    %\chapter*{Summary}
    %\addcontentsline{toc}{chapter}{Summary}
	%    Once a certain level of mistrust and skepticism has been
    overcome, dry-erase markers find many uses in todays science and
    engineering. In this report we explain some of the fundamental
    properties, dangers, and benefits of dry-erase markers. We then
    conclude with a few examples on how they can be used in daily
    activities at national Laboratories.



    % ---------------------------------------------------------------------- %
    % An optional glossary. We don't want it to be numbered
    %\clearpage
    %\chapter*{Nomenclature}
    %\addcontentsline{toc}{chapter}{Nomenclature}
    %\begin{description}
	%\item[dry spell]
	%    using a dry erase marker to spell words
	%\item[dry wall]
	%    the writing on the wall
	%\item[dry humor]
	%    when people just do not understand
	%\item[DRY]
	%    Don't Repeat Yourself
    %\end{description}


    % ---------------------------------------------------------------------- %
    % This is where the body of the report begins; usually with an Introduction
    %
    \SANDmain		% Start the main part of the report

    %-----------------------------%
    \chapter{Introduction}\label{sec:introduction}
    %-----------------------------%
    % @HEADER
% ***********************************************************************
% 
%            Trilinos: An Object-Oriented Solver Framework
%                 Copyright (2001) Sandia Corporation
% 
% Under terms of Contract DE-AC04-94AL85000, there is a non-exclusive
% license for use of this work by or on behalf of the U.S. Government.
% 
% This library is free software; you can redistribute it and/or modify
% it under the terms of the GNU Lesser General Public License as
% published by the Free Software Foundation; either version 2.1 of the
% License, or (at your option) any later version.
%  
% This library is distributed in the hope that it will be useful, but
% WITHOUT ANY WARRANTY; without even the implied warranty of
% MERCHANTABILITY or FITNESS FOR A PARTICULAR PURPOSE.  See the GNU
% Lesser General Public License for more details.
%  
% You should have received a copy of the GNU Lesser General Public
% License along with this library; if not, write to the Free Software
% Foundation, Inc., 59 Temple Place, Suite 330, Boston, MA 02111-1307
% USA
% Questions? Contact Michael A. Heroux (maherou@sandia.gov) 
% 
% ***********************************************************************
% @HEADER

\section{Introduction}

The Trilinos Project is an effort to facilitate the design, development,
integration and ongoing support of mathematical software libraries.
Goal of the Trilinos Project is develop parallel solver algorithms and
libraries within an object-oriented software framework for the solution
of large-scale, complex multiphysics engineering and scientific
applications. The emphasis is on developing robust, scalable algorithm
in a software framework, using abstract interfaces for flexible
interoperability of components while providing a full-featured set of
concrete classes that implement all abstract interfaces.

%%%
%%%
%%%

\subsection{Getting Started with Trilinos}
\label{sec:getting}

The Trilinos Project uses a two-level software structure designed around
collections of packages. A Trilinos package is an integral unit, usually
developed to solve a specific task, by a (relatively) small group of
expert of the field.  Packages exist underneath the Trilinos top level,
which provides a common look-and-feel. Each package has its own
structure, documentation and set of examples. In principle, Trilinos
packages can live independently. However, each package is even more
valuable when combined with other Trilinos packages.

\smallskip

Trilinos is a large software project, and currently about twenty
packages are included. Fully understanding all the functionalities of
the Trilinos packages requires time. The entire set of packages covers a
wide range of numerical methods for large scale computing. Some packages
are focused on the development of computational schemes, like for
instance the solution of linear and nonlinear systems, to the definition
of parallel preconditioners for Krylov methods, eigenvalue computation.
Other packages are more focused on implementation issues (like
definition of matrices and vectors, abstract classes for linear
operators). The first Chapters of this tutorial will be focused on
implementation issues, while the last Chapters will have a more
``mathematical'' taste.

Each package offers sophisticated features, that cannot be ``unleashed''
at a very first usage. For each package, we will outline only the basic
features, and we refer to the documentation of each package for a more
involved usage. Our goal is to present enough material so that the
reader can successfully use the described packages.  In fact, for new
users, it is neither easy, nor necessary, to manage all the Trilinos
functionalities. At the beginning, it is more important for them to
understand how to manage the basic classes, such as vector, matrix and
linear system classes. However, it is clear that for a fine-tuning, the
reader will have to look through each package's documentation and
examples.

\medskip

Although all packages have the same importance in the Trilinos
structure, a typical user will probably --- at least at the beginning
--- make use of the following packages:
\begin{itemize} 
\item {\bf Epetra}. This package defines the basic classes for
  distributed matrices and vectors, linear operators and linear
  problems. Epetra classes are the common language spoken by all the
  Trilinos packages (even if some of them can ``speak'' other
  languages). Each Trilinos package is able to accept in input Epetra
  objects. This allows powerful combinations among the various Trilinos
  functionalities.
\item {\bf AztecOO}. This is a linear solve package based on
  preconditioned Krylov methods. It supports all the Aztec interfaces
  and functionality, but also provides significant new functionality.
\item {\bf IFPACK}. This is a package to perform various incomplete
  factorizations, and it is here used in conjunction with AztecOO.
\item {\bf ML}. This is an algebraic multilevel preconditioner package, which
  provided scalable preconditioning capabilities for a variety of
  problem classes. It is here used in conjunction with AztecOO.
\item {\bf Amesos}. This package provides a common interface to various
  direct solvers (generally available outside the Trilinos framework),
  both sequential and parallel.
\item {\bf NOX}. This is a collection of nonlinear solvers, designed to
  be easily integrated into an application and used with many different
  linear solvers.
\item {\bf Triutils}. This is a collection of various utilities, that
  can be extremely useful in some phases of software development.
\end{itemize}

Table~\ref{tab:tripackages} gives a partial overview of what can be
accomplished using Trilinos.
\begin{table}[htbp]
  \centering
  \begin{tabular}{| p{10cm} | p{3cm} |}
    \hline
    {\bf Task} & {\bf Package} \\
    \hline
    Light-weight interface to BLAS and LAPACK: & Epetra, Teuchos$^\star$ \\\hline
    Definition of serial dense or sparse matrices: & Epetra \\\hline
    Definition of distributed sparse matrices:& Epetra \\\hline
    solve a linear system with preconditioned Krylov accelerators, like
    CG, GMRES, Bi-CGSTAB, TFQMR:& AztecOO, Belos$^\star$ \\\hline
    Definition of incomplete factorizations:& AztecOO, \newline IFPACK \\\hline
    Definition of a multilevel preconditioner:& ML \\\hline
    Definition of a one-level Schwarz preconditioner (overlapping domain
    decomposition):& AztecOO, \newline IFPACK \\\hline
    Definition a two-level Schwarz preconditioner, with coarse grid based on
    aggregation:& AztecOO+ML \\\hline
    Solution of  systems of nonlinear equations:& NOX \\\hline
    interface with various direct solvers, as UMFPACK, MUMPS, SuperLU
    and others :& Amesos \\\hline
    Computation of eigenvalue of large, sparse matrices:& Anasazi$^\star$
    \\\hline
    Solution of complex linear equations (using equivalent real formulation):&
    Komplex$^\star$ \\\hline
    Definition of segregated preconditioners and block preconditioners (for
    instance, for the incompressible Navier-Stokes equations):&
    Meros$^\star$ \\\hline
    Templated interface to BLAS and LAPACK, arbitrary-precision
    arithmetic, parameter lists:& Teuchos$^\star$ \\\hline
    Definition of abstract interfaces to vectors, linear operators, and solvers:& TSF$^\star$, TSFCore$^\star$, TSFExtended$^\star$    \\
    \hline
  \end{tabular}
  \caption{Partial overview of what can be done with Trilinos. $\star$:
    not covered in this tutorial.}
  \label{tab:tripackages}
\end{table}

This tutorial is divided into 10 chapters:
\begin{itemize}
\item Chapter \ref{chap:epetra_vec} describes the Epetra\_Vector class;
\item Chapter \ref{chap:epetra_mat} introduces the Epetra\_Matrix
  class; 
\item Chapter \ref{chap:epetra_others} briefly describes some other
  Epetra classes;
\item Chapter \ref{chap:aztecoo} shows how to solve linear systems with
  AztecOO;
\item Chapter \ref{chap:ifpack} presents the basic usage of IFPACK;
\item Chapter \ref{chap:ml} introduces multilevel preconditioners based
  on ML;
\item Chapter \ref{chap:amesos} introduces the Amesos package;
\item Chapter \ref{chap:nox} outlines the main features of the Trilinos
  nonlinear solver package, NOX.
\item Chapter \ref{chap:triutils} presents some tools provided with the
  Triutils package. 
\end{itemize}

\begin{remark}
  As already pointed out, Epetra objects are meant to be the ``common
  language'' spoken by all the Trilinos packages, and therefore the new
  user must become familiar with those objects. Therefore we suggest to
  read Chapters \ref{chap:epetra_vec}-\ref{chap:epetra_others} before
  considering other Trilinos packages. Also, Chapter~\ref{chap:aztecoo}
  should be read before Chapters~\ref{chap:ifpack} and~\ref{chap:ml}
  (even if both IFPACK and ML can be compiled and run without AztecOO).
\end{remark}

This tutorial assume a basic background in numerical methods for PDEs,
and in iterative linear and nonlinear solvers. Although not strictly
necessary, the reader is suppose to have a certain familiarity with
distributed memory computing and, to a minor extent, with MPI.

\smallskip

Note that this tutorial is not a substitute ofr individual packages
documentation. Also, for an overview of all the Trilinos packages, the
Trilinos philosophy, and a description of the packages provided by
Trilinos, the reader is referred to \cite{Trilinos-Overview}.
Developers should also consider the Trilinos Developers' Guide, which
addresses many topics, including the development tools used by Trilinos'
developers, and how to include a new package\footnote{ Trilinos provides
  a variety of services to a developer wanting to integrate a package
  into Trilinos.  They include Autoconf~\cite{Autoconf},
  Automake~\cite{Automake} and Libtool~\cite{Libtool}. Those tools
  provide a robust, full-featured set of tools for building software
  across a broad set of platforms.  Although these tools are not
  official standards, they are widely used.  All existing Trilinos
  packages use Autoconf and Automake.  Libtool support will be added in
  future releases.}.

%%%
%%%
%%%

\subsection{Installing Trilinos}
\label{sec:installing}

To obtain Trilinos, please refers to the instructions reported at the
following web site:
\begin{verbatim}
http://software.sandia.gov/Trilinos
\end{verbatim}

Trilinos has been compiled on a variety of architectures, including
Linux, Sun Solaris, SGI Irix, DEC, and many others. Trilinos has been
designed to support parallel applications. However, it can be compiled
and run on serial computer.  Detailed comments on the installation, and
an exhaustive list of FAQs, can be found at the web pages:
\begin{verbatim}
http://software.sandia.gov/Trilinos/installing_manual.html
http://software.sandia.gov/Trilinos/faq.html
\end{verbatim}


Before using Trilinos, users might decide to set the environmental
variables \verb!TRILINOS_HOME!, indicating the full path of the Trilinos
directory, \verb!TRILINOS_LIB!, indicating the location of the compiled
Trilinos library, and \verb!TRILINOS_ARCH!, containing the architecture
and the communicator currently used.  For example, using the BASH shell,
command lines of the form
\begin{verbatim}
export TRILINOS_HOME=/home/msala/Trilinos
export TRILINOS_ARCH=LINUX.MPI
export TRILINOS_LIB=${TRILINOS_HOME}/${TRILINOS_ARCH}
\end{verbatim}
can be places in the users' \verb!.bashrc! file.

\smallskip

Here, we briefly report the procedure one should follow in order to
compile Trilinos as required by the examples reported in the following
chapters \ref{chap:epetra_vec}-\ref{chap:triutils}\footnote{Amesos can
  be more difficult to compile for the unexperienced user, as it
  required some information about the packages to interface. Suggestions
  about the configuration of Amesos are reported in
  Chapter~\ref{chap:amesos}. More details about the installation of
  Trilinos can be found in \cite{Trilinos-Users-Guide}.}.  Suppose we
want to compile under LINUX with MPI. The installation procedure can be
are reported below. (\verb!$! indicates the shell prompt.)
\begin{verbatim}
$ cd ${TRILINOS_HOME}
$ mkdir ${TRILINOS_ARCH}
$ cd ${TRILINOS_ARCH}
$ ../configure --prefix="${TRILINOS_HOME}/${TRILINOS_ARCH}" \
  --enable-mpi --with-mpi-compilers \
  --enable-triutils --enable-aztecoo \
  --enable-ifpack \
  --enable-ml --enable-nox | tee configure_${TRILINOS_ARCH}.log
$ make | tee make_${TRILINOS_ARCH}.log
$ make install | tee make_install_${TRILINOS_ARCH}.log
\end{verbatim}

\begin{remark}
  All Trilinos packages can be build to run with or without MPI. If MPI
  is enabled (using \verb!--enable-mpi!), the users must know the
  procedure for beginning MPI jobs on their computer system(s). In some
  cases, options must be set on the configure line to specify the
  location of MPI include files and libraries.
\end{remark}

%%%
%%%
%%%

\subsection{Compiling and Linking a program using Trilinos}
\label{sec:intro_compiling}

In order to compile and link (part of) the Trilinos library, the use can
decide to use a Makefile as reported below. This Makefile refers to one
of the examples, reported in the NOX subdirectory of this tutorial.
\begin{verbatim}
 1: TRILINOS_HOME = /home/msala/Trilinos/
 2: TRILINOS_ARCH - LINUX_MPI
 3: TRILINOS_LIB = $(TRILINOS_HOME)$(TRILINOS_ARCH)
 4: 
 5: include $(TRILINOS_HOME)/build/makefile.$(TRILINOS_ARCH)
 6: 
 7: MY_COMPILER_FLAGS = -DHAVE_CONFIG_H $(CXXFLAGS) -c -g\
 8:                    -I$(TRILINOS_LIB)/include/
 9:
10: MY_LINKER_FLAGS = $(LDFLAGS) $(TEST_C_OBJ) \
11:         -L$(TRILINOS_LIB)/lib/ \
12:         -lnoxepetra -lnox -lifpack \
13:         -laztecoo -lepetra -llapack -lblas $(ARCH_LIBS)
14:
15: ex1: ex1.cpp
16:         $(CXX)     ex1.cpp $(MY_COMPILER_FLAGS)
17:         $(LINKER)  ex1.o   $(MY_LINKER_FLAGS)    -o ex1.exe
\end{verbatim}

Line number have been reported for  reader's convenience. 

The lines 1-3 can be omitted, see Section \ref{sec:installing}.  Line 5
includes basic definitions of Trilinos. (Note that, on some
architectures, one may need to use \verb!gmake! instead of \verb!make!.)
In line 7, the variable \verb!HAVE_CONFIG_H! is defined. Linker flags of
lines 10-13 defines the library to link (location of BLAS and LAPACK can
change on different platforms). The variable \verb!ARCH_LIBS! is defined
in line 5.

To run the compiled example in a sequential environment, simply type
\begin{verbatim}
$ ./ex1.exe
\end{verbatim}
In a MPI environment, the user might have to
use an instruction of type
\begin{verbatim}
$ mpirun -np 2 ./ex1.exe
\end{verbatim}
Please check the local MPI documentation for more details. 

%%%
%%%
%%%

\subsection{Copyright and Licensing of Trilinos}
\label{sec:copyright}

Trilinos is released under the Lesser GPL GNU Licence.

Trilinos is copyrighted by Sandia Corporation. Under the terms of
Contract DE-AC04-94AL85000, there is a non-exclusive license for use of
this work by or on behalf of the U.S. Government.  Export of this
program may require a license from the United States Government.

NOTICE: The United States Government is granted for itself and others
acting on its behalf a paid-up, nonexclusive, irrevocable worldwide
license in ths data to reproduce, prepare derivative works, and perform
publicly and display publicly.  Beginning five (5) years from July 25,
2001, the United States Government is granted for itself and others
acting on its behalf a paid-up, nonexclusive, irrevocable worldwide
license in this data to reproduce, prepare derivative works, distribute
copies to the public, perform publicly and display publicly, and to
permit others to do so.

NEITHER THE UNITED STATES GOVERNMENT, NOR THE UNITED STATES DEPARTMENT
OF ENERGY, NOR SANDIA CORPORATION, NOR ANY OF THEIR EMPLOYEES, MAKES ANY
WARRANTY, EXPRESS OR IMPLIED, OR ASSUMES ANY LEGAL LIABILITY OR
RESPONSIBILITY FOR THE ACCURACY, COMPLETENESS, OR USEFULNESS OF ANY
INFORMATION, APPARATUS, PRODUCT, OR PROCESS DISCLOSED, OR REPRESENTS
THAT ITS USE WOULD NOT INFRINGE PRIVATELY OWNED RIGHTS.

\medskip

Some parts of Trilinos are dependent on a third party code. Each third
party code comes with its own copyright and/or licensing requirements.
It is responsibility of the user to understand these requirements.

%%%
%%%
%%%

\subsection{Programming Language Used in this Tutorial}
\label{sec:language}

Trilinos is written in C++ (for most packages), and in C. Some
interfaces are provided to FORTRAN code (mainly BLAS and LAPACK
routines). Even if a limited support is included for C programs (and a
more limited for FORTRAN code), to unleashed the full power of Trilinos
we suggest to use C++. All the example programs contained in this
tutorial will be in C++; some packages contains examples in C.

%%%
%%%
%%%

\subsection{Referencing Trilinos}
\label{sec:referencing}

The Trilinos project can be referenced by using the following BiBTeX
citation information:
\begin{verbatim}
@techreport{Trilinos-Overview,
title = "{An Overview of Trilinos}",
author = "Michael Heroux and Roscoe Bartlett and Vicki Howle
Robert Hoekstra and Jonathan Hu and Tamara Kolda and
Richard Lehoucq and Kevin Long and Roger Pawlowski and
Eric Phipps and Andrew Salinger and Heidi Thornquist and
Ray Tuminaro and James Willenbring and Alan Williams ",
institution = "Sandia National Laboratories",
number = "SAND2003-2927",
year = 2003}

@techreport{Trilinos-Dev-Guide,
title = "{Trilinos Developers Guide}",
author = "Michael A. Heroux and James M. Willenbring and Robert Heaphy",
institution = "Sandia National Laboratories",
number = "SAND2003-1898",
year = 2003}

@techreport{Trilinos-Dev-Guide-II,
title = "{Trilinos Developers Guide Part II: ASCI Software Quality
Engineering Practices Version 1.0}",
author = "Michael A. Heroux and James M. Willenbring and Robert Heaphy",
institution = "Sandia National Laboratories",
number = "SAND2003-1899",
year = 2003}

@techreport{Trilinos-Users-Guide,
title = "{Trilinos Users Guide}",
author = "Michael A. Heroux and James M. Willenbring",
institution = "Sandia National Laboratories",
number = "SAND2003-2952",
year = 2003}
\end{verbatim}
These BiBTeX information can be downloaded from the web page

\begin{verb}
http://software.sandia.gov/Trilinos/citing.html
\end{verb}

%%%
%%%
%%%

\subsection{A Note on Directory Structure}
\label{sec:into_note}

Each Trilinos package in contained in the subdirectory
\begin{verbatim}
${TRILINOS_HOME}/packages
\end{verbatim}
The structure of all packages is quite similar (although not exactly
equal). As a general line, source files are in
\begin{verbatim}
${TRILINOS_HOME}/packages/<package-name>/src
\end{verbatim}
Example files are reported in \begin{verbatim}
${TRILINOS_HOME}/packages/<package-name>/examples
\end{verbatim}
and test files in
\begin{verbatim}
${TRILINOS_HOME}/packages/<package-name>/test
\end{verbatim}
The documentation is reported
\begin{verbatim}
${TRILINOS_HOME}/packages/<package-name>/doc
\end{verbatim}
Often, Trilinos developers use Doxygen\footnote{Copyright \copyright
  1997-2003 by Dimitri van Heesch. More information can by found at the
  web address {\tt http://www.stack.nl/~dimitri/doxygen/}.}. For
instance, to create the documentation for Epetra, we use can type
\begin{verbatim}
$ cd ${TRILINOS_HOME}/packages/epetra/doc
$ doxygen Doxyfile
\end{verbatim}
and then browse it using an HTML reader, or compiling the \LaTeX file
using
\begin{verbatim}
$ cd ${TRILINOS_HOME}/packages/epetra/doc/latex
$ make
\end{verbatim}

%%%
%%%
%%%

\subsection{List of Trilinos Developers}
\label{sec:intro_incomplete}

A list of the Trilinos' developers, updated to December 2003, would
include the following names (in alphabetical order):

Roscoe A. Bartlett,
Jason A. Cross,
David M. Day,
Robert Heaphy,
Michael A. Heroux (project leader),
Russell Hooper,
Vicki E. Howle,
Robert J. Hoekstra,
Jonathan J. Hu,
Tamara G. Kolda,
Richard B. Lehoucq,
Paul Lin,
Kevin R. Long,
Roger P. Pawlowski,
Michael N. Phenow,
Eric T. Phipps,
Andrew J. Rothfuss,
Marzio Sala,
Andrew G. Salinger,
Paul M. Sexton,
Kendall S. Stanley,
Heidi K. Thornquist,
Ray S. Tuminaro,
James M. Willenbring,
Alan Williams.



    %-----------------------------%
    \chapter{Multigrid background}\label{sec:multigrid}
    %-----------------------------%
    \label{sec:multigrid intro}
Here we provide a brief multigrid introduction (see~\cite{MGTutorial}
or~\cite{OwlBook} for more information). A multigrid solver tries to approximate
the original problem of interest with a sequence of smaller (\textit{coarser})
problems. The solutions from the coarser problems are combined in order to
accelerate convergence of the original (\textit{fine}) problem on the finest
grid. A simple multilevel iteration is illustrated in
Algorithm~\ref{multigrid_code}.

\begin{algorithm}
\centering
\begin{algorithmic}[0]
  \State{$A_0 = A$}
  \Function{Multilevel}{$A_k$, $b$, $u$, $k$}
    \State{// Solve $A_k$ u = b (k is current grid level)}
    \State $ u = S^{1}_m (A_k, b, u)$
      \If{$(k \ne {\bf N-1})$}
        \State{$P_k = $ determine\_interpolant( $A_k$ )}
        \State{$R_k = $ determine\_restrictor( $A_k$ )}
        \State{$\widehat{r}_{k+1} = R_k (b - A_k u )$}
        \State{$A_{k+1} = R_k A_k P_k$}
        \State{$v = 0$}
        \State{}\Call{Multilevel}{$\widehat{A}_{k+1}$, $\widehat{r}_{k+1}$, $v$, $k+1$}
        \State{$ u = u + P_{k} v$}
        \State{$ u = S^{2}_m (A_k, b, u )$}
      \EndIf
  \EndFunction
\end{algorithmic}
\caption{V-cycle multigrid with $N$ levels to solve $Ax=b$.}
\label{multigrid_code}
\end{algorithm}

In the multigrid iteration in Algorithm~\ref{multigrid_code}, the $S^{1}_m()$'s
and $S^{2}_m()$'s are called \textit{pre-smoothers} and \textit{post-smoothers}.
They are approximate solvers (e.g., symmetric Gauss-Seidel), with the subscript
$m$ denoting the number of applications of the approximate solution method. The
purpose of a smoother is to quickly reduce certain error modes in the
approximate solution on a level $k$. For symmetric problems, the pre-
and post-smoothers should be chosen to maintain symmetry (e.g., forward
Gauss-Seidel for the pre-smoother and backward Gauss-Seidel for the
post-smoother). The $P_k$'s are \textit{interpolation} matrices that transfer
solutions from coarse levels to finer levels. The $R_k$'s are
\textit{restriction} matrices that restrict a fine level solution to a coarser
level. In a geometric multigrid, $P_k$'s and $R_k$'s are determined
by the application, whereas in an algebraic multigrid they are automatically
generated. For symmetric problems, typically $R_k=P_k^T$. For nonsymmetric
problems, this is not necessarily true. The $A_k$'s are the coarse level
problems, and are generated through a Galerkin (triple matrix) product.

Please note that the algebraic multigrid algorithms implemented in \muelu{}
generate the grid transfers $P_k$ automatically and the coarse problems $A_k$
via a sparse triple matrix product. \trilinos{} provides a wide selection of
smoothers and direct solvers for use in \muelu through the \ifpack,
\ifpacktwo, \amesos, and \amesostwo packages (see \S\ref{sec:options}).



    %-----------------------------%
    \chapter{Getting Started}\label{sec:getting started}
    %-----------------------------%
    This section is meant to get you using \ifpacktwo{} as quickly as possible.
\S\ref{sec:overview} gives a brief overview of \ifpacktwo{}.
\S\ref{sec:configuration_and_build} lists \ifpacktwo{}'s dependencies on other
\trilinos{} libraries and provides a sample cmake configuration line. Finally,
some examples of code are given in~\S\ref{sec:examples in code}.

\section{Overview of \ifpacktwo{}}
\label{sec:overview}
\ifpacktwo{} is a C++ linear solver library in the \trilinos{} project~\cite{Heroux2012}.
It originally began as a migration of \ifpack{} package capabilities to a new linear
algebra stack. While it retains some commonalities with the original package, it
has since diverged significantly from it and should be treated as completely
independent package.

\ifpacktwo{} only works with \tpetra{}~\cite{TpetraURL} matrix,
vector, and map types. Like \tpetra{}, it allows for different ordinal
(index) and scalar types. \ifpacktwo{} was designed to be efficient on a wide
range of computer architectures, from workstations to supercomputers~\cite{Lin2014}.
It relies on the ``MPI+X" principle, where ``X'' can be threading or
CUDA\@. The ``X'' portion, node-level parallelism, is controlled by a node
template type. Users should refer to \tpetra{}'s documentation for information
about node and device types.

\ifpacktwo provides a number of different solvers, including
\begin{itemize}
  \item Jacobi, Gauss-Seidel, polynomial, distributed relaxation;
  \item domain decomposition solvers;
  \item incomplete factorizations.
\end{itemize}
This list of solvers is not exhaustive. Instead, references for further
information are provided throughout the text. There are many excellent
references for iterative methods, including~\cite{Saad2003}.

Complete information on available capabilities and options can be found
in~\S\ref{sec:options}.

\section{Configuration and Build}\label{sec:configuration_and_build}

\ifpacktwo{} requires a C++11 compatible compiler for compilation. The
minimum required version of compilers are GCC (4.7.2 and later),
Intel (13 and later), and clang (3.5 and later).

\subsection{Dependencies}

Table~\ref{tab:dependencies} enumerates the dependencies of \ifpacktwo. Certain
dependencies are optional, whereas others are required.  Furthermore,
\ifpacktwo's tests depend on certain libraries that are not required if you only
want to link against the \ifpacktwo library and do not want to compile its
tests. Additionally, some functionality in \ifpacktwo{} may depend on other
Trilinos packages (for instance, \amesostwo{}) that may require additional
dependencies. We refer to the documentation of those packages for a full list of
dependencies.

\begin{table}[ht]
  \centering
  \begin{tabular}{p{3.5cm} c c c c}
    \toprule
    \multirow{2}{*}{Dependency} & \multicolumn{2}{c}{Library} & \multicolumn{2}{c}{Testing} \\
    \cmidrule(r){2-3} \cmidrule(l){4-5} & Required & Optional & Required & Optional  \\
    \midrule
    % \belos                       & $\times$ &          & $\times$ & \\
    \teuchos                     & $\times$ &          & $\times$ & \\
    \tpetra                      & $\times$ &          & $\times$ & \\
    \tpetrakernels               & $\times$ &          &          & \\
    \amesostwo                   &          & $\times$ &          & $\times$  \\
    \galeri                      &          &          &          & $\times$  \\
    \xpetra                      &          & $\times$ &          & $\times$  \\
    \zoltantwo                   &          & $\times$ &          & $\times$  \\
    \textsc{ThyraTpetraAdapters} &          & $\times$ &          & \\
    \textsc{ShyLUHTS}            &          & $\times$ &          & $\times$ \\
    \midrule
    % BLAS                         & $\times$ &          & $\times$ & \\
    % LAPACK                       & $\times$ &          & $\times$ & \\
    MPI                          &          & $\times$ &          & $\times$  \\
    % Cholmod                      &          & $\times$ &          & $\times$  \\
    % SuperLU 4.3                  &          & $\times$ &          & $\times$  \\
    % QD                           &          & $\times$ &          & $\times$  \\
    \bottomrule
  \end{tabular}
  \caption{\label{tab:dependencies}\ifpacktwo{}'s required and optional dependencies,
    subdivided by whether a dependency is that of the \ifpacktwo{}{} library itself
    (\textit{Library}), or of some \ifpacktwo{}{} test (\textit{Testing}). }
\end{table}

\amesostwo and \superlu are necessary if you want to use either a sparse direct
solve or ILUTP as a subdomain solve in processor-based domain decomposition.
\zoltantwo and \xpetra are necessary if you want to reorder a matrix (e.g.,
reverse Cuthill McKee).

\subsection{Configuration}
The preferred way to configure and build \ifpacktwo{} is to do that outside of the source directory.
Here we provide a sample configure script that will enable \ifpacktwo{} and all of its optional dependencies:
\begin{lstlisting}
  export TRILINOS_HOME=/path/to/your/Trilinos/source/directory
  cmake -D BUILD_SHARED_LIBS:BOOL=ON \
        -D CMAKE_BUILD_TYPE:STRING="RELEASE" \
        -D CMAKE_CXX_FLAGS:STRING="-g" \
        -D Trilinos_ENABLE_EXPLICIT_INSTANTIATION:BOOL=ON \
        -D Trilinos_ENABLE_TESTS:BOOL=OFF \
        -D Trilinos_ENABLE_EXAMPLES:BOOL=OFF \
        -D Trilinos_ENABLE_Ifpack2:BOOL=ON \
        -D Ifpack2_ENABLE_TESTS:STRING=ON \
        -D Ifpack2_ENABLE_EXAMPLES:STRING=ON \
        -D TPL_ENABLE_BLAS:BOOL=ON \
        -D TPL_ENABLE_MPI:BOOL=ON \
        ${TRILINOS_HOME}
\end{lstlisting}

\noindent
More configure examples can be found in \texttt{Trilinos/sampleScripts}.
For more information on configuring, see the \trilinos Cmake Quickstart guide \cite{TrilinosCmakeQuickStart}.

\section{Interface to \ifpacktwo{} methods}
All \ifpacktwo operators inherit from the base class
\texttt{Ifpack2::Preconditioner}. This in turn inherits from
\texttt{Tpetra::Operator}. Thus, you may use an \ifpacktwo operator anywhere
that a \texttt{Tpetra::Operator} works. For example, you may use \ifpacktwo operators
directly as preconditioners in \trilinos' \belos package of iterative solvers.

You may either create an \ifpacktwo operator directly, by using the class and
options that you want, or by using \texttt{Ifpack2::Factory}. Some of
\ifpacktwo preconditioners only accept a \texttt{Tpetra::\\CrsMatrix} instance as
input, while others also may accept a \texttt{Tpetra::RowMatrix} (the base class
of \texttt{Tpetra::CrsMatrix}). They will decide at run time whether the input
\texttt{Tpetra::RowMatrix} is an instance of the right subclass.

\texttt{Ifpack2::Preconditioner} includes the following methods:
\begin{itemize}
  \item \texttt{initialize()}

    Performs all operations based on the graph of the matrix (without
    considering the numerical values).

  \item \texttt{compute()}

    Computes everything required to apply the preconditioner, using the matrix's
    values.

  \item \texttt{apply()}

    Applies or ``solves with'' the preconditioner.
\end{itemize}
Every time that \texttt{initialize()} is called, the object destroys all the
previously allocated information, and reinitializes the preconditioner. Every
time \texttt{compute()} is called, the object recomputes the actual values of the
preconditioner.

An \ifpacktwo preconditioner may also inherit from
\texttt{Ifpack2::CanChangeMatrix} class in order to express that users can
change its matrix (the matrix that it preconditions) after construction using a
\texttt{setMatrix} method.  Changing the matrix puts the preconditioner back in
an ``pre-initialized'' state.  You must first call \texttt{initialize()}, then
\texttt{compute()}, before you may call \texttt{apply()} on this preconditioner.
Depending on the implementation, it may be legal to set the matrix to null. In
that case, you may not call \texttt{initialize()} or \texttt{compute()} until
you have subsequently set a nonnull matrix.

\textbf{Warning.} If you are familiar with the \ifpack package~\cite{ifpack}, please be aware
that the behaviour of the \ifpacktwo preconditioner is different from \ifpack.
In \ifpack, the \texttt{ApplyInverse()} method applies or ``solves with'' the
preconditioner $M^{-1}$, and the \texttt{Apply()} method ``applies'' the
preconditioner $M$. In \ifpacktwo, the \texttt{apply()} method applies or
``solves with'' the preconditioner $M^{-1}$. \ifpacktwo has no method comparable
to \ifpack's \texttt{Apply()}.

\section{Example: \ifpacktwo preconditioner within \belos}\label{sec:examples in code}

The most commonly used scenario involving \ifpacktwo{} is using one of its
preconditioners preconditioners inside an iterative linear solver. In
\trilinos{}, the \belos{} package provides important Krylov subspace methods (such
as preconditioned CG and GMRES).

At this point, we assume that the reader is comfortable with \teuchos{} referenced-counted
pointers (RCPs) for memory management (an introduction to RCPs can be found
in~\cite{RCP2010}) and the \parameterlist class~\cite{TeuchosURL}.

First, we create an \ifpacktwo{} preconditioner using a provided \parameterlist
\begin{lstlisting}[language=C++]
 typedef Tpetra::CrsMatrix<Scalar, LocalOrdinal, GlobalOrdinal, Node>
   Tpetra_Operator;

 Teuchos::RCP<Tpetra_Operator> A;
 // create A here ...
 Teuchos::ParameterList paramList;
 paramList.set( "chebyshev: degree", 1 );
 paramList.set( "chebyshev: min eigenvalue", 0.5 );
 paramList.set( "chebyshev: max eigenvalue", 2.0 );
 // ...
 Ifpack2::Factory factory;
 RCP<Ifpack2::Ifpack2Preconditioner<> > ifpack2Preconditioner;
 ifpack2Preconditioner = factory.create( "CHEBYSHEV", A )
 ifpack2Preconditioner->setParameters( paramList );
 ifpack2Preconditioner->initialize();
 ifpack2Preconditioner->compute();
\end{lstlisting}

Besides the linear operator $A$, we also need an initial guess vector for the
solution $X$ and a right hand side vector $B$ for solving a linear system.
\begin{lstlisting}[language=C++]
 typedef Tpetra::Map<LocalOrdinal, GlobalOrdinal, Node> Tpetra_Map;
 typedef Tpetra::MultiVector<Scalar, LocalOrdinal, GlobalOrdinal, Node>
   Tpetra_MultiVector;

 Teuchos::RCP<const Tpetra_Map> map = A->getDomainMap();

 // create initial vector
 Teuchos::RCP<Tpetra_MultiVector> X =
   Teuchos::rcp( new Tpetra_MultiVector(map, numrhs) );

 // create right-hand side
 X->randomize();
 Teuchos::RCP<Tpetra_MultiVector> B =
   Teuchos::rcp( new Tpetra_MultiVector(map, numrhs) );
 A->apply( *X, *B );
 X->putScalar( 0.0 );
\end{lstlisting}
To generate a dummy example, the above code first declares two vectors. Then, a
right hand side vector is calculated as the matrix-vector product of a random vector
with the operator $A$. Finally, an initial guess is initialized with zeros.

Then, one can define a \texttt{Belos::LinearProblem} object where the
\texttt{ifpack2Preconditioner} is used for left preconditioning.
\begin{lstlisting}[language=C++]
 typedef Belos::LinearProblem<Scalar, Tpetra_MultiVector, Tpetra_Operator>
   Belos_LinearProblem;

 Teuchos::RCP<Belos_LinearProblem> problem =
   Teuchos::rcp( new Belos_LinearProblem( A, X, B ) );
 problem->setLeftPrec( ifpack2Preconditioner );
 bool set = problem.setProblem();
\end{lstlisting}

Next, we set up a \belos{} solver using some basic parameters.
\begin{lstlisting}[language=C++]
 Teuchos::RCP<Teuchos::ParameterList> belosList =
   Teuchos::rcp(new Teuchos::ParameterList);
 belosList->set( "Block Size", 1 );
 belosList->set( "Maximum Iterations", 100 );
 belosList->set( "Convergence Tolerance", 1e-10 );
 belosList->set( "Output Frequency", 1 );
 belosList->set( "Verbosity", Belos::TimingDetails + Belos::FinalSummary );

 Belos::SolverFactory<Scalar, Tpetra_MultiVector, Tpetra_Operator> solverFactory;
 Teuchos::RCP<Belos::SolverManager<Scalar, Tpetra_MultiVector, Tpetra_Operator> >
   solver = solverFactory.create( "Block CG", belosList );
 solver->setProblem( problem );
\end{lstlisting}

Finally, we solve the system.
\begin{lstlisting}[language=C++]
 Belos::ReturnType ret = solver.solve();
\end{lstlisting}

It is often more convenient to specify the parameters as part of an XML-formatted options file.
Look in the subdirectory {\tt Trilinos/packages/ifpack2/test/belos} for examples of this.

This section is only meant to give a brief introduction on how to use
\ifpacktwo{} as a preconditioner within the \trilinos{} packages for iterative
solvers. There are other, more complicated, ways to use to work with
\ifpacktwo{}. For more information on these topics, the reader may refer to the
examples and tests in the \ifpacktwo{} source directory
(\texttt{Trilinos/packages/ifpack2}).


    %-----------------------------%
    \chapter{Performance tips}\label{sec:performance}
    %-----------------------------%
    \section{How to wring the last bit of performance out of Ifpack2 (jhu,csiefer)}
\section{Published results}
Cite the PPL paper \cite{Lin2014}.


    %-----------------------------%
    \chapter{\muelu{} options} \label{sec:options}
    %-----------------------------%
    \label{sec:muelu_options}

In this section, we report the complete list of \muelu{} input parameters.  It
is important to notice, however, that \muelu{} relies on other \trilinos{}
packages to provide support for some of its algorithms. For instance,
\ifpack{}/\ifpacktwo{} provide standard smoothers like Jacobi, Gauss-Seidel or
Chebyshev, while \amesos{}/\amesostwo{} provide access to direct solvers. The
parameters affecting the behavior of the algorithms in those packages are
simply passed by \muelu{} to a routine from the corresponding library. Please
consult corresponding packages' documentation for a full list of supported
algorithms and corresponding parameters.

\section{Using parameters on individual levels}
Some of the parameters that affect the preconditioner can in principle be
different from level to level. By default, parameters affect all levels in
a multigrid hierarchy.

The settings on a particular levels can be changed by using level sublists.
A level sublist is a \parameterlist{} sublist with the name ``level XX'', where XX is the level number. The
parameter names in the sublist do not require any modifications. For example,
the following fragment of code
\begin{lstlisting}[language=XML]
  <ParameterList name="level 2">
    <Parameter name="smoother: type" type="string" value="CHEBYSHEV"/>
  </ParameterList>
\end{lstlisting}
changes the smoother for level 2 to be a polynomial smoother.

\section{Parameter validation}
By default, \muelu{} validates the input parameter list. A parameter that is
misspelled, is unknown, or has an incorrect value type will cause an exception to be
thrown and execution to halt.

\begin{mycomment}
Spaces are important within a parameter's name. Please separate words
by just one space, and make sure there are no leading or trailing spaces.
\end{mycomment}

The option \verb|print initial parameters| prints the initial list given to the
interpreter. The option \verb|print unused parameters| prints the list of unused
parameters.

% ==================== GENERAL ====================
\section{General options}
\label{sec:options_general}

\begin{table}[h!]
  \begin{center}
    \begin{tabular}{p{3cm} p{12cm}}
      \toprule
      Verbosity level           & Description \\
      \midrule
      \verb!none!               & No output \\
      \verb!low!                & Errors, important warnings, and some statistics \\
      \verb!medium!             & Same as \verb!low!, but with more statistics \\
      \verb!high!               & Errors, all warnings, and all statistics \\
      \verb!extreme!            & Same as \verb!high!, but also includes output from other packages (\textit{i.e.}, \zoltan{}) \\
      \bottomrule
    \end{tabular}
    \caption{Verbosity levels.}
\label{t:verbosity_types}
  \end{center}
\end{table}

\begin{table}[h!]
  \begin{center}
    \begin{tabular}{p{4.3cm} p{4.3cm} c p{4.5cm}}
      \toprule
      Problem type               & Multigrid algorithm    & Block size  & Smoother \\
      \midrule
      \verb!unknown!             & --                     & --          & -- \\
      \verb!Poisson-2D!          & Smoothed aggregation   & 1           & Chebyshev \\
      \verb!Poisson-3D!          & Smoothed aggregation   & 1           & Chebyshev \\
      \verb!Elasticity-2D!       & Smoothed aggregation   & 2           & Chebyshev \\
      \verb!Elasticity-3D!       & Smoothed aggregation   & 3           & Chebyshev \\
      \verb!ConvectionDiffusion! & Petrov-Galerkin  AMG   & 1           & Gauss-Seidel \\
      \verb!MHD!                 & Unsmoothed aggregation & --          & Additive Schwarz method with one level of overlap and ILU(0) as a subdomain solver \\
      \bottomrule
    \end{tabular}
    \caption{Supported problem types (``--'' means not set).}
\label{t:problem_types}
  \end{center}
\end{table}


\cbb{problem: type}{string}{"unknown"}{Type of problem to be solved. Possible values: see Table~\ref{t:problem_types}.}
          
\cbb{verbosity}{string}{"high"}{Control of the amount of printed information. Possible values: see Table~\ref{t:verbosity_types}.}
          
\cbb{number of equations}{int}{1}{Number of PDE equations at each grid node. Only constant block size is considered.}
          
\cbb{max levels}{int}{10}{Maximum number of levels in a hierarchy.}
          
\cbb{cycle type}{string}{"V"}{Multigrid cycle type. Possible values: "V", "W".}
          
\cbb{problem: symmetric}{bool}{true}{Symmetry of a problem. This setting affects the construction of a restrictor. If set to true, the restrictor is set to be the transpose of a prolongator. If set to false, underlying multigrid algorithm makes the decision.}
          

% ==================== SMOOTHERS ====================
\section{Smoothing and coarse solver options}
\label{sec:options_smoothing}

\muelu{} relies on other \trilinos{} packages to provide level smoothers and
coarse solvers. \ifpack{} and \ifpacktwo{} provide standard smoothers (see
Table~\ref{tab:smoothers}), and \amesos{} and \amesostwo{} provide direct
solvers (see Table~\ref{tab:coarse}). Note that it is completely possible to use
any level smoother as a direct solver.

\muelu{} checks parameters \verb|smoother: * type| and \verb|coarse: type| to
determine:
\begin{itemize}
  \item what package to use (i.e., is it a smoother or a direct solver);
  \item (possibly) transform a smoother type

    \ding{42} \ifpack{} and \ifpacktwo{} use different smoother type names,
    e.g., ``point relaxation stand-alone'' vs ``RELAXATION''.  \muelu{} tries to follow
    \ifpacktwo{} notation for smoother types. Please consult \ifpacktwo{}
    documentation~\cite{Ifpack2URL} for more information.
\end{itemize}
The parameter lists \verb|smoother: * params| and \verb|coarse: params| are
passed directly to the corresponding package without any examination of their
content. Please consult the documentation of the corresponding packages for a list of
possible values.

By default, \muelu{} uses one sweep of symmetric Gauss-Seidel for both pre- and
post-smoothing, and SuperLU for coarse system solver.

\begin{table}[tbh]
  \begin{center}
    \begin{tabular}{p{4.0cm} p{10cm}}
      \toprule
      \texttt{smoother: type}           & \\
      \midrule
      \verb|RELAXATION|                 & Point relaxation smoothers, including
                                          Jacobi, Gauss-Seidel, symmetric Gauss-Seidel, etc. The exact
                                          smoother is chosen by specifying \texttt{relaxation: type} parameter in
                                          the \texttt{smoother: params} sublist. \\
      \verb|CHEBYSHEV|                  & Chebyshev polynomial smoother. \\
      \verb|ILUT|, \verb|RILUK|         & Local (processor-based) incomplete factorization methods. \\
      \bottomrule
    \end{tabular}
    \caption{Commonly used smoothers provided by \ifpack{}/\ifpacktwo{}. Note
    that these smoothers can also be used as coarse grid solvers.}
\label{tab:smoothers}
  \end{center}
\end{table}

\begin{table}[tbh]
  \begin{center}
    \begin{tabular}{p{4.0cm} c c p{7cm}}
      \toprule
      \texttt{coarse: type}             & \amesos{} & \amesostwo{} &  \\
      \midrule
      \verb|KLU|                        & x & & Default \amesos{} solver~\cite{klu}. \\
      \verb|KLU2|                       & & x & Default \amesostwo{} solver~\cite{amesos2_belos}. \\
      \verb|SuperLU|                    & x & x & Third-party serial sparse direct solver~\cite{Li2011}. \\
      \verb|SuperLU_dist|               & x & x & Third-party parallel sparse direct solver~\cite{Li2011}. \\
      \verb|Umfpack|                    & x & & Third-party solver~\cite{umfpack}. \\
      \verb|Mumps|                      & x & & Third-party solver~\cite{mumps}. \\
      \bottomrule
    \end{tabular}
    \caption{Commonly used direct solvers provided by \amesos{}/\amesostwo{}}
\label{tab:coarse}
  \end{center}
\end{table}


\cbb{smoother: pre or post}{string}{"both"}{Pre- and post-smoother combination. Possible values: "pre" (only pre-smoother), "post" (only post-smoother), "both" (both pre-and post-smoothers), "none" (no smoothing).}

\cba{smoother: type}{string}{Smoother type. Possible values: see Table~\ref{tab:smoothers}.}

\cba{smoother: pre type}{string}{Pre-smoother type. Possible values: see Table~\ref{tab:smoothers}.}

\cba{smoother: post type}{string}{Post-smoother type. Possible values: see Table~\ref{tab:smoothers}.}

\cba{smoother: params}{\parameterlist}{Smoother parameters. For standard smoothers, \muelu passes them directly to the appropriate package library.}

\cba{smoother: pre params}{\parameterlist}{Pre-smoother parameters. For standard smoothers, \muelu passes them directly to the appropriate package library.}

\cba{smoother: post params}{\parameterlist}{Post-smoother parameters. For standard smoothers, \muelu passes them directly to the appropriate package library.}

\cbb{smoother: overlap}{int}{0}{Smoother subdomain overlap.}

\cbb{smoother: pre overlap}{int}{0}{Pre-smoother subdomain overlap.}

\cbb{smoother: post overlap}{int}{0}{Post-smoother subdomain overlap.}

\cbb{coarse: max size}{int}{2000}{Maximum dimension of a coarse grid. \muelu will stop coarsening once it is achieved.}

\cbb{coarse: type}{string}{"SuperLU"}{Coarse solver. Possible values: see Table~\ref{tab:coarse}.}

\cba{coarse: params}{\parameterlist}{Coarse solver parameters. \muelu passes them directly to the appropriate package library.}

\cbb{coarse: overlap}{int}{0}{Coarse solver subdomain overlap.}


% ==================== AGGREGATION ====================
\section{Aggregation options}
\label{sec:options_aggregation}

\begin{table}[h!]
  \begin{center}
    \begin{tabular}{p{5.0cm} p{10cm}}
      \toprule
      \verb!uncoupled! & Attempts to construct aggregates of optimal size ($3^d$
                         nodes in $d$ dimensions). Each process works independently, and
                         aggregates cannot span multiple processes.\\
      \verb!coupled!   & Attempts to construct aggregates of optimal size ($3^d$
                         nodes in $d$ dimensions). Aggregates are allowed to
                         cross processor boundaries. \textbf{Use carefully}. If
                         unsure, use \verb!uncoupled! instead.\\
      %\verb!METIS!     & Use graph partitioning algorithm to create aggregates,
      %                   working process-wise. Number of nodes in each aggregate
      %                   is specified with option \texttt{aggregation: max agg
      %                   size}. \\
      % \verb!ParMETIS!  & As \verb!METIS!, but partition global graph. Aggregates
                         % can span arbitrary number of processes. Specify global
                         % number of aggregates with {\tt aggregation: global
                         % number}. \\
      \bottomrule
    \end{tabular}
    \caption{Available coarsening schemes. }
\label{t:aggregation}
  \end{center}
\end{table}


\cbb{aggregation: type}{string}{"uncoupled"}{Aggregation scheme. Possible values: see Table~\ref{t:aggregation}.}
          
\cbb{aggregation: mode}{string}{"uncoupled"}{Controls whether aggregates are allowed to cross processor boundaries. Possible values: "uncoupled" aggregates cannot cross processor boundaries.}
          
\cbb{aggregation: ordering}{string}{"natural"}{Node ordering strategy. Possible values: "natural" (local index order), "graph" (filtered graph breadth-first order), "random" (random local index order).}
          
\cbb{aggregation: phase 1 algorithm}{string}{"Distance2"}{Look at distance2 hops when aggregating.}
          
\cbb{aggregation: symmetrize graph after dropping}{bool}{false}{Symmetrize the graph used for aggregation.}
          
\cbb{aggregation: use blocking}{bool}{false}{Drop connections between different blocks. Blocks are given by a vector "BlockNumber" of scalar type LocalOrdinal that is passed via the "user data" list.}
          
\cbb{aggregation: drop scheme}{string}{"classical"}{Connectivity dropping scheme for a graph used in
      aggregation. Possible values: "point-wise" and "cut-drop". "signed classical sa", "classical", "distance laplacian", "signed classical", "block diagonal", "block diagonal classical", "block diagonal distance laplacian", "block diagonal signed classical", "block diagonal colored signed classical", "signed classical distance laplacian", "signed classical sa distance laplacian" are also allowed, but discouraged. Instead, the corresponding combination of "aggregation: strength-of-connection: matrix", "aggregation: strength-of-connection: measure" and "aggregation: drop scheme" should be used.}
          
\cbb{aggregation: strength-of-connection: matrix}{string}{"A"}{The matrix that is used for constructing the graph used for aggregation. Possible values: "A" (meaning the level matrix itself) and "distance laplacian".}
          
\cbb{aggregation: strength-of-connection: measure}{string}{"smoothed aggregation"}{The strength measure used for constructing the graph. Possible values: "smoothed aggregation", "signed smoothed aggregation", "signed ruge-stueben", "unscaled".}
          
\cbb{aggregation: distance laplacian metric}{string}{unweighted}{Metric used to compute the distance Laplacian. Possible values: "unweighted", "material"}
          
\cbb{aggregation: drop tol}{double}{0.0}{Connectivity dropping threshold for a graph used in aggregation.}
          
\cbb{aggregation: use ml scaling of drop tol}{bool}{false}{Enables ML-style scaling of drop tol, where the drop tol halves with each successive level.}
          
\cbb{aggregation: min agg size}{int}{2}{Minimum size of an aggregate.}
          
\cbb{aggregation: max agg size}{int}{-1}{Maximum size of an aggregate (-1 means unlimited).}
          
\cbb{aggregation: compute aggregate qualities}{bool}{false}{Whether to compute aggregate quality estimates.}
          
\cbb{aggregation: brick x size}{int}{2}{Number of points for x axis in "brick" aggregation (limited to 3).}
          
\cbb{aggregation: brick y size}{int}{2}{Number of points for y axis in "brick" aggregation (limited to 3).}
          
\cbb{aggregation: brick z size}{int}{2}{Number of points for z axis in "brick" aggregation (limited to 3).}
          
\cbb{aggregation: brick x Dirichlet}{bool}{false}{Asserts that Dirichlet conditions are applied in
        the x-direction and the Dirichlet DOFs are not aggregated.}
          
\cbb{aggregation: brick y Dirichlet}{bool}{false}{Asserts that Dirichlet conditions are applied in
        the y-direction and the Dirichlet DOFs are not aggregated.}
          
\cbb{aggregation: brick z Dirichlet}{bool}{false}{Asserts that Dirichlet conditions are applied in
        the z-direction and the Dirichlet DOFs are not aggregated.}
          
\cbb{aggregation: Dirichlet threshold}{double}{0.0}{Threshold for determining whether entries are zero during Dirichlet row detection.}
          
\cbb{aggregation: greedy Dirichlet}{bool}{false}{Force the aggregate to be Dirichlet if any DOFs in the aggregate are Dirichlet (default is aggregates are Dirichlet only if all DOFs in the aggregate are Dirichlet).}
          
\cbb{aggregation: deterministic}{bool}{false}{Boolean indicating whether or not aggregation will be run deterministically in the kokkos refactored path (only used in uncoupled aggregation).}
          
\cbb{aggregation: coloring algorithm}{string}{serial}{Choice of distance 2 independent set or coloring algorithm used by Uncoupled Aggregation, when using kokkos refactored aggregation. See Table \ref{t:coloring_algs} for more information.}
          
\cbb{aggregation: dropping may create Dirichlet}{bool}{true}{If true, any matrix row has nonzero off-diagonal entries will be treated as Dirichlet if aggregation dropping leaves only the diagonal entry.}
          
\cbb{aggregation: export visualization data}{bool}{false}{Export data for visualization post-processing.}
          
\cbb{aggregation: output filename}{string}{""}{Filename to write VTK visualization data to.}
          
\cbb{aggregation: output file: time step}{int}{0}{Time step ID for non-linear problems.}
          
\cbb{aggregation: output file: iter}{int}{0}{Iteration for non-linear problems.}
          
\cbb{aggregation: output file: agg style}{string}{Point Cloud}{Style of aggregate visualization.}
          
\cbb{aggregation: output file: fine graph edges}{bool}{false}{Whether to draw all fine node connections along with the aggregates.}
          
\cbb{aggregation: output file: coarse graph edges}{bool}{false}{Whether to draw all coarse node connections along with the aggregates.}
          
\cbb{aggregation: output file: build colormap}{bool}{false}{Whether to output a random colormap in a separate XML file.}
          
\cbb{aggregation: output file: aggregate qualities}{bool}{false}{Whether to plot the aggregate quality.}
          
\cbb{aggregation: output file: material}{bool}{false}{Whether to plot the material.}
          
\cbb{aggregation: mesh layout}{string}{Global Lexicographic}{Type of ordering for structured mesh aggregation. Possible values: "Global Lexicographic" and "Local Lexicographic".}
          
\cbb{aggregation: output type}{string}{Aggregates}{Type of object holding the aggregation data. Possible values: "Aggregates" or "CrsGraph".}
          
\cbb{aggregation: coarsening rate}{string}{{3}}{Coarsening rate per spatial dimensions, the string must be interpretable as an array by Teuchos.}
          
\cbb{aggregation: number of spatial dimensions}{int}{3}{The number of spatial dimensions in the problem.}
          
\cbb{aggregation: coarsening order}{int}{0}{The interpolation order used while constructing these aggregates, this value will be passed to the prolongator factory. There, possible values are 0 for piece-wise constant and 1 for piece-wise linear interpolation to transfer values from coarse points to fine points. }
          

% ==================== REBALANCING ====================
\section{Rebalancing options}
\label{sec:options_rebalancing}


\cbb{repartition: enable}{bool}{false}{Rebalancing on/off switch.}
          
\cbb{repartition: partitioner}{string}{"zoltan2"}{Partitioning package to use. Possible values: "zoltan" (\zoltan{} library), "zoltan2" (\zoltantwo{} library).}
          
\cba{repartition: params}{\parameterlist}{Partitioner parameters. \muelu passes them directly to the appropriate package library. In particular, this allows to choose a partitioning algorithm from \zoltan{} or \zoltan2{} or from external packages such as \parmetis{}.}
          
\cbb{repartition: start level}{int}{2}{Minimum level to run partitioner. \muelu does not rebalance levels finer than this one.}
          
\cbb{repartition: min rows per proc}{int}{800}{Minimum number of rows per MPI process. If the actual number if smaller, then rebalancing occurs. The value is not used if "repartition: min rows per thread" is positive.}
          
\cbb{repartition: target rows per proc}{int}{0}{Target number of rows per MPI process after rebalancing. If the value is set to 0, it will use the value of "repartition: min rows per proc"}
          
\cbb{repartition: min rows per thread}{int}{0}{Minimum number of rows per thread. If the actual number if smaller, then rebalancing occurs. If the value is set to 0, no repartitioning based on thread count will occur.}
          
\cbb{repartition: target rows per thread}{int}{0}{Target number of rows per thread after rebalancing. If the value is set to 0, it will use the value of "repartition: min rows per thread".}
          
\cbb{repartition: max imbalance}{double}{1.2}{Maximum nonzero imbalance ratio. If the actual number is larger, the rebalancing occurs.}
          
\cbb{repartition: remap parts}{bool}{true}{Postprocessing for partitioning to reduce data migration.}
          
\cbb{repartition: rebalance P and R}{bool}{false}{Explicit rebalancing of R and P during the setup. This speeds up the solve, but slows down the setup phases.}
          
\cbb{repartition: explicit via new copy rebalance P and R}{bool}{false}{Fully explicit rebalancing of R and P during setup that makes new copy and invokes fill copy.  Slowest rebalancing option, but needed by combine mg algo.}
          

% ==================== MULTIGRID ====================
\section{Multigrid algorithm options}
\label{sec:options_mg}

\begin{table}[h!]
  \begin{center}
    \begin{tabular}{p{3.5cm} p{11cm}}
      \toprule
      \verb!sa!         & Classic smoothed aggregation~\cite{VMB1996} \\
      \verb!unsmoothed! & Aggregation-based, same as \verb!sa! but without damped Jacobi prolongator improvement step \\
      \verb!pg!         & Prolongator smoothing using $A$, restriction smoothing using $A^T$, local damping factors~\cite{ST2008} \\
      \verb!emin!       & Constrained minimization of energy in basis functions of grid transfer operator~\cite{WTWG2014,OST2011} \\
      \bottomrule
    \end{tabular}
    \caption{Available multigrid algorithms for generating grid transfer matrices. }
\label{t:mgs}
  \end{center}
\end{table}


\cbb{multigrid algorithm}{string}{"sa"}{Multigrid method. Possible values: see Table~\ref{t:mgs}.}
          
\cbb{semicoarsen: coarsen rate}{int}{3}{Rate at which to coarsen unknowns in the z direction.}
          
\cbb{sa: damping factor}{double}{1.33}{Damping factor for smoothed aggregation.}
          
\cbb{sa: use filtered matrix}{bool}{true}{Matrix to use for smoothing the tentative prolongator. The two options are: to use the original matrix, and to use the filtered matrix with filtering based on filtered graph used for aggregation.}
          
\cbb{interp: interpolation order}{int}{1}{Interpolation order used to interpolate values from coarse points to fine points. Possible values are 0 for piece-wise constant interpolation and 1 for piece-wise linear interpolation. This parameter is set to 1 by default.}
          
\cbb{interp: build coarse coordinates}{bool}{true}{If false, skip the calculation of coarse coordinates.}
          
\cbb{filtered matrix: use lumping}{bool}{true}{Lump (add to diagonal) dropped entries during the construction of a filtered matrix. This allows user to preserve constant nullspace.}
          
\cbb{filtered matrix: reuse eigenvalue}{bool}{true}{Skip eigenvalue calculation during the construction of a filtered matrix by reusing eigenvalue estimate from the original matrix. This allows us to skip heavy computation, but may lead to poorer convergence.}
          
\cbb{emin: iterative method}{string}{"cg"}{Iterative method to use for energy minimization of initial prolongator in energy-minimization. Possible values: "cg" (conjugate gradient), "gmres" (generalized minimum residual), "sd" (steepest descent).}
          
\cbb{emin: num iterations}{int}{2}{Number of iterations to minimize initial prolongator energy in energy-minimization.}
          
\cbb{emin: num reuse iterations}{int}{1}{Number of iterations to minimize the reused prolongator energy in energy-minimization.}
          
\cbb{emin: pattern}{string}{"AkPtent"}{Sparsity pattern to use for energy minimization. Possible values: "AkPtent".}
          
\cbb{emin: pattern order}{int}{1}{Matrix order for the "AkPtent" pattern.}
          
\cbb{emin: use filtered matrix}{bool}{true}{Matrix to use for smoothing for energy minimization. The two options are: to use the original matrix, and to use the filtered matrix with filtering based on filtered graph used for aggregation.}
          

% ==================== REUSE ====================
\section{Reuse options}
\label{sec:options_reuse}

Reuse options are currently only used with \verb!sa! multigrid algorithm. We
also assume that the matrix preserves graph structure, and only matrix values
change.

\begin{table}[h!]
  \begin{center}
    \begin{tabular}{p{3.0cm} p{12cm}}
      \toprule
      \verb!none!   & No reuse \\
      \verb!emin!   & Reuse old prolongator as an initial guess to energy
                      minimization, and reuse the prolongator pattern \\
      \verb!RP!     & Reuse smoothed prolongator and restrictor. Smoothers are
                      recomputed.  \ding{42} \verb!RP! should reuse matrix graphs for
                      matrix-matrix product, but currently that is disabled as only \epetra{}
                      supports it. \\

      % \verb!tP!     & Reuse tentative prolongator. The graphs of smoothed prolongator and matrices in Galerkin product are reused only
                      % if filtering is not being used ({\it i.e.}, either \verb!sa: use filtered matrix! or \verb!aggregation: drop tol! is
                      % false) \\
      \verb!RAP!    & Recompute only the finest level smoothers, reuse all other operators \\
      \verb!full!   & Reuse everything \\
      \bottomrule
    \end{tabular}
    \caption{Available coarsening schemes. }
\label{t:reuse_types}
  \end{center}
\end{table}


\cbb{reuse: type}{string}{"none"}{Reuse options for consecutive hierarchy construction. This speeds up the setup phase, but may lead to poorer convergence. Possible values: see Table~\ref{t:reuse_types}.}


% ==================== MISCELLANEOUS ====================
\section{Miscellaneous options}


\cba{export data}{\parameterlist}{Exporting a subset of the hierarchy data in a
      file. Currently, the list can contain any of the following parameter
      names (``A'', ``P'', ``R'', ``Nullspace'', ``Coordinates'') of type \texttt{string}
      and value ``\{levels separated by commas\}''. A
      matrix/multivector with a name ``X'' is saved in two or three
      three MatrixMarket files: a) data is saved in
      \textit{X\_level.mm}; b) its row map is saved in
      \textit{rowmap\_X\_level.mm}; c) its column map (for matrices) is saved in
      \textit{colmap\_X\_level.mm}.}
          
\cbb{print initial parameters}{bool}{true}{Print parameters provided for a hierarchy construction.}
          
\cbb{print unused parameters}{bool}{true}{Print parameters unused during a hierarchy construction.}
          
\cbb{transpose: use implicit}{bool}{false}{Use implicit transpose for the restriction operator.}
          


    %-----------------------------%
    \chapter{\muemex: The MATLAB Interface for \muelu} \label{sec:muemex}
    %-----------------------------%
    %%%%%%%%%%%%%%%%%%%%%%%%%%%%%%%%%%%%%%%%%%%%%%%%%%%%%%%%%%%%%%%%%%%
\section{MUEMEX: The MATLAB Interface for MueLu} \label{sec:muemex}
%%%%%%%%%%%%%%%%%%%%%%%%%%%%%%%%%%%%%%%%%%%%%%%%%%%%%%%%%%%%%%%%%%%
MueMex is MueLu's interface to the MATLAB environment. It allows access
to a limited set of routines either MueLu as a preconditioner,
Belos as a solver and Epetra or Tpetra for data structures.
It is designed to provide access to MueLu's aggregation and
solver routines from MATLAB and does little else. MueMex allows users to
setup and solve arbitrarily many problems, so long as memory suffices.
More than one problem can be set up simultaneoulsy.

\subsection{Cmake Configure and Make}\label{sec:muemex:cmake}
To use MueMex, Trilinos must be configured with (at least) the
following options:

\begin{verbatim}
cmake \
-D Trilinos_ENABLE_Amesos:BOOL=ON \
-D Trilinos_ENABLE_Amesos2:BOOL=ON \
-D Amesos2_ENABLE_KLU2:BOOL=ON \
-D Trilinos_ENABLE_AztecOO:BOOL=ON \
-D Trilinos_ENABLE_Epetra:BOOL=ON \
-D Trilinos_ENABLE_EpetraExt:BOOL=ON \
-D Trilinos_ENABLE_Ifpack:BOOL=ON \
-D Trilinos_ENABLE_MueLu:BOOL=ON \
-D Trilinos_ENABLE_Teuchos:BOOL=ON \
-D Trilinos_ENABLE_Fortran:BOOL=OFF \
../Trilinos
\end{verbatim}

Most additional options can be specified as well.  It is important to
note that MueMex does not work properly with MPI, hence MPI must be
disabled in order to compile MueMex.  The MATLAB\_ARCH option is new to
the cmake build system, and involves the MATLAB-specific architecture
code for your system.  There is currently no automatic way to extract
this, so it must be user-specified.  As of MATLAB 7.9 (R2009b), common
arch codes are:
\begin{center}
\begin{tabular}{l|l}
Code& OS\\
\hline
glnx86& 32-bit Linux (intel/amd)\\
glnxa64& 64-bit Linux (intel/amd)\\
sol64& 64-bit Solaris(sparc)\\
sola64& 64-bit Solaris(intel/amd)\\
maci64& 64-bit MacOS\\
maci& 32-bit MacOS\\
\end{tabular}
\end{center}

On 64-bit Intel/AMD architectures, Trilinos and all relevant TPLs
(note: this includes BLAS and LAPACK)
must be compiled with the \texttt{-fPIC} option.  This necessitates adding:
\begin{verbatim}
-D CMAKE_CXX_FLAGS:STRING="-fPIC" \
-D CMAKE_C_FLAGS:STRING="-fPIC" \
-D CMAKE_Fortran_FLAGS:STRING="-fPIC" \
\end{verbatim}
to the cmake configure line.  Trilinos does not play nicely with
MATLAB's default LAPACK and BLAS on 64-bit machines, so be sure to
specify your own with something like:
\begin{verbatim}
-D LAPACK_LIBRARY_DIRS:STRING="<path to my lapack.a>" \
-D BLAS_LIBRARY_DIRS:STRING="<path to my blas.a>" \
\end{verbatim}
Using static linking for LAPACK and BLAS prevents MATLAB's default libraries to take precedence.
%
If MueMex randomly crashes when you turn 'PDE equations' above one,
chances are MueMex is finding the wrong BLAS/LAPACK libraries. Be sure
cmake is finding the right copy. Before starting MATLAB, set
LD_PRELOAD to the paths of libstdc++.so corresponding to the version of GCC used
to build Trilinos, and the paths of libblas.so and liblapack.so on your local system.

For example:

\begin{verbatim}
export LD_PRELOAD=/projects/install/rhel6-x86_64/sems/compiler/gcc/4.9.2/base/lib64/libstdc++.so:/usr/lib64/libblas.so:/usr/lib64/liblapack.so
\end{verbatim}

MueMex has only been tested with cmake 
on MATLAB 7.14 (R2012a) on a 64-bit Linux system.  The cmake build uses 
\texttt{mexext}, so any version of MATLAB before version 7.2 (R2006a) will
never work.  MueMex is also unlikely to work on Solaris.

\subsection{Using MueMex}\label{sec:muemex:usage}
MueMex is designed to be interfaced with via the MATLAB script
\texttt{muelu.m}.  There are five modes in which MueMex can be run:
\begin{enumerate}
\item Setup Mode --- Performs the problem setup for MueLu.
  Depending on whether or not the \texttt{Linear Algebra} option is
  used, MueMex creates either an unpreconditioned Epetra problem,
  an Epetra problem with MueLu, or a Tpetra problem with MueLu.
  The default is \texttt{epetra} for real matrices and \texttt{tpetra}
  for complex matrices. The \texttt{tpetra} option
  supports both real and complex and will infer the scalar type
  from the matrix passed during setup.  This call returns a problem 
  handle used to reference the problem in the future, and (optionally)
  the operator complexity, if a preconditioner is being used.
\item Solve Mode --- Given a problem handle and a right-hand side, MueMex
  solves the problem specified.  Setup mode must be called before
  solve mode.
\item Cleanup Mode --- Frees the memory allocated to internal MueLu,
  Epetra and Tpetra objects.  This can be called with a particular 
  problem handle, in which case it frees that problem, or without one, 
  in which case all MueMex memory is freed.
\item Status Mode --- Prints out status information on problems which
  have been set up.  Like cleanup, it can be called with or without a
  particular problem handle.
\end{enumerate}
All of these modes, with the exception of status and cleanup take
option lists which will be directly converted into
\texttt{Teuchos::ParameterList} objects by MueMex, as key-value pairs.
Options passed during setup will apply to the MueLu preconditioner, and
options passed during a solve will apply to Belos.

\subsubsection{Setup Mode}
Setup mode is called as follows:
\begin{verbatim}
>> [h, oc] = muelu('setup', A[, 'parameter', value,...]) 
\end{verbatim}
The parameter \texttt{A} represents the sparse matrix to perform aggregation on
and the parameter/value pairs represent standard MueLu options.

The routine returns a problem handle, \texttt{h}, and the operator
complexity \texttt{oc} for the operator.  In addition to the standard
options, setup mode has one unique option of its own:

\choicebox{\tt Linear Algebra}{[{\tt string}] Whether to use
  'epetra unprec', 'epetra', or 'tpetra'. Default is 'epetra' for
  real matrix and 'tpetra' for complex matrix.}

\subsubsection{Solve Mode}
Solve mode is called as follows:
\begin{verbatim}
>> [x, its] = muelu(h[, A], b[, 'parameter', value,...])
\end{verbatim}
The parameter \texttt{h} is a problem handle returned by the
setup mode call, \texttt{A} is the sparse matrix with which to
solve and \texttt{b} is the right-hand side.  Parameter/value pairs
to configure the Belos solver are listed as above. If A is not supplied,
the matrix provided when setting up the problem will be used. \texttt{x} is
the solution multivector with the same dimensions as \texttt{b}, and \texttt{its}
is the number of iterations Belos needed to solve the problem.

All of these options are taken directly from Belos, so consult its
manual for more information. Belos output style and verbosity settings
are implemented as enums, but can be set by name in MueMex. For example:

\begin{verbatim}
> x = muelu(0, b, 'Verbosity', 'Warnings + IterationDetails', 'Output Style', 'Brief');
\end{verbatim}

Verbosity settings can be separated by spaces, '+' or ','. Belos::Brief
is the default output style.

\subsubsection{Cleanup Mode}
Cleanup mode is called as follows:
\begin{verbatim}
>>  muelu('cleanup'[, h])
\end{verbatim}
The parameter \texttt{h} is a problem handle returned by the
setup mode call and is optional.  If \texttt{h} is provided, that
problem is cleaned up.  If the option is not provided all currently
set up problems are cleaned up.

\subsubsection{Status Mode}
Status mode is called as follows:
\begin{verbatim}
>>  muelu('status'[, h])
\end{verbatim}
The parameter \texttt{h} is a problem handle returned by the
setup mode call and is optional.  If \texttt{h} is provided, status
information for that problem is printed.  If the option is not provided all currently
set up problems have status information printed.

\subsection{Tips and Tricks }\label{sec:muemex:tips}

Internally, MATLAB represents all data as doubles unless you go
through efforts to do otherwise.  MueMex detects integer parameters by
a relative error test, seeing if the relative difference between the
value from MATLAB and the value of the \texttt{int}-typecast value are
less than 1e-15.  Unfortunately, this means that MueMex will choose the
incorrect type for parameters which are doubles that happen to have an
integer value (a good example of where this might happen would be the parameter
`smoother Chebyshev: alpha', which defaults to 30.0).  Since MueMex does no
internal typechecking of
parameters (it uses MueLu's internal checks), it has no way of detecting
this conflict.  From the user's perspective, avoiding this is as
simple as adding a small perturbation (greater than a relative 1e-15)
to the parameter that makes it non-integer valued.


    %-----------------------------%
    %\chapter{YAML Parameter Lists}\label{sec:yaml}
    %-----------------------------%
    %YAML is a human-readable data serialization format. MueLu provides a
YAML parameter list interpreter. It produces Teuchos::ParameterList
objects equivalent to those produced by the Teuchos XML helper functions.

Here is a simple example XML parameter list:
\begin{verbatim}
<ParameterList>
  <ParameterList Input>
    <Parameter name="values" type="Array(double)" value="{54.3 -4.5 2.0}"/>
    <Parameter name="myfunc" type="string" value="
def func(a, b):
  return a * 2 - b"/>
  </ParameterList>
  <ParameterList Solver>
    <Parameter name="iterations" type="int" value="5"/>
    <Parameter name="tolerance" type="double" value="1e-7"/>
    <Parameter name="do output" type="bool" value="true"/>
    <Parameter name="output file" type="string" value="output.txt"/>
  </ParameterList>
</ParameterList>
\end{verbatim}

Here is an equivalent YAML parameter list:
\begin{verbatim}
%YAML 1.1
---
ANONYMOUS:
  Input:
    values: [54.3, -4.5, 2.0]
    myfunc: |-

      def func(a, b):
        return a * 2 - b
  Solver:
    iterations: 5
    tolerance: 1e-7
    do output: yes
    output file: output.txt
...
\end{verbatim}

The nested structure and key-value pairs of these two lists are identical.
To a program querying them for settings, they are indistinguishable.

These are the general rules for creating a YAML parameter list:
\begin{itemize}
\item First line must be ``\%YAML 1.1'', second must be ``---'', and last must be ``...''
\item Nested map structure is determined by indentation. SPACES ONLY, NO TABS!
\item As with the above example, for a top-level anonymous parameter list, ``ANONYMOUS:'' must be explicit
\item Type is inferred. 5 is an int, 5.0 is a double, and '5.0' is a string
\item Quotation marks (single or double) are optional for strings, but required for strings with special characters: \verb.:{}[],&*#?|-<>=!%@\.
\item Quotation marks also turn non-string types into strings: '3' is a string
\item As with XML parameter lists, keys are regular strings
\item Even though YAML supports several names for bool true/false, only ``true'' and ``false'' are supported by the parameter list reader.
\item Arrays of int, double and string supported. exampleArray: {[}hello, world, goodbye{]}
\item {[}3, 4, 5{]} is an int array, {[}3, 4, 5.0{]} is a double array, and {[}3, '4', 5.0{]} is a string array
\item For multi-line strings, place ``$|-$'' after the ``key:'' and then indent the string one level deeper than the key
\item To preserve indentation in a multiline string, place ``$|2-$'' and then indent your string's content by 2 spaces relative to the key.
\end{itemize}


    %\nocite{*}

    % ---------------------------------------------------------------------- %
    % References
    %
    \clearpage
    % If hyperref is included, then \phantomsection is already defined.
    % If not, we need to define it.
    \providecommand*{\phantomsection}{}
    \phantomsection
    \addcontentsline{toc}{chapter}{References}
    \bibliographystyle{plain}
    \bibliography{mueluguide}


    % ---------------------------------------------------------------------- %
    %
    \appendix
    \chapter{Copyright and License}
    \input{license}
    %\chapter{Historical Perspective}
	%    This is an example of an appendix.

    If we follow~\cite{Sand98-0730} strictly, we would have to have
    a separate bibliography section for each appendix.  The style
    file doesn't provide that, but it can be done using the {\tt
    bibunits} and {\tt chapterbib} packages.

    If there are many subsections in an appendix, it should also
    have its own table of contents. Again, the SAND report class
    file does not provide that functionality.

    \ifthenelse{\boolean{reportSAND}}   {
	\section{The Past a Long Time Ago}
    }{
	\subsection{The Past a Long Time Ago}
    }
	This is where we talk about things so old nobody can verify
	them. We are safe.

    \ifthenelse{\boolean{reportSAND}}   {
	\section{The Past More Recently}
    }{
	\subsection{The Past More Recently}
    }
	Now we have to be a little bit more careful, since records
	exist from that time, and some people still alive actually
	lived back then.


    \chapter{ML compatibility}
     
\label{sec:ml_options}
\muelu provides a basic compatibility layer for \ml parameter lists. This allows \ml users to quickly perform some experiments with \muelu. Long term, we would like to have users use the new \muelu interface, as that is where most of new features will be made accessible. One should make note of the fact that it may not be possible to make ML deck do exactly same things in \ml and \muelu, as internally \ml implicitly makes some decision that we have no control over and which could be different from \muelu.

\noindent There are basically two distinct ways to use \ml input parameters with \muelu:
\begin{description}
\item[MLParameterListInterpreter:] This class is the pendant of the \texttt{ParameterListInterpreter} class for the \muelu parameters. It accepts parameter lists or XML files with \ml parameters and generates a \muelu multigrid hierarchy. It supports only a well-defined subset of \ml parameters which have a equivalent parameter in \muelu.
\item[ML2MueLuParameterTranslator:] This class is a simple wrapper class which translates \ml parameters to the corresponding \muelu parameters. It has to be used in combination with the \muelu \texttt{ParameterListInterpreter} class to generate a \muelu multigrid hierarchy. It is also meant to be used in combination with the \texttt{CreateEpetraPreconditioner} and \texttt{CreateTpetraPreconditioner} routines (see \S\ref{sec:examples in code}). It supports only a small subset of the \ml parameters.
\end{description}

\section{Compatible \ml parameters}

\subsection{General \ml options}

\mlcbb{ML output}{int}{0}{MLParameterListInterpreter, ML2MueLuParameterTranslator}{Control of the amount of printed information. Possible values: 0-10 with 0=no output and 10=maximum verbosity.}   
      
\mlcbb{PDE equations}{int}{1}{MLParameterListInterpreter, ML2MueLuParameterTranslator}{Number of PDE equations at each grid node. Only constant block size is considered.}   
       
\mlcbb{max levels}{int}{10}{MLParameterListInterpreter, ML2MueLuParameterTranslator}{Maximum number of levels in a hierarchy.}   
      
\mlcbb{prec type}{string}{"MGV"}{MLParameterListInterpreter, ML2MueLuParameterTranslator}{Multigrid cycle type. Possible values: "MGV", "MGW". Other values are NOT supported by MueLu.}   
      

\subsection{Smoothing and coarse solver options}

\mlcbb{smoother: type}{string}{"symmetric Gauss-Seidel"}{MLParameterListInterpreter, ML2MueLuParameterTranslator}{Smoother type for fine- and intermedium multigrid levels. Possible values: "Jacobi", "Gauss-Seidel", "symmetric Gauss-Seidel", "Chebyshev", "ILU".}

\mlcbb{smoother: sweeps}{int}{2}{MLParameterListInterpreter, ML2MueLuParameterTranslator}{Number of smoother sweeps for relaxation based level smoothers. In case of Chebyshev smoother it denotes the polynomial degree.}

\mlcbb{smoother: damping factor}{double}{1.0}{MLParameterListInterpreter, ML2MueLuParameterTranslator}{Damping factor for relaxation based level smoothers.}

\mlcbb{smoother: Chebyshev alpha}{double}{20}{MLParameterListInterpreter, ML2MueLuParameterTranslator}{Eigenvalue ratio for Chebyshev level smoother.}

 
\mlcbb{smoother: pre or post}{string}{"both"}{MLParameterListInterpreter, ML2MueLuParameterTranslator}{Pre- and post-smoother combination. Possible values: "pre" (only pre-smoother), "post" (only post-smoother), "both" (both pre-and post-smoothers).}   
      
\mlcbb{max size}{int}{128}{MLParameterListInterpreter, ML2MueLuParameterTranslator}{Maximum dimension of a coarse grid. \ml will stop coarsening once it is achieved.}   
      

\mlcbb{coarse: type}{string}{"Amesos-KLU"}{MLParameterListInterpreter, ML2MueLuParameterTranslator}{Solver for coarsest level. Possible values: "Amesos-KLU", "Amesos-Superlu" (depending on \muelu installation).}


\subsection{Transfer operator options}

\mlcbb{energy minimization: enable}{int}{0}{MLParameterListInterpreter, ML2MueLuParameterTranslator}{Enable energy minimization transfer operators (using Petrov-Galerkin approach).}
          
\mlcbb{aggregation: damping factor}{double}{1.33}{MLParameterListInterpreter, ML2MueLuParameterTranslator}{Damping factor for smoothed aggregation.}
          

\subsection{Rebalancing options}

\mlcbb{repartition: enable}{int}{0}{MLParameterListInterpreter}{Rebalancing on/off switch. Only limited support for repartitioning. Does not use provided node coordinates.}   
      
\mlcbb{repartition: start level}{int}{1}{MLParameterListInterpreter}{Minimum level to run partitioner. \muelu does not rebalance levels finer than this one.}   
      
\mlcbb{repartition: min per proc}{int}{512}{MLParameterListInterpreter}{Minimum number of rows per MPI process. If the actual number if smaller, then rebalancing occurs.}   
      
\mlcbb{repartition: max min ratio}{double}{1.3}{MLParameterListInterpreter}{Maximum nonzero imbalance ratio. If the actual number is larger, the rebalancing occurs.}   
      



    %\chapter{Some Other Appendix}
	%    Just to show what a second Appendix would look like. It contains
    a table. Each appendix is supposed to be self-contained, so
    tables and figures are not supposed to show up in the main
    table of contents. There can be a separate table of contents
    for each appendix.

    \begin{table}[ht]
	\centering
	\caption{A small table}
	\bigskip

	\begin{tabular}{|c|c|}
	    \hline
		A & B  \\ \hline
		C & D  \\ \hline
	\end{tabular}
	\label{tab3}
    \end{table}

    \begin{figure}[ht]
	\centering
	\begin{picture}(50,50)(0,0)
	    \put(25,25){\circle{1}}
	    \put(25,25){\circle{5}}
	    \put(25,25){\circle{10}}
	    \put(25,25){\circle{15}}
	    \put(25,25){\circle{20}}
	    \put(25,25){\circle{25}}
	    \put(25,25){\circle{30}}
	    \put(25,25){\circle{35}}
	    \put(25,25){\circle{40}}
	    \put(25,25){\circle{45}}
	    \put(25,25){\circle{50}}
	\end{picture}
	\caption{Dizzy yet?}
	\label{fig4}
    \end{figure}


    % \printindex

    %
% This is an example of how to create the distribution page. Some
% distributions are required by Sandia; e.g. the housekeeping copies.
% Depending on the type of report; e.g. CRADA, Patent Caution, etc.
% additional distribution lines may have to be added. See the
% "Guide for Preparing SAND Reports"
%
% SANDdistribution takes CA or NM as an optional argument. If given,
% the approrpiate housekeeping copies are inserted autmatically.
% Inside the SANDdistribution environment, several commands can be used
% insert the distributions for CRADA, LDRD, etc. See example below.
%
% You can leave the CA or NM option off and not use any of the SANDdist*
% commands. This will allow you to create a distribution list manually.
%
\begin{SANDdistribution}[NM]
    % Housekeeping copies necessary for every unclassified report:
    % \SANDdistCRADA	% If this report is about CRADA work
    % \SANDdistPatent	% If this report has a Patent Caution or Patent Interest
    % \SANDdistLDRD	% If this report is about LDRD work

    % Some external Addresses
    \SANDdistExternal{1}{An Address\\ 99 $99^{th}$ street NW\\City, State}
    \SANDdistExternal{3}{Some Address\\ and street\\City, State}
    \SANDdistExternal{12}{Another Address\\ On a street\\City, State\\U.S.A.}
    \bigskip


    % The following MUST BE between the external and internal distributions!
    % \SANDdistClassified % If this report is classified


    % Internal Addresses
    \SANDdistInternal{1}{1319}{Rolf Riesen}{1423}
    \SANDdistInternal{1}{1110}{Another One}{01400}

    % Example of a mail channel use (instead of a mail stop)
    \SANDdistInternalM{1}{M9999}{Someone}{01234}

\end{SANDdistribution}


\end{document}

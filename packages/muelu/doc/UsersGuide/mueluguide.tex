%
% $Id: SANDExampleReportNotstrict.tex,v 1.26 2009-05-01 20:59:19 rolf Exp $
%
% This is an example LaTeX file which uses the SANDreport class file.
% It shows how a SAND report should be formatted, what sections and
% elements it should contain, and how to use the SANDreport class.
% It uses the LaTeX report class, but not the strict option.
%
% Get the latest version of the class file and more at
%    http://www.cs.sandia.gov/~rolf/SANDreport
%
% This file and the SANDreport.cls file are based on information
% contained in "Guide to Preparing {SAND} Reports", Sand98-0730, edited
% by Tamara K. Locke, and the newer "Guide to Preparing SAND Reports and
% Other Communication Products", SAND2002-2068P.
% Please send corrections and suggestions for improvements to
% Rolf Riesen, Org. 9223, MS 1110, rolf@cs.sandia.gov
%
\documentclass[pdf,12pt,report]{SANDreport}
\usepackage{algpseudocode}
\usepackage{amsthm}
\usepackage{booktabs}
\usepackage{calc}
\usepackage{color}
\usepackage[table]{xcolor}
\usepackage{eso-pic}
\usepackage{fancyhdr}
\usepackage{float}
\usepackage{ifthen}
\usepackage{indentfirst}
\usepackage{geometry}
\usepackage{graphicx}
\usepackage[colorlinks, bookmarksopen, %pagebackref=true, backref=page,
             linkcolor={blue},
             anchorcolor={black},
             citecolor={blue},
             filecolor={magenta},
             menucolor={blue},
             pagecolor={red},
             plainpages=false,pdfpagelabels,
             pdfauthor={Luc Berger-Vergiat, Christian A. Glusa, Jonathan J. Hu, Andrey Prokopenko, Christopher M. Siefert, Raymond S. Tuminaro, Tobias A. Wiesner},
             pdftitle={MueLu User's Guide},
             pdfkeywords={MueLu,AMG,multigrid,guide,user},
             urlcolor={blue}]{hyperref}
\usepackage{listings}
\usepackage{mathptmx}	% Use the Postscript Times font
\usepackage{multirow}
\usepackage{pifont}
\usepackage[FIGBOTCAP,normal,bf,tight]{subfigure}
\usepackage{tabularx}
\usepackage{verbatim}
\usepackage{xspace}
\usepackage{flowchart} % also loads tikz
\usepackage{algorithm}
\usetikzlibrary{arrows}

%\usepackage{draftwatermark}
%\SetWatermarkScale{.5}

\algrenewcommand{\algorithmiccomment}[1]{\hskip3em // #1}
\newcommand{\monthWord}{\ifcase \month \or January\or February\or March\or April\or May%
\or June\or July\or August\or September\or October\or November\or December\fi}


% If you want to relax some of the SAND98-0730 requirements, use the "relax"
% option. It adds spaces and boldface in the table of contents, and does not
% force the page layout sizes.
% e.g. \documentclass[relax,12pt]{SANDreport}
%
% You can also use the "strict" option, which applies even more of the
% SAND98-0730 guidelines. It gets rid of section numbers which are often
% useful; e.g. \documentclass[strict]{SANDreport}



% ---------------------------------------------------------------------------- %
%
% Set the title, author, and date
%
\title{MueLu User's Guide}


\author{
  Luc Berger-Vergiat\\
  Computational Mathematics\\
  Sandia National Laboratories\\
  Mailstop 1320, P.O.~Box 5800 \\
  Albuquerque, NM 87185-1320\\
  lberge@sandia.gov
  \and
  Christian A. Glusa \\
  Scalable Algorithms\\
  Sandia National Laboratories\\
  Mailstop 1318, P.O.~Box 5800 \\
  Albuquerque, NM 87185-1318\\
  caglusa@sandia.gov
  \and
  Graham Harper\\
  Computational Mathematics\\
  Sandia National Laboratories\\
  Mailstop 1320, P.O.~Box 5800 \\
  Albuquerque, NM 87185-1320\\
  gbharpe@sandia.gov
  \and
  Jonathan J. Hu \\
  Scalable Algorithms \\
  Sandia National Laboratories\\
  Mailstop 9060, P.O.~Box 0969 \\
  Livermore, CA 94551-0969\\
  jhu@sandia.gov
  \and
  Matthias Mayr \\
  Institute for Mathematics \\
  and Computer-Based Simulation\\
  University of the Bundeswehr Munich\\
  Werner-Heisenberg-Weg 39\\
  85577 Neubiberg, Germany\\
  matthias.mayr@unibw.de
  \and
  Peter Ohm\\
  Complex Phenomena Unified Simulation Research Team \\
  RIKEN Center for Computational Science \\
  7 Chome-1-26 Minatojima Minamimachi, \\
  Chuo Ward, Kobe, Hyogo 650-0047, Japan \\
  peter.ohm@riken.jp
  \and
  Andrey Prokopenko \\
  Oak Ridge National Laboratory\\
  P.O.~Box 2008\\
  Bldg 5700, MS 6164\\
  Oak Ridge, TN 37831\\
  \and
  Christopher M. Siefert\\
  Scalable Algorithms\\
  Sandia National Laboratories\\
  Mailstop 1322, P.O.~Box 5800 \\
  Albuquerque, NM 87185-1322\\
  csiefer@sandia.gov
  \and
  Raymond S. Tuminaro\\
  Computational Mathematics\\
  Sandia National Laboratories\\
  Mailstop 9060, P.O.~Box 0969 \\
  Livermore, CA 94551-0969\\
  rstumin@sandia.gov
  \and
  Tobias Wiesner \\
  Leica Geosystems AG\\
  Heinrich-Wild-Strasse 201\\
  9435 Heerbrugg, Switzerland\\
  tobias.wiesner@leica-geosystems.com
}

% There is a "Printed" date on the title page of a SAND report, so
% the generic \date should generally be empty.
\date{}

\input{definitions}

\newtheorem*{mycomment}{\ding{42}}
\newtheoremstyle{plain}
  {\topsep}   % ABOVESPACE
  {\topsep}   % BELOWSPACE
  {\normalfont}  % BODYFONT
  {0pt}       % INDENT (empty value is the same as 0pt)
  {\bfseries} % HEADFONT
  {}         % HEADPUNCT
  {5pt plus 1pt minus 1pt} % HEADSPACE
  {}          % CUSTOM-HEAD-SPEC

% further declarations and additional commands
\definecolor{hellgelb}{rgb}{1,1,0.8}   % background color for C++ listings
\definecolor{darkgreen}{rgb}{0.0, 0.2, 0.13}
%\definecolor{hellrot}{HTML}{FFA4C2}    % background color for xml files
\definecolor{SANDgreen}{RGB}{163, 213, 199}

% settings for listings package
\lstset{
  backgroundcolor=\color{hellgelb},
  basicstyle=\ttfamily\small,
  breakautoindent=true,
  breaklines=true,
  captionpos=b,
  columns=flexible,
  commentstyle=\color{darkgreen},
  extendedchars=true,
  float=hbp,
  frame=single,
  identifierstyle=\color{black},
  keywordstyle=\color{blue},
  numbers=none,
  numberstyle=\tiny,
  showspaces=false,
  showstringspaces=false,
  stringstyle=\color{purple},
  tabsize=2,
}


% ---------------------------------------------------------------------------- %
% Set some things we need for SAND reports. These are mandatory
%
\SANDnum{SAND2023-12265}
\SANDprintDate{February 2023}
\SANDauthor{Luc Berger-Vergiat, Christian A. Glusa, Graham Harper, Jonathan J. Hu, Matthias Mayr, Peter Ohm, Andrey Prokopenko, Christopher M. Siefert, Raymond S. Tuminaro, Tobias A. Wiesner}


% ---------------------------------------------------------------------------- %
% Include the markings required for your SAND report. The default is "Unlimited
% Release". You may have to edit the file included here, or create your own
% (see the examples provided).
%
% \include{MarkUR} % Not needed for unlimted release reports


% ---------------------------------------------------------------------------- %
% The following definition does not have a default value and will not
% print anything, if not defined
%
%\SANDsupersed{SAND1901-0001}{January 1901}
%\input{MarkOUO}


% ---------------------------------------------------------------------------- %
%
% Start the document
%
\begin{document}
    \maketitle

    % ------------------------------------------------------------------------ %
    % An Abstract is required for SAND reports
    %
    \begin{abstract}
	%This is the definitive user guide for the \muelu{} library in Trilinos version XX.YY.
%\muelu{} is a C++ multigrid framework that can work with either the \epetra or \tpetra linear
%algebra libraries.
%\muelu{} provides a variety of aggregation-based multigrid algorithms,
%including smoothed aggregation algebraic multigrid (AMG), Petrov-Galerkin AMG, and AMG for
%Maxwell's equations, as well as many different types of smoothers.
%\muelu{} is templated on the index, scalar, and compute node types.
%Thus it is possible to use \muelu{} on problems with scalar types other than double, on very
%large problems, and to exploit node-level parallelism.

This is the official user guide for the \muelu{} multigrid library in Trilinos VOTD,
\monthWord, \the\year. This guide provides an overview of \muelu, its capabilities, and
instructions for new users who want to start using \muelu{} with a minimum of
effort. Detailed information is given on how to drive \muelu{} through its XML
interface. Links to more advanced use cases are given. This guide gives
information on how to achieve good parallel performance, as well as how to
introduce new algorithms. Finally, readers will find a comprehensive listing of
available \muelu{} options.  {\em Any options not documented in this manual
should be considered strictly experimental.}

%%% Local Variables:
%%% mode: latex
%%% TeX-master: "mueluguide"
%%% End:

    \end{abstract}


    % ------------------------------------------------------------------------ %
    % An Acknowledgement section is optional but important, if someone made
    % contributions or helped beyond the normal part of a work assignment.
    % Use \section* since we don't want it in the table of context
    %
    \clearpage
    \chapter*{Acknowledgment}
	\input{acknowledgment}


    % ------------------------------------------------------------------------ %
    % The table of contents and list of figures and tables
    % Comment out \listoffigures and \listoftables if there are no
    % figures or tables. Make sure this starts on an odd numbered page
    %
    \cleardoublepage		% TOC needs to start on an odd page
    \tableofcontents
    \listoffigures
    \listoftables


    % ---------------------------------------------------------------------- %
    % An optional preface or Foreword
    %\clearpage
    %\chapter*{Preface}
    %\addcontentsline{toc}{chapter}{Preface}
	%\input{CommonPreface}


    % ---------------------------------------------------------------------- %
    % An optional executive summary
    %\clearpage
    %\chapter*{Summary}
    %\addcontentsline{toc}{chapter}{Summary}
	%\input{CommonSummary}


    % ---------------------------------------------------------------------- %
    % An optional glossary. We don't want it to be numbered
    %\clearpage
    %\chapter*{Nomenclature}
    %\addcontentsline{toc}{chapter}{Nomenclature}
    %\begin{description}
	%\item[dry spell]
	%    using a dry erase marker to spell words
	%\item[dry wall]
	%    the writing on the wall
	%\item[dry humor]
	%    when people just do not understand
	%\item[DRY]
	%    Don't Repeat Yourself
    %\end{description}


    % ---------------------------------------------------------------------- %
    % This is where the body of the report begins; usually with an Introduction
    %
    \SANDmain		% Start the main part of the report

    %-----------------------------%
    \chapter{Introduction}\label{sec:introduction}
    %-----------------------------%
    This guide gives an overview of \muelu{}'s capabilities.  If you are looking for
a tutorial, please refer to the \muelu{} tutorial in \verb!muelu/doc/Tutorial/!
(see also~\cite{MueLuTutorial}). New users should start with~\S\ref{sec:getting
started}. It strives to give the new user all the information he/she might need
to begin using \muelu{} quickly. Users interested in performance, especially in
parallel context, should refer to~\S\ref{sec:performance}.  Users looking for a
particular option should consult~\S\ref{sec:options}, containing a complete set of
supported options in \muelu{}. \\

\noindent
If you find any errors or omissions in this guide, have comments or suggestions,
or would like to contribute to \muelu{} development, please contact the \muelu{}
\href{mailto:muelu-users@software.sandia.gov}{users list}, or
\href{mailto:muelu-developers@software.sandia.gov}{developers list} or open an
issue on the \href{https://github.com/trilinos/Trilinos}{Trilinos github repository}.


%%% Local Variables:
%%% mode: latex
%%% TeX-master: "mueluguide"
%%% End:


    %-----------------------------%
    \chapter{Multigrid background}\label{sec:multigrid}
    %-----------------------------%
    \input{multigrid}


    %-----------------------------%
    \chapter{Getting Started}\label{sec:getting started}
    %-----------------------------%
    This section is meant to get you using \ifpacktwo{} as quickly as possible.
\S\ref{sec:overview} gives a brief overview of \ifpacktwo{}.
\S\ref{sec:configuration_and_build} lists \ifpacktwo{}'s dependencies on other
\trilinos{} libraries and provides a sample cmake configuration line. Finally,
some examples of code are given in~\S\ref{sec:examples in code}.

\section{Overview of \ifpacktwo{}}
\label{sec:overview}
\ifpacktwo{} is a C++ linear solver library in the \trilinos{} project~\cite{Heroux2012}.
It originally began as a migration of \ifpack{} package capabilities to a new linear
algebra stack. While it retains some commonalities with the original package, it
has since diverged significantly from it and should be treated as completely
independent package.

\ifpacktwo{} only works with \tpetra{}~\cite{TpetraURL} matrix,
vector, and map types. Like \tpetra{}, it allows for different ordinal
(index) and scalar types. \ifpacktwo{} was designed to be efficient on a wide
range of computer architectures, from workstations to supercomputers~\cite{Lin2014}.
It relies on the ``MPI+X" principle, where ``X'' can be threading or
CUDA\@. The ``X'' portion, node-level parallelism, is controlled by a node
template type. Users should refer to \tpetra{}'s documentation for information
about node and device types.

\ifpacktwo provides a number of different solvers, including
\begin{itemize}
  \item Jacobi, Gauss-Seidel, polynomial, distributed relaxation;
  \item domain decomposition solvers;
  \item incomplete factorizations.
\end{itemize}
This list of solvers is not exhaustive. Instead, references for further
information are provided throughout the text. There are many excellent
references for iterative methods, including~\cite{Saad2003}.

Complete information on available capabilities and options can be found
in~\S\ref{sec:options}.

\section{Configuration and Build}\label{sec:configuration_and_build}

\ifpacktwo{} requires a C++11 compatible compiler for compilation. The
minimum required version of compilers are GCC (4.7.2 and later),
Intel (13 and later), and clang (3.5 and later).

\subsection{Dependencies}

Table~\ref{tab:dependencies} enumerates the dependencies of \ifpacktwo. Certain
dependencies are optional, whereas others are required.  Furthermore,
\ifpacktwo's tests depend on certain libraries that are not required if you only
want to link against the \ifpacktwo library and do not want to compile its
tests. Additionally, some functionality in \ifpacktwo{} may depend on other
Trilinos packages (for instance, \amesostwo{}) that may require additional
dependencies. We refer to the documentation of those packages for a full list of
dependencies.

\begin{table}[ht]
  \centering
  \begin{tabular}{p{3.5cm} c c c c}
    \toprule
    \multirow{2}{*}{Dependency} & \multicolumn{2}{c}{Library} & \multicolumn{2}{c}{Testing} \\
    \cmidrule(r){2-3} \cmidrule(l){4-5} & Required & Optional & Required & Optional  \\
    \midrule
    % \belos                       & $\times$ &          & $\times$ & \\
    \teuchos                     & $\times$ &          & $\times$ & \\
    \tpetra                      & $\times$ &          & $\times$ & \\
    \tpetrakernels               & $\times$ &          &          & \\
    \amesostwo                   &          & $\times$ &          & $\times$  \\
    \galeri                      &          &          &          & $\times$  \\
    \xpetra                      &          & $\times$ &          & $\times$  \\
    \zoltantwo                   &          & $\times$ &          & $\times$  \\
    \textsc{ThyraTpetraAdapters} &          & $\times$ &          & \\
    \textsc{ShyLUHTS}            &          & $\times$ &          & $\times$ \\
    \midrule
    % BLAS                         & $\times$ &          & $\times$ & \\
    % LAPACK                       & $\times$ &          & $\times$ & \\
    MPI                          &          & $\times$ &          & $\times$  \\
    % Cholmod                      &          & $\times$ &          & $\times$  \\
    % SuperLU 4.3                  &          & $\times$ &          & $\times$  \\
    % QD                           &          & $\times$ &          & $\times$  \\
    \bottomrule
  \end{tabular}
  \caption{\label{tab:dependencies}\ifpacktwo{}'s required and optional dependencies,
    subdivided by whether a dependency is that of the \ifpacktwo{}{} library itself
    (\textit{Library}), or of some \ifpacktwo{}{} test (\textit{Testing}). }
\end{table}

\amesostwo and \superlu are necessary if you want to use either a sparse direct
solve or ILUTP as a subdomain solve in processor-based domain decomposition.
\zoltantwo and \xpetra are necessary if you want to reorder a matrix (e.g.,
reverse Cuthill McKee).

\subsection{Configuration}
The preferred way to configure and build \ifpacktwo{} is to do that outside of the source directory.
Here we provide a sample configure script that will enable \ifpacktwo{} and all of its optional dependencies:
\begin{lstlisting}
  export TRILINOS_HOME=/path/to/your/Trilinos/source/directory
  cmake -D BUILD_SHARED_LIBS:BOOL=ON \
        -D CMAKE_BUILD_TYPE:STRING="RELEASE" \
        -D CMAKE_CXX_FLAGS:STRING="-g" \
        -D Trilinos_ENABLE_EXPLICIT_INSTANTIATION:BOOL=ON \
        -D Trilinos_ENABLE_TESTS:BOOL=OFF \
        -D Trilinos_ENABLE_EXAMPLES:BOOL=OFF \
        -D Trilinos_ENABLE_Ifpack2:BOOL=ON \
        -D Ifpack2_ENABLE_TESTS:STRING=ON \
        -D Ifpack2_ENABLE_EXAMPLES:STRING=ON \
        -D TPL_ENABLE_BLAS:BOOL=ON \
        -D TPL_ENABLE_MPI:BOOL=ON \
        ${TRILINOS_HOME}
\end{lstlisting}

\noindent
More configure examples can be found in \texttt{Trilinos/sampleScripts}.
For more information on configuring, see the \trilinos Cmake Quickstart guide \cite{TrilinosCmakeQuickStart}.

\section{Interface to \ifpacktwo{} methods}
All \ifpacktwo operators inherit from the base class
\texttt{Ifpack2::Preconditioner}. This in turn inherits from
\texttt{Tpetra::Operator}. Thus, you may use an \ifpacktwo operator anywhere
that a \texttt{Tpetra::Operator} works. For example, you may use \ifpacktwo operators
directly as preconditioners in \trilinos' \belos package of iterative solvers.

You may either create an \ifpacktwo operator directly, by using the class and
options that you want, or by using \texttt{Ifpack2::Factory}. Some of
\ifpacktwo preconditioners only accept a \texttt{Tpetra::\\CrsMatrix} instance as
input, while others also may accept a \texttt{Tpetra::RowMatrix} (the base class
of \texttt{Tpetra::CrsMatrix}). They will decide at run time whether the input
\texttt{Tpetra::RowMatrix} is an instance of the right subclass.

\texttt{Ifpack2::Preconditioner} includes the following methods:
\begin{itemize}
  \item \texttt{initialize()}

    Performs all operations based on the graph of the matrix (without
    considering the numerical values).

  \item \texttt{compute()}

    Computes everything required to apply the preconditioner, using the matrix's
    values.

  \item \texttt{apply()}

    Applies or ``solves with'' the preconditioner.
\end{itemize}
Every time that \texttt{initialize()} is called, the object destroys all the
previously allocated information, and reinitializes the preconditioner. Every
time \texttt{compute()} is called, the object recomputes the actual values of the
preconditioner.

An \ifpacktwo preconditioner may also inherit from
\texttt{Ifpack2::CanChangeMatrix} class in order to express that users can
change its matrix (the matrix that it preconditions) after construction using a
\texttt{setMatrix} method.  Changing the matrix puts the preconditioner back in
an ``pre-initialized'' state.  You must first call \texttt{initialize()}, then
\texttt{compute()}, before you may call \texttt{apply()} on this preconditioner.
Depending on the implementation, it may be legal to set the matrix to null. In
that case, you may not call \texttt{initialize()} or \texttt{compute()} until
you have subsequently set a nonnull matrix.

\textbf{Warning.} If you are familiar with the \ifpack package~\cite{ifpack}, please be aware
that the behaviour of the \ifpacktwo preconditioner is different from \ifpack.
In \ifpack, the \texttt{ApplyInverse()} method applies or ``solves with'' the
preconditioner $M^{-1}$, and the \texttt{Apply()} method ``applies'' the
preconditioner $M$. In \ifpacktwo, the \texttt{apply()} method applies or
``solves with'' the preconditioner $M^{-1}$. \ifpacktwo has no method comparable
to \ifpack's \texttt{Apply()}.

\section{Example: \ifpacktwo preconditioner within \belos}\label{sec:examples in code}

The most commonly used scenario involving \ifpacktwo{} is using one of its
preconditioners preconditioners inside an iterative linear solver. In
\trilinos{}, the \belos{} package provides important Krylov subspace methods (such
as preconditioned CG and GMRES).

At this point, we assume that the reader is comfortable with \teuchos{} referenced-counted
pointers (RCPs) for memory management (an introduction to RCPs can be found
in~\cite{RCP2010}) and the \parameterlist class~\cite{TeuchosURL}.

First, we create an \ifpacktwo{} preconditioner using a provided \parameterlist
\begin{lstlisting}[language=C++]
 typedef Tpetra::CrsMatrix<Scalar, LocalOrdinal, GlobalOrdinal, Node>
   Tpetra_Operator;

 Teuchos::RCP<Tpetra_Operator> A;
 // create A here ...
 Teuchos::ParameterList paramList;
 paramList.set( "chebyshev: degree", 1 );
 paramList.set( "chebyshev: min eigenvalue", 0.5 );
 paramList.set( "chebyshev: max eigenvalue", 2.0 );
 // ...
 Ifpack2::Factory factory;
 RCP<Ifpack2::Ifpack2Preconditioner<> > ifpack2Preconditioner;
 ifpack2Preconditioner = factory.create( "CHEBYSHEV", A )
 ifpack2Preconditioner->setParameters( paramList );
 ifpack2Preconditioner->initialize();
 ifpack2Preconditioner->compute();
\end{lstlisting}

Besides the linear operator $A$, we also need an initial guess vector for the
solution $X$ and a right hand side vector $B$ for solving a linear system.
\begin{lstlisting}[language=C++]
 typedef Tpetra::Map<LocalOrdinal, GlobalOrdinal, Node> Tpetra_Map;
 typedef Tpetra::MultiVector<Scalar, LocalOrdinal, GlobalOrdinal, Node>
   Tpetra_MultiVector;

 Teuchos::RCP<const Tpetra_Map> map = A->getDomainMap();

 // create initial vector
 Teuchos::RCP<Tpetra_MultiVector> X =
   Teuchos::rcp( new Tpetra_MultiVector(map, numrhs) );

 // create right-hand side
 X->randomize();
 Teuchos::RCP<Tpetra_MultiVector> B =
   Teuchos::rcp( new Tpetra_MultiVector(map, numrhs) );
 A->apply( *X, *B );
 X->putScalar( 0.0 );
\end{lstlisting}
To generate a dummy example, the above code first declares two vectors. Then, a
right hand side vector is calculated as the matrix-vector product of a random vector
with the operator $A$. Finally, an initial guess is initialized with zeros.

Then, one can define a \texttt{Belos::LinearProblem} object where the
\texttt{ifpack2Preconditioner} is used for left preconditioning.
\begin{lstlisting}[language=C++]
 typedef Belos::LinearProblem<Scalar, Tpetra_MultiVector, Tpetra_Operator>
   Belos_LinearProblem;

 Teuchos::RCP<Belos_LinearProblem> problem =
   Teuchos::rcp( new Belos_LinearProblem( A, X, B ) );
 problem->setLeftPrec( ifpack2Preconditioner );
 bool set = problem.setProblem();
\end{lstlisting}

Next, we set up a \belos{} solver using some basic parameters.
\begin{lstlisting}[language=C++]
 Teuchos::RCP<Teuchos::ParameterList> belosList =
   Teuchos::rcp(new Teuchos::ParameterList);
 belosList->set( "Block Size", 1 );
 belosList->set( "Maximum Iterations", 100 );
 belosList->set( "Convergence Tolerance", 1e-10 );
 belosList->set( "Output Frequency", 1 );
 belosList->set( "Verbosity", Belos::TimingDetails + Belos::FinalSummary );

 Belos::SolverFactory<Scalar, Tpetra_MultiVector, Tpetra_Operator> solverFactory;
 Teuchos::RCP<Belos::SolverManager<Scalar, Tpetra_MultiVector, Tpetra_Operator> >
   solver = solverFactory.create( "Block CG", belosList );
 solver->setProblem( problem );
\end{lstlisting}

Finally, we solve the system.
\begin{lstlisting}[language=C++]
 Belos::ReturnType ret = solver.solve();
\end{lstlisting}

It is often more convenient to specify the parameters as part of an XML-formatted options file.
Look in the subdirectory {\tt Trilinos/packages/ifpack2/test/belos} for examples of this.

This section is only meant to give a brief introduction on how to use
\ifpacktwo{} as a preconditioner within the \trilinos{} packages for iterative
solvers. There are other, more complicated, ways to use to work with
\ifpacktwo{}. For more information on these topics, the reader may refer to the
examples and tests in the \ifpacktwo{} source directory
(\texttt{Trilinos/packages/ifpack2}).


    %-----------------------------%
    \chapter{Performance tips}\label{sec:performance}
    %-----------------------------%
    In practice, it can be very challenging to find an appropriate set of multigrid
parameters for a specific problem, especially if few details are known about the
underlying linear system. In this Chapter, we provide some advice for improving
multigrid performance.

\begin{mycomment}
For optimizing multigrid parameters, it is highly recommended to set the
verbosity to \verb|high| or \verb|extreme| for \muelu{} to output more
information (e.g., for the effect of the chosen parameters to the aggregation
and coarsening process).
\end{mycomment}

Some general advice:
\begin{itemize}
  \item
    Choose appropriate iterative linear solver (e.g., GMRES for non-symmetric problems).
    If available, set options to perform as few all-reduces as possible.
    (E.g. \texttt{Use Single Reduction} in \belos.)

  \item
    Start with the recommended settings for particular problem types. See
    Table~\ref{t:problem_types}.

  \item
    Choose reasonable basic multigrid parameters
    (see~\S\ref{sec:options_general}), including maximum number of multigrid
    levels (\texttt{max levels}) and maximum allowed coarse size of the problem
    (\texttt{coarse: max size}). Take fine level problem size and sparsity
    pattern into account for a reasonable choice of these parameters.

  \item
    Select an appropriate transfer operator strategy
    (see~\S\ref{sec:options_mg}). For symmetric problems, you should start with smoothed
    aggregation multigrid. For non-symmetric problems, a Petrov-Galerkin smoothed
    aggregation method is a good starting point, though smoothed aggregation may
    also perform well.

  \item
    Try \texttt{unsmoothed} operators instead of smoothed aggregation (\texttt{sa}).
    Scalability in terms of iterations performed will suffer from this,
    but solution times might go down since the operators are less dense,
    and less communication is performed.

  \item
    Enable implicit restrictor construction (\texttt{transpose:} \texttt{use implicit}) for symmetric
    problems.

  \item
    Enable triple matrix products instead of two matrix-matrix products for
    the construction of coarse operators (\texttt{rap: triple product}).
    This is beneficial as long as the involved operators are not too dense.
    For \texttt{unsmoothed} hierarchies, it is always faster.

  \item
    Find good level smoothers (see~\S\ref{sec:options_smoothing}). If a problem
    is symmetric positive definite, choose a smoother with a matrix-vector
    computational kernel, such as the Chebyshev polynomial smoother. If you are
    using relaxation smoothers, we recommend starting with optimizing the
    damping parameter. Once you have found a good damping parameter for your
    problem, you can increase the number of smoothing iterations.

  \item
    Adjust aggregation parameters if you experience bad coarsening ratios
    (see~\S\ref{sec:options_aggregation}). Particularly, try adjusting the
    minimum (\texttt{aggregation:} \texttt{min agg size}) and maximum
    (\texttt{aggregation:} \texttt{max agg size}) aggregation parameters. For a
    2D (3D) isotropic problem on a regular mesh, the aggregate size should be
    about 9 (27) nodes per aggregate.

  \item
    Replace a direct solver with an iterative method (\texttt{coarse: type}) if
    your coarse level solution becomes too expensive (see~\S\ref{sec:options_smoothing}).

  \item
    If on-node parallelism is required, make sure to enable the \kokkos code path (\texttt{use kokkos refactor}).
    If Gauss-Seidel smoothing is used, switch to multi-threaded Gauss-Seidel (see~\S\ref{sec:options_smoothing}).

\end{itemize}

Some advice for parallel runs include:
\begin{enumerate}
  \item
    Enable matrix rebalancing when running in parallel (\texttt{repartition:}
    \texttt{enable}).

  \item
    Use smoothers invariant to the number of processors, such as
    polynomial smoothing, if possible.

  \item
    Adjust rebalancing parameters (see~\S\ref{sec:options_rebalancing}). Try
    choosing rebalancing parameters so that you end up with one processor on the
    coarsest level for the direct solver (this avoids additional communication).

  \item
    If the \texttt{multijagged} algorithm from \zoltan2 is used, try setting the premigration option.

  \item
    Enable implicit rebalancing of prolongators and restrictors
    (\texttt{repartition: rebalance P and R}).
\end{enumerate}

%%% Local Variables:
%%% mode: latex
%%% TeX-master: "mueluguide"
%%% End:


    %-----------------------------%
    \chapter{\muelu{} options} \label{sec:options}
    %-----------------------------%
    \label{sec:muelu_options}

In this section, we report the complete list of \muelu{} input parameters.  It
is important to notice, however, that \muelu{} relies on other \trilinos{}
packages to provide support for some of its algorithms. For instance,
\ifpack{}/\ifpacktwo{} provide standard smoothers like Jacobi, Gauss-Seidel or
Chebyshev, while \amesos{}/\amesostwo{} provide access to direct solvers. The
parameters affecting the behavior of the algorithms in those packages are
simply passed by \muelu{} to a routine from the corresponding library. Please
consult corresponding packages' documentation for a full list of supported
algorithms and corresponding parameters.

\section{Using parameters on individual levels}
Some of the parameters that affect the preconditioner can in principle be
different from level to level. By default, parameters affect all levels in
a multigrid hierarchy.

The settings on a particular level can be changed by using level sublists.
A level sublist is a \parameterlist{} sublist with the name ``level XX'', where XX is the level number. The
parameter names in the sublist do not require any modifications. For example,
the following fragment of code
\begin{lstlisting}[language=XML]
  <ParameterList name="level 2">
    <Parameter name="smoother: type" type="string" value="CHEBYSHEV"/>
  </ParameterList>
\end{lstlisting}
changes the smoother for level 2 to be a Chebyshev-type polynomial smoother.

\section{Parameter validation}
By default, \muelu{} validates the input parameter list. A parameter that is
misspelled is unknown. A parameter with an incorrect value type is also treated as invalid.
Both cases will cause an exception to be
thrown and execution to halt.

\begin{mycomment}
Spaces are important within a parameter's name. Please separate words
by just one space, and make sure there are no leading or trailing spaces.
\end{mycomment}

The option \verb|print initial parameters| prints the initial list given to the
interpreter. The option \verb|print unused parameters| prints the list of unused
parameters.

% ==================== GENERAL ====================
\section{General options}
\label{sec:options_general}

\begin{table}[h!]
  \begin{center}
    \begin{tabular}{p{3cm} p{12cm}}
      \toprule
      Verbosity level           & Description \\
      \midrule
      \verb!none!               & No output \\
      \verb!low!                & Errors, important warnings, and some statistics \\
      \verb!medium!             & Same as \verb!low!, but with more statistics \\
      \verb!high!               & Errors, all warnings, and all statistics \\
      \verb!extreme!            & Same as \verb!high!, but also includes output from other packages (\textit{i.e.}, \zoltan{}) \\
      \bottomrule
    \end{tabular}
    \caption{Verbosity levels.}
\label{t:verbosity_types}
  \end{center}
\end{table}

\begin{table}[h!]
  \begin{center}
    \begin{tabular}{p{4.3cm} p{4.3cm} c p{4.5cm}}
      \toprule
      Problem type                 & Multigrid algorithm    & Block size  & Smoother \\
      \midrule
      \verb!unknown!               & --                     & --          & -- \\
      \verb!Poisson-2D!            & Smoothed aggregation   & 1           & Chebyshev \\
      \verb!Poisson-3D!            & Smoothed aggregation   & 1           & Chebyshev \\
      \verb!Elasticity-2D!         & Smoothed aggregation   & 2           & Chebyshev \\
      \verb!Elasticity-3D!         & Smoothed aggregation   & 3           & Chebyshev \\
      \verb!Poisson-2D-complex!    & Smoothed aggregation   & 1           & Symmetric Gauss-Seidel \\
      \verb!Poisson-3D-complex!    & Smoothed aggregation   & 1           & Symmetric Gauss-Seidel \\
      \verb!Elasticity-2D-complex! & Smoothed aggregation   & 2           & Symmetric Gauss-Seidel \\
      \verb!Elasticity-3D-complex! & Smoothed aggregation   & 3           & Symmetric Gauss-Seidel \\
      \verb!ConvectionDiffusion!   & Petrov-Galerkin  AMG   & 1           & Gauss-Seidel \\
      \verb!MHD!                   & Unsmoothed aggregation & --          & Additive Schwarz method with one level of overlap and ILU(0) as a subdomain solver \\
      \bottomrule
    \end{tabular}
    \caption{Supported problem types (``--'' means not set).}
\label{t:problem_types}
  \end{center}
\end{table}


\cbb{problem: type}{string}{"unknown"}{Type of problem to be solved. Possible values: see Table~\ref{t:problem_types}.}
          
\cbb{verbosity}{string}{"high"}{Control of the amount of printed information. Possible values: see Table~\ref{t:verbosity_types}.}
          
\cbb{number of equations}{int}{1}{Number of PDE equations at each grid node. Only constant block size is considered.}
          
\cbb{max levels}{int}{10}{Maximum number of levels in a hierarchy.}
          
\cbb{cycle type}{string}{"V"}{Multigrid cycle type. Possible values: "V", "W".}
          
\cbb{problem: symmetric}{bool}{true}{Symmetry of a problem. This setting affects the construction of a restrictor. If set to true, the restrictor is set to be the transpose of a prolongator. If set to false, underlying multigrid algorithm makes the decision.}
          
\cbb{xml parameter file}{string}{""}{An XML file from which to read additional
      parameters.  In case of a conflict, parameters manually set on
      the list will override parameters in the file. If the string is
      empty a file will not be read.}
          
\cbb{hierarchy label}{string}{""}{Label for the hierarchy. Is applied to timer labels.}
          

% ==================== SMOOTHERS ====================
\section{Smoothing and coarse solver options}
\label{sec:options_smoothing}

\muelu{} relies on other \trilinos{} packages to provide level smoothers and
coarse solvers. \ifpack{} and \ifpacktwo{} provide standard smoothers (see
Table~\ref{tab:smoothers}), and \amesos{} and \amesostwo{} provide direct
solvers (see Table~\ref{tab:coarse_solvers}). Note that it is completely possible to use
any level smoother as a direct solver.

\muelu{} checks parameters \verb|smoother: * type| and \verb|coarse: type| to
determine:
\begin{itemize}
  \item what package to use (i.e., is it a smoother or a direct solver);
  \item (possibly) transform a smoother type

    \ding{42} \ifpack{} and \ifpacktwo{} use different smoother type names,
    e.g., ``point relaxation stand-alone'' vs ``RELAXATION''.  \muelu{} tries to follow
    \ifpacktwo{} notation for smoother types. Please consult \ifpacktwo{}
    documentation~\cite{Ifpack2} for more information.
\end{itemize}
The parameter lists \verb|smoother: * params| and \verb|coarse: params| are
passed directly to the corresponding package without any examination of their
content. Please consult the documentation of the corresponding packages for a list of
possible values.

By default, \muelu{} uses one sweep of symmetric Gauss-Seidel for both pre- and
post-smoothing, and SuperLU for coarse system solver.

\begin{table}[tbh]
  \begin{center}
    \begin{tabular}{p{4.0cm} p{10cm}}
      \toprule
      \texttt{smoother: type}           & \\
      \midrule
      \verb|RELAXATION|                 & Point relaxation smoothers, including
                                          Jacobi, Gauss-Seidel, symmetric Gauss-Seidel,
                                          multithreaded (coloring-based) Gauss-Seidel, etc. The exact
                                          smoother is chosen by specifying \texttt{relaxation: type} parameter in
                                          the \texttt{smoother: params} sublist. \\
      \verb|CHEBYSHEV|                  & Chebyshev polynomial smoother. \\
      \verb|ILUT|, \verb|RILUK|         & Local (processor-based) incomplete factorization methods. \\
      \bottomrule
    \end{tabular}
    \caption{Commonly used smoothers provided by \ifpack{}/\ifpacktwo{}. Note
    that these smoothers can also be used as coarse grid solvers.}
\label{tab:smoothers}
  \end{center}
\end{table}

\begin{table}[tbh]
  \begin{center}
    \begin{tabular}{p{4.0cm} c c p{7cm}}
      \toprule
      \texttt{coarse: type}             & \amesos{} & \amesostwo{} &  \\
      \midrule
      \verb|KLU|                        & x & & Default \amesos{} solver~\cite{klu}. \\
      \verb|KLU2|                       & & x & Default \amesostwo{} solver~\cite{amesos2_belos}. \\
      \verb|SuperLU|                    & x & x & Third-party serial sparse direct solver~\cite{Li2011}. \\
      \verb|SuperLU_dist|               & x & x & Third-party parallel sparse direct solver~\cite{Li2011}. \\
      \verb|Umfpack|                    & x & & Third-party solver~\cite{umfpack}. \\
      \verb|Mumps|                      & x & & Third-party solver~\cite{mumps}. \\
      \bottomrule
    \end{tabular}
    \caption{Commonly used direct solvers provided by \amesos{}/\amesostwo{}}
\label{tab:coarse_solvers}
  \end{center}
\end{table}

In certain cases, the user may want to do no smoothing on a particular level, or do no solve on the coarsest level.
\begin{itemize}
  \item To skip smoothing, use the option \verb!smoother: pre or post! with value \verb!none!.
  \item To skip the coarse grid solve, use the option \verb!coarse: type! with value \verb!none!.
\end{itemize}

When a problem can be solved using structured aggregation algorithms it is also possible to use the structured line detection factory,
 this will allow \muelu{} to pass additional information to \ifpack2{} enabling it to perform line smoothing.
An example of line smoothing is provided in \texttt{packages/trilinoscouplings/examples/scaling/muelu\_Ifpack2\_line\_detection.xml}.


\cbb{smoother: pre or post}{string}{"both"}{Pre- and post-smoother combination. Possible values: "pre" (only pre-smoother), "post" (only post-smoother), "both" (both pre-and post-smoothers), "none" (no smoothing).}
          
\cbb{smoother: type}{string}{"RELAXATION"}{Smoother type. Possible values: see Table~\ref{tab:smoothers}.}
          
\cbb{smoother: pre type}{string}{"RELAXATION"}{Pre-smoother type. Possible values: see Table~\ref{tab:smoothers}.}
          
\cbb{smoother: post type}{string}{"RELAXATION"}{Post-smoother type. Possible values: see Table~\ref{tab:smoothers}.}
          
\cba{smoother: params}{\parameterlist}{Smoother parameters. For standard smoothers, \muelu passes them directly to the appropriate package library.}
          
\cba{smoother: pre params}{\parameterlist}{Pre-smoother parameters. For standard smoothers, \muelu passes them directly to the appropriate package library.}
          
\cba{smoother: post params}{\parameterlist}{Post-smoother parameters. For standard smoothers, \muelu passes them directly to the appropriate package library.}
          
\cbb{smoother: overlap}{int}{0}{Smoother subdomain overlap.}
          
\cbb{smoother: pre overlap}{int}{0}{Pre-smoother subdomain overlap.}
          
\cbb{smoother: post overlap}{int}{0}{Post-smoother subdomain overlap.}
          
\cbb{coarse: max size}{int}{2000}{Maximum dimension of a coarse grid. \muelu will stop coarsening once it is achieved.}
          
\cbb{coarse: type}{string}{"KLU"}{Coarse solver. Possible values: see Table~\ref{tab:coarse_solvers}.}
          
\cba{coarse: params}{\parameterlist}{Coarse solver parameters. \muelu passes them directly to the appropriate package library.}
          
\cbb{coarse: overlap}{int}{0}{Coarse solver subdomain overlap.}
          

% ==================== AGGREGATION ====================
\section{Aggregation options}
\label{sec:options_aggregation}

\begin{table}[H]
  \begin{center}
    \begin{tabular}{p{5.0cm} p{10cm}}
      \toprule
      \verb!structured!   & Attempts to construct hexahedral aggregates on a structured
                            mesh using a default coarsening rate of $3$ in each spatial
                            dimension.\\
      \verb!hybrid!       & This option takes in a user parameter that varies on each
                            rank and that specifies whether the local aggregation
                            scheme should be \verb!structured! or \verb!unstructured!.\\
      \verb!uncoupled!    & Attempts to construct aggregates of optimal size ($3^d$
                            nodes in $d$ dimensions). Each process works independently, and
                            aggregates cannot span multiple processes.\\
      \verb!brick!        & Attempts to construct rectangular aggregates \\
      %\verb!METIS!     & Use graph partitioning algorithm to create aggregates,
      %                   working process-wise. Number of nodes in each aggregate
      %                   is specified with option \texttt{aggregation: max agg
      %                   size}. \\
      % \verb!ParMETIS!  & As \verb!METIS!, but partition global graph. Aggregates
                         % can span arbitrary number of processes. Specify global
                         % number of aggregates with {\tt aggregation: global
                         % number}. \\
      \bottomrule
    \end{tabular}
    \caption{Available coarsening schemes. }
\label{t:aggregation}
  \end{center}
\end{table}

\begin{table}[H]
  \begin{center}
    \begin{tabular}{p{4.0cm} p{8.5cm} p{1.5cm} p{2.0cm}}
      Algorithm & Description & Parallel? & Deterministic? \\
      \toprule
      \verb!mis2 aggregation! &     Uses distance-2 MIS to perform aggregation similar to non-Kokkos refactor.
                                    Faster than coloring-based aggregation.
                                    Cannot control min/max aggregate size. & Yes & Yes \\
      \verb!mis2 coarsening! &      Uses distance-2 MIS to perform simple coarsening where only independent vertices become aggregate roots.
                                    This means the coarse ratio is higher than in \verb!mis2 aggregation!.
                                    Faster than coloring-based aggregation.  Cannot control min/max aggregate size. & Yes & Yes \\
      \verb!serial! &               Computes a D2 coloring on host, and then does 4-phase aggregation on device. & No & Optional \\
      \verb!default! &              Computes \verb!serial! on a serial device and \verb!net based bit set! on a parallel device. & Yes & Optional \\
      \verb!vertex based! &         Computes a D2 coloring in parallel with conflicts resolved by neighbors-of-neighbors loop. Then does 4-phase aggregation on device. & Yes & No \\
      \verb!vertex based bit set! & Computes the same as \verb!vertex based!, but with an optimization using a 32-bit integer instead of 32 bools to track forbidden colors. & Yes & No \\
      \verb!edge filtering! &       Computes the same as \verb!vertex based bit set!, but with an optimization that can filter and skip some edges. & Yes & No \\
      \verb!net based bit set! &    Computes a D2 coloring in parallel with net-based algorithm, which is asymptotically faster than vertex-based (though not always faster in practice).
                                    Then does 4-phase aggregation on device. & Yes & No \\
      \bottomrule
    \end{tabular}
    \caption{Available choices for \texttt{aggregation: coloring algorithm}. This controls Kokkos-refactored uncoupled aggregation. }
\label{t:coloring_algs}
  \end{center}
\end{table}

Table \ref{t:coloring_algs} lists the available values for \verb!aggregation: coloring algorithm!, which (in Kokkos-refactored uncoupled aggregation) controls how aggregate roots are selected and aggregates are constructed. Under ``Deterministic?'', Yes means the same aggregates are produced for a given graph every run, on any machine. Optional means deterministic only if \verb!aggregation: deterministic! is true, and in this case determinism comes with a speed penalty.


\cbb{aggregation: type}{string}{"uncoupled"}{Aggregation scheme. Possible values: see Table~\ref{t:aggregation}.}
          
\cbb{aggregation: mode}{string}{"uncoupled"}{Controls whether aggregates are allowed to cross processor boundaries. Possible values: "uncoupled" aggregates cannot cross processor boundaries.}
          
\cbb{aggregation: ordering}{string}{"natural"}{Node ordering strategy. Possible values: "natural" (local index order), "graph" (filtered graph breadth-first order), "random" (random local index order).}
          
\cbb{aggregation: phase 1 algorithm}{string}{"Distance2"}{Look at distance2 hops when aggregating.}
          
\cbb{aggregation: drop scheme}{string}{"classical"}{Connectivity dropping scheme for a graph used in
      aggregation. Possible values: "classical", "distance laplacian",
      "unsupported vector smoothing"}
          
\cbb{aggregation: distance laplacian metric}{string}{unweighted}{Metric used to compute the distance Laplacian. Possible values: "unweighted", "material"}
          
\cbb{aggregation: drop tol}{double}{0.0}{Connectivity dropping threshold for a graph used in aggregation.}
          
\cbb{aggregation: use ml scaling of drop tol}{bool}{false}{Enables ML-style scaling of drop tol, where the drop tol halves with each successive level.}
          
\cbb{aggregation: min agg size}{int}{2}{Minimum size of an aggregate.}
          
\cbb{aggregation: max agg size}{int}{-1}{Maximum size of an aggregate (-1 means unlimited).}
          
\cbb{aggregation: compute aggregate qualities}{bool}{false}{Whether to compute aggregate quality estimates.}
          
\cbb{aggregation: brick x size}{int}{2}{Number of points for x axis in "brick" aggregation (limited to 3).}
          
\cbb{aggregation: brick y size}{int}{2}{Number of points for y axis in "brick" aggregation (limited to 3).}
          
\cbb{aggregation: brick z size}{int}{2}{Number of points for z axis in "brick" aggregation (limited to 3).}
          
\cbb{aggregation: brick x Dirichlet}{bool}{false}{Asserts that Dirichlet conditions are applied in
        the x-direction and the Dirichlet DOFs are not aggregated.}
          
\cbb{aggregation: brick y Dirichlet}{bool}{false}{Asserts that Dirichlet conditions are applied in
        the y-direction and the Dirichlet DOFs are not aggregated.}
          
\cbb{aggregation: brick z Dirichlet}{bool}{false}{Asserts that Dirichlet conditions are applied in
        the z-direction and the Dirichlet DOFs are not aggregated.}
          
\cbb{aggregation: Dirichlet threshold}{double}{0.0}{Threshold for determining whether entries are zero during Dirichlet row detection.}
          
\cbb{aggregation: greedy Dirichlet}{bool}{false}{Force the aggregate to be Dirichlet if any DOFs in the aggregate are Dirichlet (default is aggregates are Dirichlet only if all DOFs in the aggregate are Dirichlet).}
          
\cbb{aggregation: deterministic}{bool}{false}{Boolean indicating whether or not aggregation will be run deterministically in the kokkos refactored path (only used in uncoupled aggregation).}
          
\cbb{aggregation: coloring algorithm}{string}{serial}{Choice of distance 2 independent set or coloring algorithm used by Uncoupled Aggregation, when using kokkos refactored aggregation. See Table \ref{t:coloring_algs} for more information.}
          
\cbb{aggregation: dropping may create Dirichlet}{bool}{true}{If true, any matrix row has nonzero off-diagonal entries will be treated as Dirichlet if aggregation dropping leaves only the diagonal entry.}
          
\cbb{aggregation: export visualization data}{bool}{false}{Export data for visualization post-processing.}
          
\cbb{aggregation: output filename}{string}{""}{Filename to write VTK visualization data to.}
          
\cbb{aggregation: output file: time step}{int}{0}{Time step ID for non-linear problems.}
          
\cbb{aggregation: output file: iter}{int}{0}{Iteration for non-linear problems.}
          
\cbb{aggregation: output file: agg style}{string}{Point Cloud}{Style of aggregate visualization.}
          
\cbb{aggregation: output file: fine graph edges}{bool}{false}{Whether to draw all fine node connections along with the aggregates.}
          
\cbb{aggregation: output file: coarse graph edges}{bool}{false}{Whether to draw all coarse node connections along with the aggregates.}
          
\cbb{aggregation: output file: build colormap}{bool}{false}{Whether to output a random colormap in a separate XML file.}
          
\cbb{aggregation: output file: aggregate qualities}{bool}{false}{Whether to plot the aggregate quality.}
          
\cbb{aggregation: output file: material}{bool}{false}{Whether to plot the material.}
          
\cbb{aggregation: mesh layout}{string}{Global Lexicographic}{Type of ordering for structured mesh aggregation. Possible values: "Global Lexicographic" and "Local Lexicographic".}
          
\cbb{aggregation: output type}{string}{Aggregates}{Type of object holding the aggregation data. Possible values: "Aggregates" or "CrsGraph".}
          
\cbb{aggregation: coarsening rate}{string}{{3}}{Coarsening rate per spatial dimensions, the string must be interpretable as an array by Teuchos.}
          
\cbb{aggregation: number of spatial dimensions}{int}{3}{The number of spatial dimensions in the problem.}
          
\cbb{aggregation: coarsening order}{int}{0}{The interpolation order used while constructing these aggregates, this value will be passed to the prolongator factory. There, possible values are 0 for piece-wise constant and 1 for piece-wise linear interpolation to transfer values from coarse points to fine points. }
          

The option \texttt{aggregation: strength-of-connection matrix} allows to select the matrix which is used in the construction of the strength-of-connection.
Two options are available:
\begin{itemize}
\item \texttt{A}: use the system matrix,
\item \texttt{distance laplacian}: use a distance Laplacian matrix that has been constructed using the graph of the system matrix \(A\), nodal coordinates \(X\) and a metric \(d\) determined by the option \texttt{aggregation: distance laplacian metric}, i.e.
  \begin{equation}
    L_{d}(A, X)_{ij} :=
    \begin{cases}
      -\frac{1}{d(X_{i,:},X_{j,:})^{2}} & \text{if } i\neq j \text{ and } A_{ij}\neq 0,\\
      0 & \text{if } i\neq j \text{ and } A_{ij}= 0,\\
      -\sum_{j\neq i}L_{d}(A, X)_{ij} & \text{if } i=j.
    \end{cases}
  \end{equation}
\end{itemize}

The available options for \texttt{aggregation: strength-of-connection measure} are listed in Table \ref{t:aggregation_soc_measures}.

\begin{table}[H]
  \begin{center}
    \begin{tabular}{ll}
      \verb!aggregation: strength-of-connection measure! & \\
      \toprule
      \verb!smoothed aggregation! & \(\frac{\left|M_{ij}\right|}{\sqrt{\left|M_{ii}\right|\left|M_{jj}\right|}} \) \\
      \verb!signed smoothed aggregation! & \(\frac{-\operatorname{sign} \left(M_{ij}\right) \left|M_{ij}\right|}{\sqrt{\left|M_{ii}\right|\left|M_{jj}\right|}} \) \\
      \verb!signed ruge-stueben! & \(\frac{-\operatorname{Re} \left(M_{ij}\right)}{\sqrt{\left|M_{ii}\right|\left|M_{jj}\right|}} \) \\
      \verb!unscaled! & \(\left|M_{ij}\right|\) \\
      \bottomrule
    \end{tabular}
    \caption{Available choices for \texttt{aggregation: strength-of-connection measure}. Depending on the choice of \texttt{aggregation: strength-of-connection matrix} the matrix \(M\) is either the system matrix (\texttt{A}) or the distance Laplacian matrix (\texttt{distance laplacian}).}
\label{t:aggregation_soc_measures}
  \end{center}
\end{table}


Once a strength-of-connection matrix \(S\) has been computed a dropping scheme is applied to determine which entries of the graph should be kept or dropped.
All schemes use the dropping tolerance \(\theta\) set by \texttt{aggregation: drop tol}.
Two options are available for \texttt{aggregation: drop scheme}:
\begin{itemize}
\item \texttt{point-wise}: Drop entries in point-wise fashion, i.e. drop \((i,j)\) if \(\left|S_{ij}\right|\leq\theta\).
\item \texttt{cut-drop}: Drop entries in row-wise fashion. In the \(i\)-th row \(S_{i,:}\), sorts entries by absolute value and finds the first relative gap between entries of size bigger than \(\theta\). Drop all smaller entries.
\end{itemize}

A classical choice is to use \texttt{aggregation: strength-of-connection matrix} = \texttt{A}, \texttt{aggregation: strength-of-connection measure} = \texttt{smoothed aggregation} and \texttt{aggregation: drop scheme} = \texttt{point-wise}, resulting in the dropping criterion
\begin{equation}
  \frac{\left|A_{ij}\right|}{\sqrt{\left|A_{ii}\right|\left|A_{jj}\right|}} \leq \theta.
\end{equation}

To guarantee backward compatibility, \texttt{aggregation: drop scheme} also allows several legacy values.
These values trigger particular combinations of other parameter settings as shown in Table~\ref{t:old_aggregation_schemes}.

\begin{table}[H]
  \begin{center}
    \begin{tabular}{p{7.0cm} | p{10cm}}
      \texttt{aggregation: drop scheme} & \texttt{aggregation: strength-of-connection matrix}, \texttt{aggregation: strength-of-connection measure}\\
      \toprule
      \texttt{classical} & \texttt{A}, \texttt{smoothed aggregation} \\
      \texttt{signed classical sa} & \texttt{A}, \texttt{signed smoothed aggregation} \\
      \texttt{signed classical} & \texttt{A}, \texttt{signed ruge-stueben} \\
      \texttt{distance laplacian} &  \texttt{distance laplacian}, \texttt{smoothed aggregation} \\
      \texttt{signed classical sa distance laplacian} &  \texttt{distance laplacian}, \texttt{signed smoothed aggregation} \\
      \texttt{signed classical distance laplacian} &  \texttt{distance laplacian}, \texttt{signed ruge-stueben} \\
      \bottomrule
    \end{tabular}
    \caption{Legacy choices for \texttt{aggregation: drop scheme}.}
\label{t:old_aggregation_schemes}
  \end{center}
\end{table}

% ==================== REBALANCING ====================
\section{Rebalancing options}
\label{sec:options_rebalancing}


\cbb{repartition: enable}{bool}{false}{Rebalancing on/off switch.}
          
\cbb{repartition: partitioner}{string}{"zoltan2"}{Partitioning package to use. Possible values: "zoltan" (\zoltan{} library), "zoltan2" (\zoltantwo{} library).}
          
\cba{repartition: params}{\parameterlist}{Partitioner parameters. \muelu passes them directly to the appropriate package library.}
          
\cbb{repartition: start level}{int}{2}{Minimum level to run partitioner. \muelu does not rebalance levels finer than this one.}
          
\cbb{repartition: min rows per proc}{int}{800}{Minimum number of rows per MPI process. If the actual number if smaller, then rebalancing occurs.}

\cbb{repartition: target rows per proc}{int}{0}{Target number of rows per MPI process after rebalancing. If the value is set to 0, it will use the value of "repartition: min rows per proc"}
          
\cbb{repartition: max imbalance}{double}{1.2}{Maximum nonzero imbalance ratio. If the actual number is larger, the rebalancing occurs.}
          
\cbb{repartition: remap parts}{bool}{true}{Postprocessing for partitioning to reduce data migration.}
          
\cbb{repartition: rebalance P and R}{bool}{false}{Explicit rebalancing of R and P during the setup. This speeds up the solve, but slows down the setup phases.}
          

% ==================== MULTIGRID ====================
\section{Multigrid algorithm options}
\label{sec:options_mg}

\begin{table}[H]
  \begin{center}
    \begin{tabular}{p{3.5cm} p{11cm}}
      \toprule
      \verb!sa!          & Classic smoothed aggregation~\cite{VMB1996} \\
      \verb!unsmoothed!  & Aggregation-based, same as \verb!sa! but without damped Jacobi prolongator improvement step \\
      \verb!pg!          & Prolongator smoothing using $A$, restriction smoothing using $A^T$, local damping factors~\cite{ST2008} \\
      \verb!emin!        & Constrained minimization of energy in basis functions of grid transfer operator~\cite{WTWG2014,OST2011} \\
      \verb!interp!      & Interpolation based grid transfer operator, using piece-wise constant or linear interpolation from coarse nodes to fine nodes. This requires the use of structured aggregation.\\
      \verb!semicoarsen! & Semicoarsening grid transfer operator used to reduce a n-dimensional problem into a (n-1)-dimensional problem by coarsening fully in one of the spacial dimensions~\cite{TPSTP2015}.\\
      \verb!replicate!      & Transforms existing scalar prolongator to one  for PDE systems with multiple dofs per node.\\
      \verb!pcoarsen!    & \\
      \bottomrule
    \end{tabular}
    \caption{Available multigrid algorithms for generating grid transfer matrices. }
\label{t:mgs}
  \end{center}
\end{table}


\cbb{multigrid algorithm}{string}{"sa"}{Multigrid method. Possible values: see Table~\ref{t:mgs}.}
          
\cbb{semicoarsen: coarsen rate}{int}{3}{Rate at which to coarsen unknowns in the z direction.}
          
\cbb{sa: damping factor}{double}{1.33}{Damping factor for smoothed aggregation.}
          
\cbb{sa: use filtered matrix}{bool}{true}{Matrix to use for smoothing the tentative prolongator. The two options are: to use the original matrix, and to use the filtered matrix with filtering based on filtered graph used for aggregation.}
          
\cbb{interp: interpolation order}{int}{1}{Interpolation order used to interpolate values from coarse points to fine points. Possible values are 0 for piece-wise constant interpolation and 1 for piece-wise linear interpolation. This parameter is set to 1 by default.}
          
\cbb{interp: build coarse coordinates}{bool}{true}{If false, skip the calculation of coarse coordinates.}
          
\cbb{filtered matrix: use lumping}{bool}{true}{Lump (add to diagonal) dropped entries during the construction of a filtered matrix. This allows user to preserve constant nullspace.}
          
\cbb{filtered matrix: use root stencil}{bool}{false}{Use root-node based sparsification of the filtered
      matrix. This usually reduces operator complexity in the case of small aggregates.}
          
\cbb{filtered matrix: reuse eigenvalue}{bool}{true}{Skip eigenvalue calculation during the construction of a filtered matrix by reusing eigenvalue estimate from the original matrix. This allows us to skip heavy computation, but may lead to poorer convergence.}
          
\cbb{emin: iterative method}{string}{"cg"}{Iterative method to use for energy minimization of initial prolongator in energy-minimization. Possible values: "cg" (conjugate gradient), "gmres" (generalized minimum residual), "sd" (steepest descent).}
          
\cbb{emin: num iterations}{int}{2}{Number of iterations to minimize initial prolongator energy in energy-minimization.}
          
\cbb{emin: num reuse iterations}{int}{1}{Number of iterations to minimize the reused prolongator energy in energy-minimization.}
          
\cbb{emin: pattern}{string}{"AkPtent"}{Sparsity pattern to use for energy minimization. Possible values: "AkPtent".}
          
\cbb{emin: pattern order}{int}{1}{Matrix order for the "AkPtent" pattern.}
          
\cbb{emin: use filtered matrix}{bool}{true}{Matrix to use for smoothing for energy minimization. The two options are: to use the original matrix, and to use the filtered matrix with filtering based on filtered graph used for aggregation.}
          

% ==================== REUSE ====================
\section{Reuse options}
\label{sec:options_reuse}

Reuse options are a way for a user to speed up the setup stage of multigrid.
The main requirement to use reuse is that the matrix' graph structure does not
change. Only matrix values are allowed to change.

The reuse options control the degree to which the multigrid hierarchy is preserved
for a subsequent setup call.

In addition, please note that not all combinations of multigrid algorithms and
reuse options are valid, or even make sense. For instance, the "emin" reuse
option should only be used with the "emin" multigrid algorithm.

Table~\ref{t:reuse_types} contains the information about different reuse
options. The options are ordered in increasing number of reuse components, from
the no reuse to the full reuse ("full").

\begin{table}[H]
  \begin{center}
    \begin{tabular}{p{3.0cm} p{12cm}}
      \toprule
      \verb!none!   & No reuse \\
      \verb!S!      & Reuse only the symbolic information of the level smoothers. \\
      \verb!tP!     & Reuse tentative prolongator. The graphs of smoothed
                      prolongator and matrices in Galerkin product are reused
                      only if filtering is not being used ({\it i.e.}, either
                      \verb!sa: use filtered matrix! or \verb!aggregation: drop tol! is false) \\
      \verb!emin!   & Reuse old prolongator as an initial guess to energy
                      minimization, and reuse the prolongator pattern \\
      \verb!RP!     & Reuse smoothed prolongator and restrictor. Smoothers are
                      recomputed.  \ding{42} \verb!RP! should reuse matrix graphs for
                      matrix-matrix product, but currently that is disabled as only \epetra{}
                      supports it. \\
      \verb!RAP!    & Recompute only the finest level smoothers, reuse all other operators \\
      \verb!full!   & Reuse everything \\
      \bottomrule
    \end{tabular}
    \caption{Available reuse options.}
\label{t:reuse_types}
  \end{center}
\end{table}

\input{options_reuse}

% ==================== MISCELLANEOUS ====================
\section{Miscellaneous options}


\cba{export data}{\parameterlist}{Exporting a subset of the hierarchy data in a
      file. Currently, the list can contain any of the following parameter
      names (``A'', ``P'', ``R'', ``Nullspace'', ``Coordinates'', ``Aggregates'') of type \texttt{string}
      and value ``\{levels separated by commas\}''. A
      matrix/multivector with a name ``X'' is saved in two or three
      three MatrixMarket files: a) data is saved in
      \textit{X\_level.mm}; b) its row map is saved in
      \textit{rowmap\_X\_level.mm}; c) its column map (for matrices) is saved in
      \textit{colmap\_X\_level.mm}.}
          
\cbb{print initial parameters}{bool}{true}{Print parameters provided for a hierarchy construction.}
          
\cbb{print unused parameters}{bool}{true}{Print parameters unused during a hierarchy construction.}
          
\cbb{transpose: use implicit}{bool}{false}{Use implicit transpose for the restriction operator.}
          
\cbb{transfers: half precision}{bool}{false}{Replace transfer operators P and R (if explicitely constructed) with half precision versions for the solve phase..}
          
\cbb{nullspace: calculate rotations}{bool}{false}{When nullspace internally generated by muelu, calculate null space rotations in addition to translations.}
          
\cbb{nullspace: suppress dimension check}{bool}{false}{Suppress safety check to ensure that nullspace dimension is at least equal or greater than the number of PDEs per mesh node.}
          
\cbb{use kokkos refactor}{bool}{false}{Switch on the new \kokkos based version for on-node parallelism.}
          
\cbb{rap: triple product}{bool}{false}{Use all-at-once triple matrix product kernel}
          


% ==================== MAXWELL ====================
\section{Maxwell solver options}


\cbb{refmaxwell: mode}{string}{"additive"}{Specifying the order of solve of the block system. Allowed values are: "additive" (default), "121", "212"}
          
\cbb{refmaxwell: disable addon}{bool}{true}{Specifing whether the addon should be built for stabilization}
          
\cba{refmaxwell: 11list}{\parameterlist}{Specifies the multigrid solver for the 11 block}
          
\cba{refmaxwell: 22list}{\parameterlist}{Specifies the multigrid solver for the 22 block}
          
\cbb{refmaxwell: use as preconditioner}{bool}{false}{Assume zero initial guess}
          
\cbb{refmaxwell: dump matrices}{bool}{false}{Dump matrices to disk.}
          
\cbb{refmaxwell: subsolves on subcommunicators}{bool}{false}{Redistribute the two subsolves to disjoint sub-communicators (so that the additive solve can occur in parallel).}
          
\cbb{refmaxwell: ratio AH / A22 subcommunicators}{double}{1.0}{Ratio for the split into sub-communicators.}
          

%%% Local Variables:
%%% mode: latex
%%% TeX-master: "mueluguide"
%%% End:


    %-----------------------------%
    \chapter{\muemex: The MATLAB Interface for \muelu} \label{sec:muemex}
    %-----------------------------%
    \input{muemex}

    %-----------------------------%
    %\chapter{YAML Parameter Lists}\label{sec:yaml}
    %-----------------------------%
    %YAML is a human-readable data serialization format. MueLu provides a
YAML parameter list interpreter. It produces Teuchos::ParameterList
objects equivalent to those produced by the Teuchos XML helper functions.

Here is a simple example XML parameter list:
\begin{verbatim}
<ParameterList>
  <ParameterList Input>
    <Parameter name="values" type="Array(double)" value="{54.3 -4.5 2.0}"/>
    <Parameter name="myfunc" type="string" value="
def func(a, b):
  return a * 2 - b"/>
  </ParameterList>
  <ParameterList Solver>
    <Parameter name="iterations" type="int" value="5"/>
    <Parameter name="tolerance" type="double" value="1e-7"/>
    <Parameter name="do output" type="bool" value="true"/>
    <Parameter name="output file" type="string" value="output.txt"/>
  </ParameterList>
</ParameterList>
\end{verbatim}

Here is an equivalent YAML parameter list:
\begin{verbatim}
%YAML 1.1
---
ANONYMOUS:
  Input:
    values: [54.3, -4.5, 2.0]
    myfunc: |-

      def func(a, b):
        return a * 2 - b
  Solver:
    iterations: 5
    tolerance: 1e-7
    do output: yes
    output file: output.txt
...
\end{verbatim}

The nested structure and key-value pairs of these two lists are identical.
To a program querying them for settings, they are indistinguishable.

These are the general rules for creating a YAML parameter list:
\begin{itemize}
\item First line must be ``\%YAML 1.1'', second must be ``---'', and last must be ``...''
\item Nested map structure is determined by indentation. SPACES ONLY, NO TABS!
\item As with the above example, for a top-level anonymous parameter list, ``ANONYMOUS:'' must be explicit
\item Type is inferred. 5 is an int, 5.0 is a double, and '5.0' is a string
\item Quotation marks (single or double) are optional for strings, but required for strings with special characters: \verb.:{}[],&*#?|-<>=!%@\.
\item Quotation marks also turn non-string types into strings: '3' is a string
\item As with XML parameter lists, keys are regular strings
\item Even though YAML supports several names for bool true/false, only ``true'' and ``false'' are supported by the parameter list reader.
\item Arrays of int, double and string supported. exampleArray: {[}hello, world, goodbye{]}
\item {[}3, 4, 5{]} is an int array, {[}3, 4, 5.0{]} is a double array, and {[}3, '4', 5.0{]} is a string array
\item For multi-line strings, place ``$|-$'' after the ``key:'' and then indent the string one level deeper than the key
\item To preserve indentation in a multiline string, place ``$|2-$'' and then indent your string's content by 2 spaces relative to the key.
\end{itemize}


    %\nocite{*}

    % ---------------------------------------------------------------------- %
    % References
    %
    \clearpage
    % If hyperref is included, then \phantomsection is already defined.
    % If not, we need to define it.
    \providecommand*{\phantomsection}{}
    \phantomsection
    \addcontentsline{toc}{chapter}{References}
    \bibliographystyle{plain}
    \bibliography{mueluguide}


    % ---------------------------------------------------------------------- %
    %
    \appendix
    \chapter{Copyright and License}
    \label{sec:license}
\begin{center}
MueLu: A package for multigrid based preconditioning

Copyright (c) 2012 NTESS and the MueLu contributors.
\end{center}

\noindent
Copyright 2012 National Technology \& Engineering Solutions of Sandia,
LLC (NTESS). Under the terms of Contract DE-NA0003525 with NTESS, the
U.S. Government retains certain rights in this software.

\noindent
Redistribution and use in source and binary forms, with or without
modification, are permitted provided that the following conditions are
met:

\begin{enumerate}
  \item Redistributions of source code must retain the above copyright
    notice, this list of conditions and the following disclaimer.

  \item Redistributions in binary form must reproduce the above copyright
    notice, this list of conditions and the following disclaimer in the
    documentation and/or other materials provided with the distribution.

  \item Neither the name of the Corporation nor the names of the
    contributors may be used to endorse or promote products derived from
    this software without specific prior written permission.
\end{enumerate}

\noindent
THIS SOFTWARE IS PROVIDED BY THE COPYRIGHT HOLDERS AND CONTRIBUTORS
``AS IS'' AND ANY EXPRESS OR IMPLIED WARRANTIES, INCLUDING, BUT NOT
LIMITED TO, THE IMPLIED WARRANTIES OF MERCHANTABILITY AND FITNESS
FOR A PARTICULAR PURPOSE ARE DISCLAIMED\@. IN NO EVENT SHALL THE
COPYRIGHT HOLDER OR CONTRIBUTORS BE LIABLE FOR ANY DIRECT, INDIRECT,
INCIDENTAL, SPECIAL, EXEMPLARY, OR CONSEQUENTIAL DAMAGES (INCLUDING,
BUT NOT LIMITED TO, PROCUREMENT OF SUBSTITUTE GOODS OR SERVICES\@;
LOSS OF USE, DATA, OR PROFITS\@; OR BUSINESS INTERRUPTION) HOWEVER
CAUSED AND ON ANY THEORY OF LIABILITY, \\WHETHER IN CONTRACT, STRICT
LIABILITY, OR TORT (INCLUDING NEGLIGENCE OR OTHERWISE) ARISING IN
ANY WAY OUT OF THE USE OF THIS SOFTWARE, EVEN IF ADVISED OF THE
POSSIBILITY OF SUCH DAMAGE\@.


%%% Local Variables:
%%% mode: latex
%%% TeX-master: "mueluguide"
%%% End:

    %\chapter{Historical Perspective}
	%\input{CommonHistory}

    \chapter{ML compatibility}
    \input{mloptions}

    %\chapter{Some Other Appendix}
	%\input{CommonAppendix}

    % \printindex

    %
% This is an example of how to create the distribution page. Some
% distributions are required by Sandia; e.g. the housekeeping copies.
% Depending on the type of report; e.g. CRADA, Patent Caution, etc.
% additional distribution lines may have to be added. See the
% "Guide for Preparing SAND Reports"
%
% SANDdistribution takes CA or NM as an optional argument. If given,
% the approrpiate housekeeping copies are inserted autmatically.
% Inside the SANDdistribution environment, several commands can be used
% insert the distributions for CRADA, LDRD, etc. See example below.
%
% You can leave the CA or NM option off and not use any of the SANDdist*
% commands. This will allow you to create a distribution list manually.
%
\begin{SANDdistribution}[NM]
    % Housekeeping copies necessary for every unclassified report:
    % \SANDdistCRADA	% If this report is about CRADA work
    % \SANDdistPatent	% If this report has a Patent Caution or Patent Interest
    % \SANDdistLDRD	% If this report is about LDRD work

    % Some external Addresses

    % \SANDdistExternal{3}{Some Address\\ and street\\City, State}
    % \SANDdistExternal{12}{Another Address\\ On a street\\City, State\\U.S.A.}
    %\bigskip

    % The following MUST BE between the external and internal distributions!
    % \SANDdistClassified % If this report is classified

    % Internal Addresses
    \SANDdistInternal{2}{0612}{Review \& Approval Desk}{4916}
    \SANDdistInternal{1}{0897}{Todd S. Coffey}{1543}
    \SANDdistInternal{2}{0899}{Technical Library}{9610}
    \SANDdistInternal{1}{1318}{Roscoe A. Bartlett}{1446}
    \SANDdistInternal{1}{1318}{Roger P. Pawlowski}{1446}
    \SANDdistInternal{1}{1320}{Curtis C. Ober}{1446}
    \SANDdistInternal{1}{9018}{Central Technical Files}{8945-1}
    
    % Example of a mail channel use (instead of a mail stop)
    % \SANDdistInternalM{1}{M9999}{Someone}{01234}

\end{SANDdistribution}


\end{document}


\cbb{aggregation: type}{string}{"uncoupled"}{Aggregation scheme. Possible values: see Table~\ref{t:aggregation}.}
          
\cbb{aggregation: mode}{string}{"uncoupled"}{Controls whether aggregates are allowed to cross processor boundaries. Possible values: "coupled" aggregates can cross processor boundaries, "uncoupled" aggregates cannot cross processor boundaries.}
          
\cbb{aggregation: ordering}{string}{"natural"}{Node ordering strategy. Possible values: "natural" (local index order), "graph" (filtered graph breadth-first order), "random" (random local index order).}
          
\cbb{aggregation: drop scheme}{string}{"classical"}{Connectivity dropping scheme for a graph used in
      aggregation. Possible values: "classical", "distance laplacian",
      "unsupported vector smoothing"}
          
\cbb{aggregation: drop tol}{double}{0.0}{Connectivity dropping threshold for a graph used in aggregation.}
          
\cbb{aggregation: min agg size}{int}{2}{Minimum size of an aggregate.}
          
\cbb{aggregation: max agg size}{int}{-1}{Maximum size of an aggregate (-1 means unlimited).}
          
\cbb{aggregation: compute aggregate qualities}{bool}{false}{Whether to compute aggregate quality estimates.}
          
\cbb{aggregation: brick x size}{int}{2}{Number of points for x axis in "brick" aggregation (limited to 3).}
          
\cbb{aggregation: brick y size}{int}{2}{Number of points for y axis in "brick" aggregation (limited to 3).}
          
\cbb{aggregation: brick z size}{int}{2}{Number of points for z axis in "brick" aggregation (limited to 3).}
          
\cbb{aggregation: brick x Dirichlet}{bool}{false}{Asserts that Dirichlet conditions are applied in
        the x-direction and the Dirichlet DOFs are not aggregated.}
          
\cbb{aggregation: brick y Dirichlet}{bool}{false}{Asserts that Dirichlet conditions are applied in
        the y-direction and the Dirichlet DOFs are not aggregated.}
          
\cbb{aggregation: brick z Dirichlet}{bool}{false}{Asserts that Dirichlet conditions are applied in
        the z-direction and the Dirichlet DOFs are not aggregated.}
          
\cbb{aggregation: Dirichlet threshold}{double}{0.0}{Threshold for determining whether entries are zero during Dirichlet row detection.}
          
\cbb{aggregation: deterministic}{bool}{false}{Boolean indicating whether or not aggregation will be run deterministically in the kokkos refactored path (only used in uncoupled aggregation).}
          
\cbb{aggregation: coloring algorithm}{string}{serial}{Choice of distance 2 coloring algorithm used by Uncoupled Aggregation. Currently set by default to COLORING\_D2\_SERIAL.}
          
\cbb{aggregation: export visualization data}{bool}{false}{Export data for visualization post-processing.}
          
\cbb{aggregation: output filename}{string}{""}{Filename to write VTK visualization data to.}
          
\cbb{aggregation: output file: time step}{int}{0}{Time step ID for non-linear problems.}
          
\cbb{aggregation: output file: iter}{int}{0}{Iteration for non-linear problems.}
          
\cbb{aggregation: output file: agg style}{string}{Point Cloud}{Style of aggregate visualization.}
          
\cbb{aggregation: output file: fine graph edges}{bool}{false}{Whether to draw all fine node connections along with the aggregates.}
          
\cbb{aggregation: output file: coarse graph edges}{bool}{false}{Whether to draw all coarse node connections along with the aggregates.}
          
\cbb{aggregation: output file: build colormap}{bool}{false}{Whether to output a random colormap in a separate XML file.}
          
\cbb{aggregation: mesh layout}{string}{Global Lexicographic}{Type of ordering for structured mesh aggregation. Possible values: "Global Lexicographic" and "Local Lexicographic".}
          
\cbb{aggregation: output type}{string}{Aggregates}{Type of object holding the aggregation data. Possible values: "Aggregates" or "CrsGraph".}
          
\cbb{aggregation: coarsening rate}{string}{{3}}{Coarsening rate per spatial dimensions, the string must be interpretable as an array by Teuchos.}
          
\cbb{aggregation: number of spatial dimensions}{int}{3}{The number of spatial dimensions in the problem.}
          
\cbb{aggregation: coarsening order}{int}{0}{The interpolation order used while constructing these aggregates, this value will be passed to the prolongator factory. There, possible values are 0 for piece-wise constant and 1 for piece-wise linear interpolation to transfer values from coarse points to fine points. }
          
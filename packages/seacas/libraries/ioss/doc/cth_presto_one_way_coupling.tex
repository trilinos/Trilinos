% This file was converted to LaTeX by Writer2LaTeX ver. 0.4
% see http://www.hj-gym.dk/~hj/writer2latex for more info
\documentclass[11pt,twoside]{article}
\usepackage{alltt}
\usepackage{calc}
\usepackage{amsmath,amssymb,amsfonts,textcomp}
\usepackage{hyperref}
\hypersetup{colorlinks=true, linkcolor=blue, filecolor=blue, pagecolor=blue, urlcolor=blue}
% Text styles
\newcommand\textstyleInternetlink[1]{\textcolor{blue}{#1}}
\newcommand\textstyleFootnoteSymbol[1]{\textsuperscript{#1}}
% Outline numbering
\setcounter{secnumdepth}{0}
\newcommand{\cmd}[1]
   {\mbox{\tt #1}\null}
\newcommand{\code}[1]
   {\mbox{\bf\tt #1}\null}
\newcommand{\file}[1]
   {\mbox{\bf\em #1}\null}
\newcommand{\param}[1]
   {\mbox{\{\em #1\}}\null}

% Pages styles (master pages)
\setlength\parindent{0in}
\setlength{\parskip}{1ex plus 0.5ex minus 0.2ex}
\setlength\textwidth{6.5in}
\makeatletter
\newcommand\ps@Standard{%
\renewcommand\@oddhead{}%
\renewcommand\@evenhead{}%
\renewcommand\@oddfoot{}%
\renewcommand\@evenfoot{}%
\setlength\paperwidth{8.5in}\setlength\paperheight{11in}\setlength\voffset{-1in}\setlength\hoffset{-1in}\setlength\topmargin{1in}\setlength\headheight{12pt}\setlength\headsep{0cm}\setlength\footskip{12pt+0cm}\setlength\textheight{11in-1in-1in-0cm-12pt-0cm-12pt}\setlength\oddsidemargin{1in}\setlength\textwidth{8.5in-1in-1in}
\renewcommand\thepage{\arabic{page}}
\setlength{\skip\footins}{0.0398in}\renewcommand\footnoterule{\vspace*{-0.0071in}\noindent\textcolor{black}{\rule{0.25\columnwidth}{0.0071in}}\vspace*{0.0398in}}
}
\makeatother
\pagestyle{plain}
% footnotes configuration
\makeatletter
\renewcommand\thefootnote{\arabic{footnote}}
\makeatother

\newenvironment{source}
{\small\begin{quote}\begin{alltt}}
{\end{alltt}\end{quote}\normalsize}

\begin{document}
\clearpage\pagestyle{Standard}
\section[CHSSI CTH Presto One{}-Way Coupling Procedure]{CHSSI CTH Presto
One{}-Way Coupling Procedure}
{Version: \centering \today}

Points of contact at SNL:
\begin{quote}
David Crawford,
\href{mailto:dacrawf@sandia.gov}{\nolinkurl{dacrawf@sandia.gov}}
505{}-845{}-8975  (CTH)

Arne Gullerud,
\href{mailto:asgulle@sandia.gov}{\nolinkurl{asgulle@sandia.gov}}
505{}-844{}-4326  (Presto)

Greg Sjaardema,
\href{mailto:sierra-help@sandia.gov}{\nolinkurl{sierra-help@sandia.gov}}
(one{}-way coupling tools, this document)
\end{quote}

\section{Overview}
The CHSSI-sponsored One{}-Way coupling of the Eulerian code \code{CTH}
and the Lagrangian finite element code \code{Presto} is primarily
focused on simulating the response of structures due to blast loads.
The one-way coupling capability assumes that the structural response
time is much greater than the characteristic time of the impinging
pressure wave from the explosion and, therefore, the pressure loading
response can be calculated by CTH on a rigid representation of the
structural model.  These pressures are then read by Presto and applied
as time-varying boundary conditions to the finite element
representation of the structural model to calculate the structural
response.  There is no feedback or communication of data from the
\code{presto} analysis back to \code{CTH}; all communication is
one{}-way from \code{CTH} to \code{presto}.

The codes involved in this process include CTH, Presto, and a few
utility programs (described below) used to facilitate the data
movement and data manipulation.

\section{Procedure Overview}
The one{}-way coupling procedure consists of the following steps:
\begin{enumerate}
\item
Generate finite element mesh of the structural model. The model must
consist entirely of volume elements (hexes, tetrahedrons, wedges) at
this time; structural elements (shells, beams, trusses) are not yet
supported. A workaround for shell models is discussed in Section~\ref{sec:shell}.

\item Generate a model consisting of the exposed surface (`skin') of the
structural model.

\item Embed the exposed surface model in the CTH grid using a \cmd{diatom} package.

\item Perform CTH blast analysis and export time{}-averaged pressures on the
surface model.

\item Merge the CTH pressure data with the original finite element mesh for
use by Presto.

\item Perform the Presto structural analysis using transient pressure data.
\end{enumerate}

\subsection{Step 1: Generate finite element model}
NOTE: This step is outside the scope of this document; the finite
element model for the structural mesh can be generated using any
finite element mesh generation package.


\subsection{Step 2:  Generation of exposed surface model}\label{sec:surfmodel}

The Sierra application \code{skinner} is used to generate an exodusII
file containing the exposed surface facets and outward pointing normals
of the structural mesh as required by CTH.  The \code{skinner}
application requires an input file that defines the filename of the
structural finite{}-element model (exodusII format) and the filename
for output.  The input file format is:
\begin{source}
BEGIN SIERRA \textit{label}
BEGIN FINITE ELEMENT MODEL \textit{model\_label}
  DATABASE NAME = \file{structural\_model.g}
  OMIT BLOCK block\_[?]      # optional
 END

 BEGIN MODEL TO SKIN
  USE FINITE ELEMENT MODEL \textit{model\_label}

  BEGIN RESULTS OUTPUT
    DATABASE NAME = \file{surface\_model.g}
    OUTPUT MESH IS EXPOSED SURFACE
  END
END
END SIERRA
\end{source}
Note that the Sierra parsing service dictates the syntax of this input
file and is somewhat verbose.  All data in uppercase font is required
(although it can be entered in either upper or lowercase).  The data
in lowercase italics is arbitrary, but whatever is entered for
\textit{model\_label} must match in both locations.  The \file{bold
lowercase} data specify filenames.  If the filenames do not start with
a slash (`/'), then the specified path is relative to the location
where the user starts the application. If the filenames do start with a
slash (`/'), then the specified path is assumed to be an absolute path
to the location of the files.  The `\cmd{OMIT BLOCK}' commands can
be used to limit the element blocks inserted into the CTH simulation.


The \code{skinner} application is run similarly to other Sierra
applications.  Assuming that the user has the \code{sierra} script in their
path and that the input file is named \file{input\_file.i}, one way to
run the application would be:
\begin{source}
sierra skinner {-}i \file{input\_file.i}
\end{source}
There may be other methods for running sierra applications depending on
specifics at the various installation sites.  If everything runs
correctly, the result of the above command should be a new file
(\file{surface\_model.g} in the above example) containing surface
elements defining the exposed surface of the structural model. This
file will also contain data defining the outward normal vector at each
node and face as needed by CTH.

\subsection{Steps 3 and 4: Define CTH Model and Run CTH Analysis}
CTH handles one{}-way coupling with Presto as a special
option within the \cmd{diatom} material insertion interface.  It treats the
Presto mesh as an absolutely rigid material using the \cmd{rigid} capability
that has been in CTH for some time.  \cmd{rigid} needs to be invoked when
using one{}-way coupling by inserting a \cmd{rigid} keyword in the \cmd{control}
section of the input deck as follows:
\begin{source}
control
${\ldots}$*(other keywords)
rigid
${\ldots}$*(other keywords)
endc
\end{source}

All the example problems included with this primer invoke \cmd{rigid}
in this way.  It is assumed that CTH is being used in GEN{}-less mode
when using the one-way capability.

The actual insertion of Presto geometry and the harvesting of
resultant pressures for handoff to Presto are handled by \cmd{diatom}.
Insertion of the surface description is handled by a special \cmd{one{}-way
diatom} package. Typical usage is:
\begin{source}
diatom
${\ldots}$
  package 'My Presto Mesh'
    one_way
    dt_out = 1e-6
    insert exodus_surf
      file = \file{surface\_model.g}
      inch-psi
      *d_scale=2.54
      *p_scale=14.5e-6
      *bl = nn
    endi
  endp
${\ldots}$
* (other packages of normal CTH materials to fill out the rest of the problem)
${\ldots}$
enddiatom
\end{source}
The \cmd{one\_way} keyword denotes the package as containing an
\cmd{exodus\_surf} description. The \cmd{dt\_out} keyword specifies how often
coupling data is to be added to the output Exodus file (described
below). The \cmd{inch{}-psi} keyword specifies the units to be used
by the \cmd{exodus\_surf} file; default units are cgs (cm,
dynes/cm\textsuperscript{2}). The optional keyword \cmd{d\_scale} is
the factor for converting \cmd{exodus\_surf} distance to
centimeters and the keyword \cmd{p\_scale} is the factor for
converting dynes/cm\textsuperscript{2} to \cmd{exodus\_surf}
pressure units. The \cmd{inch{}-psi} keyword will set the
\cmd{d\_scale} and \cmd{p\_scale} to 2.54 and 14.5e{}-6
respectively\footnote{More shortcut conversion specifiers like
\cmd{inch{}-psi} can be added as needed. Please contact the CTH developers if you have a specific request for
this.}. The optional keyword \cmd{bl} can be used to specify a
particular element block (an integer number) from the Exodus file for
insertion into the CTH mesh. If \cmd{bl} is not specified then all
element blocks are inserted.  You may use as many \cmd{one\_way} packages
as desired.  It is recommended that you place \cmd{one\_way} packages
(actually all \cmd{rigid} packages) at the beginning of the \cmd{diatom}
description to reflect the fact that they will always be inserted
before regular CTH materials (regardless of their placement in the
\cmd{diatom} section).

Ordering of the \cmd{one\_way} packages should not matter.  It is not
recommended that \cmd{one\_way} packages be allowed to physically
overlap (or to overlap with other rigid material) as this will lead to
incorrect behavior.


The output file produced by CTH for coupling with Presto is named
\file{\param{name}.cthpres}, where \file{\param{name}} is the name of
the exodusII mesh file used to define the rigid geometry
(\file{surface\_model} in the above example).  This output file
contains an extra element variable called \cmd{CTH\_Pressure}; this
variable is the time{}-varying pressure computed by CTH on each of the
faces in the rigid geometry.  These pressures may be viewed and
animated in any visualization software supporting the exodusII file
format.  However, additional processing is needed before they can be
used as input to Presto as described in step~5 below.

The Presto mesh inserted into CTH via the \cmd{exodus\_surf} description is
assumed to be non{}-moving and non{}-deforming.  It is modeled
internally as an absolutely rigid, frictionless material (i.e. the
normal velocity component is zeroed at the surface of the Presto
material but tangential velocity component is unaffected).  Because
CTH uses a rectilinear grid and face centered velocity components, the
rigid approximation is not perfect, however, and some artifacts may
occur. There is a natural tendency for material to `bounce' off of the
rigid material as a consequence of conservation of momentum and the
finite size of CTH cells.  This `bouncy' behavior leads occasionally
to regions of zero pressure (void) against the surface of the Presto
mesh.  Typically, these regions are small and the zero pressure values
last only briefly.  Because momentum is conserved however, the total
impulse passed to the Presto mesh is correct even in the presence of
occasional zero pressures.

A unique aspect of the algorithm being used in the one\_way coupling
capability is our use of a time{}-integral for computing the average
pressure on each facet of the surface exodus mesh.  Typically the
\cmd{dt\_out} time interval specified is much longer than the average CTH
time step.  CTH internally accumulates an average value of the
pressure acting on each facet across the entire \cmd{dt\_out} time
interval.  Consequently, typical pressure values sent to the output
exodus file are time{}-integrated averages across multiple CTH cycles.

\subsection{Step 5:  Merge Pressure Data With Original Structural Model}

One of the outputs from the CTH analysis should be a file
(`\file{surf\_model.cthpres}' in the above example) containing the calculated
pressures as a function of time on each face.  This data must be
merged with the original finite element model.  The merged file is
then used as the finite element model for the Presto analysis.  The sierra application
\code{cth\_pressure\_map}\footnote{In earlier releases of the
CTH--Presto one-way coupling process, the name for this code was
\code{io\_shell}. If \code{cth\_pressure\_map} does not exist, use
\code{io\_shell} instead.} is used for this.  The basic minimal
command line for this is:
\begin{source}
sierra cth\_pressure\_map {-}i \file{merge\_file.i}
\end{source}
The \file{merge\_file.i} file contains three lines:
\begin{source}
\file{structural\_model.g} exodusII
\file{surface\_model.cthpres} exodusII
\file{structural\_model\_with\_pressures.g} exodusII
\end{source}
where \file{structural\_model.g} is the original structural finite element
model, \file{surface\_model.cthpres} is the surface model with pressures
created by CTH, and \file{structural\_model\_with\_pressures.g} is the
output file containing the structural finite element model merged with
the pressure data which will be used as input to the Presto analysis.

There are several ways in which the pressure data can be modified to
suit the needs of the Presto analysis.  Each of these modifications is
controlled by an option passed to the \code{cth\_pressure\_map} application.  By
default, \textit{all} data calculated by CTH is merged unchanged with
the finite element model.  The following modifications are supported:

\begin{description}
\item [Convert from absolute pressure to gage pressure.]  The pressure data
calculated by CTH is \textit{absolute} pressure while Presto analyses
typically use \textit{gage} pressure.  The option `\cmd{{}-O
{--}Convert\_Gage}'\footnote{ The strange `\cmd{{-}O "{--}option
value"}' syntax is dictated by the sierra script as a method for
passing `uncommon' options.  If the executable is run directly without
using the sierra script, then the `\cmd{{-}O}' and the double quotes
can be omitted.  Note that typing `\cmd{{}-O option}' instead of
`\cmd{{}-O {--}option}' is a common error; always make sure that the
option has the required double-dash (`{}--`).} will convert the
absolute pressures to gage pressure.  The pressure data at time 0.0 is
assumed to be the ambient absolute pressure value.
\item [Specify maximum time to be merged.]  If the analyst does not want all
data on the pressure file to be merged with the structural model, the
data can be limited by specifying the time up to which the data is
desired.  The option is `\cmd{{-}O "{--}Maximum\_Time
\param{max\_time}"}' where `\param{max\_time}' is the maximum time
desired.
\item [Pressure values at long time.]  One of the assumptions of the one{}-way
coupling is that the blast time is much less than the structural
response time.  Thus the pressure data should only exist for a small
portion of the structural analysis time.  \textit{By default in Presto
the pressure value existing at the end of the pressure data file will
be used for all times after that time. } If the CTH blast analysis
were run to a quasistatic equilibrium state, this would be the correct
behavior.  However, the CTH analysis is typically not run for such a
long physical time period and therefore the pressure data will not have
reached an equilibrium state.  The user can specify the pressure
values to use for all times beyond the time period of the CTH analysis.
The options are:

\begin{enumerate}
\item \cmd{{-}O "{--}Final\_Pressure ZERO"  }(The long{}-time
pressures are all zero.)
\item \cmd{{-}O "{--}Final\_Pressure INITIAL"}  (The long{}-time
pressures are the same as the pressures at time = 0.0 on the pressure
data file.  Note that if the \cmd{Convert\_Gage }option is
specified, then \cmd{ZERO} and \cmd{INITIAL} will give the same
results.)
\item \cmd{{-}O "{--}Final\_Pessure FINAL"}  (The calculated
pressures at the end of the file will be used from that time onward.)
\item \cmd{{-}O "{--}Final\_Pressure OFFSET"}  (The long{}-time
pressures are the same as the pressure value specified via the
\cmd{Offset\_pressure} option. Note that if the \cmd{Convert\_Gage }option is
specified, then the final pressures will also be modified by the gage
pressure correction.
\end{enumerate}

\item [Long time pressure time.]  If the `\cmd{{--}Final\_Pressure}' option
is specified, the merging code adds a new time step with the specified
final pressure. By default, the delta time between the last calculated
pressures and the final pressure data is the same as the delta time
between the last two calculated pressures. Depending on the pressure
state at which the CTH analysis was terminated, this may result in a
sharp pressure gradient at this time.  If a longer delta time is
wanted, it can be specified via the \cmd{{}-O
"{--}Final\_Time\_Delta \param{dt}"} option where
\param{dt} is the desired time delta. This
option has no affect if the \cmd{{--}Final\_Pressure} option is
not specified.
\item [Short time.]  If the \cmd{{}--Minimum\_time} option is specified the
merging code deletes the pressures for times less than the specified
value. All output time values will be offset by the time value of the
first timestep with a time larger than the specified minimum time.
\item [Offset time.] If the \cmd{{}--Offset\_time} option is specified,
then all output time values will be offset by the specified value from
the input time values.  In addition, an initial time step at time 0.0
will be output with all pressure values set to 0.0 (or the pressure
offset value if specified).

\item [Offset pressure.] If the \cmd{{}--Offset\_pressure} option is
specified, then the specified value will be added to all output pressures
\end{description}

A more complicated example of running the pressure merging step is:

\begin{source}
sierra cth_pressure_map -i \file{merge\_file.i}
   {-}O "{--}Final\_Pressure ZERO"
   {-}O "{--}Maximum\_Time 0.0015"
   {-}O {--}Convert\_Gage
   {-}O "{--}Final\_Time\_Delta 0.001"
   {-}O "{--}Minimum\_time 0.0001"
\end{source}

In this example pressures are merged with structural finite element
model, converted from absolute pressure to gage pressure.  The
pressure data is only merged after time 0.0001 up to time 0.0015.
 After that time the pressure data is zeroed and the zeroed data is
written at time 0.0025 (0.0015 + 0.001).

\subsection{Step 6: Perform Presto Structural Analysis}\label{sec:step6}

Within Presto, the pressure data and the finite element model are read
from the same exodusII database
(\file{structural\_model\_with\_pressures.g }in the above example).
The following `pressure block' located in the `region scope' tells
presto where to find the pressure data:

\begin{source}
begin pressure
   surface = ssblock\_1
   read variable = cth\_pressure
end
\end{source}

Here, \cmd{ssblock\_1} is the new surface created by the merge process in
step~5 above.  The actual surface name can be
determined via the \cmd{explore} exodusII query program.  The syntax and
output is:
\begin{source}
explore structural_model_with_presssures.g
EXPLORE> list ssets
EXPLORE> exit

 Set    1 (#1):     11520 elements (index=1)     ... name = "surface_hex8_quadface4_1"
 Set    2 (#2):      1440 elements (index=11521) ... name = "surface_hex8_quadface4_2"
 Set    3 (#3):     51264 elements (index=12961) ... name = "ssblock_1"
    cth_pressure
\end{source}
Note that sideset 3 which is named \cmd{ssblock\_1} has the variable
\cmd{cth\_pressure} defined on it and is the last sideset in the model.

\section{Notes}

\begin{itemize}
\item It is important to note that the pressure mapping from the CTH generated
results to the Presto input mesh is \em{topological} and not \em{spatial}.
 Thus, the finite element mesh skinned and used in the CTH simulation
must have the same connectivity and numbering when used in the Presto
simulation, albeit with possible additional elements.  Thus, it is
somewhat `tricky' to modify the structural finite{}-element model
without rerunning the CTH analysis, although it can be done with care.
 A general spatial mapping routine could be implemented as well for
increased flexibility.

\item We recommend having at least three CTH cells through any solid material
(through thickness) in order to avoid any `bleed through' of pressure
from one side to the other, e.g on thin plates.

\item The pressure distributions can be spatially clipped within Presto input
deck by using the `\cmd{remove surface}' option within the
\cmd{pressure} block.

\item Care must be exercised when solid material in the CTH simulation is in
contact with the inserted finite{}-element model as only the scalar
hydrostatic pressure is mapped and not the traction vector.  Future
enhancements could include mapping of the full traction vector obtained
from the stress tensor.

\item It is possible to export to the exodus output file the CTH pressure
contribution to the external force vector in the Presto simulation.
 These values can also be summed over a surface using commands within
the Presto input deck for output to the exodus history file.

\end{itemize}

\section{Shell Workaround}\label{sec:shell}

The one-way coupling process is currently only supported for finite
element models consisting of hexahedral elements.  The reason for this
limitation is that the \code{CTH} mesh embedding process requires that the
surface description be ``watertight'' and that the ``back'' of a
surface face must extend out of the \code{CTH} cell that the front
face is in.  Basically, the problem is that the zero-thickness shells
can provide an ambiguous surface description.

However, since many models are composed of geometries for which shell
elements would be the logical element type to use in the \code{Presto}
structural response calculations, a ``kluge'' or ``workaround'' has
been developed which gives the analyst the capability to use shell
elements in the \code{Presto} analysis and still provide \code{CTH}
with a well-defined surface mesh. The workaround involves some extra
work for the analyst.

The workaround is to add one or more layers of hex elements to the
``back'' (away from the pressure source) of the shell structure. The
hexes are then deleted from the model when the structural code is
run. The hex elements are only needed to provide a non-zero thickness
surface geometry for the \code{CTH} analysis and are not present in
the \code{Presto} analysis.  The hex-shell interface {\em must} be
contiguous; that is, the hexes and shells at the interface must share
the nodes. The process for doing this varies in the different mesh
generation programs, but it is a common meshing procedure so it should
be supported in all of the programs.

The one-way coupling process for structures containing shells is
executed as described earlier in this document except for a couple
modifications:
\begin{itemize}
\item The \cmd{OMIT BLOCK} command is used to delete the hex elements
which have been added as backing to the shells (If they are truly not
part of the structural model). The syntax for this is shown in
Step~2; the command would be added to the
\cmd{BEGIN FINITE ELEMENT MODEL} command block in the \code{Presto} input file in Step~6.
\item If the \code{Presto} analysis is run in parallel, then the load
balancing can be skewed by the hex elements which will be omitted
which can cause the analysis to be unbalanced.  To fix this, the load
balancing script \code{loadbal} is told to ``weight'' the retained
element blocks much more than the deleted element blocks. For example,
if the model contains a shell element block with id 10 and the hex
elements to be deleted are in element block 20; a loadbal command to
run this model on 16 processors would be:
\begin{source}
 loadbal -n "-w eb=10:1000 -w eb=20:1" -p 16 -inertial \ldots
\end{source}
\begin{quote}
Where:
\begin{description}
   \item[\cmd{-n}] indicates to pass the following option on to the \code{nem\_slice} procsss.
   \item[\cmd{"\ldots"}] is the option that will be passed to \code{nem\_slice}.
   \item[\cmd{-w eb=10:1000}] specifies that the element block with the id of
10 has a weight of 1000.  This is the shell element block that will be
retained in the analysis.
   \item[\cmd{-w eb=20:1}] specifies that the element block with the id of 20
has a weight of 1.  This is the hex element block that will be omitted
from the analysis.
\end{description}
This will typically give a good load balance for the retained element blocks.
\end{quote}
\end{itemize}
\end{document}

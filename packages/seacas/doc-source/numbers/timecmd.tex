\cmddef{\cmdverb{TMIN}
   \param{tmin} \default{minimum database time}
} {
\cmd{TMIN} sets the minimum selected time \param{tmin} to the specified
parameter value. If the user-selected-times mode is in effect, the mode
is changed to the all-available-times mode.

In interval-times mode, if \param{nintv} is selected (by a \cmd{NINTV}
or \cmd{ZINTV} command), \param{delt} is calculated. If \param{delt} is
selected (by a \cmd{DELTIME} command), \param{nintv} is calculated.
}

\cmddef{\cmdverb{TMAX}
   \param{tmax} \default{maximum database time}
} {
\cmd{TMAX} sets the maximum selected time \param{tmax} to the specified
parameter value. If the user-selected-times mode is in effect, the mode
is changed to the all-available-times mode.

In interval-times mode, if \param{nintv} is selected (by a \cmd{NINTV}
or \cmd{ZINTV} command), \param{delt} is calculated. If \param{delt} is
selected (by a \cmd{DELTIME} command), \param{nintv} is calculated.
}

\cmddef{\cmdverb{NINTV}
   \param{nintv} \default{10 or the number of database time steps $-$ 1,
      whichever is smaller}
} {
\cmd{NINTV} sets the number of selected time intervals \param{nintv} to
the specified parameter value and changes the mode to the interval-times
mode with a delta offset. The selected time interval \param{delt} is
calculated.
}

\cmddef{\cmdverb{ZINTV}
   \param{nintv} \default{10 or the number of database time steps,
      whichever is smaller}
} {
\cmd{ZINTV} sets the number of selected time intervals \param{nintv} to
the specified parameter value and changes the mode to the interval-times
mode with a zero offset. The selected time interval \param{delt} is
calculated.
}

\cmddef{\cmdverb{DELTIME}
   \param{delt}
      \default{$(tmax-tmin) / (nintv-1)$, where \param{nintv} is 10
      or the number of database time steps, whichever is smaller}
} {
\cmd{DELTIME} sets the selected time interval \param{delt} to the
specified parameter value and changes the mode to the interval-times
mode with a zero offset. The number of selected time intervals
\param{nintv} is calculated.
}

\cmddef{\cmdverb{ALLTIMES}
} {
\cmd{ALLTIMES} changes the mode to the all-available-times mode.
}

\newpage
\cmddef{\cmdverb{TIMES}
   [\cmd{ADD},]
   \param{t$_{1}$}, \param{t$_{2}$}, \ldots\
      \default{no times selected}
} {
\cmd{TIMES} changes the mode to the user-selected-times mode and selects
times \param{t$_{1}$}, \param{t$_{2}$}, etc. The closest time step from the
database is selected for each specified time.

Normally, a \cmd{TIMES} command selects only the listed time steps. If
\cmd{ADD} is the first parameter, the listed steps are added to the
current selected times. Any other time step selection command clears all
\cmd{TIMES} selected times.

Up to the maximum number of time steps in the database may be specified.
Times are selected in the order encountered on the database, regardless
of the order the times are specified in the command. Duplicate
references to a time step are ignored.
}

\cmddef{\cmdverb{STEPS}
   [\cmd{ADD},]
   \param{n$_{1}$}, \param{n$_{2}$}, \ldots\
      \default{no steps selected}
} {
The \cmd{STEPS} command is equivalent to the \cmd{TIMES} command except
that it selects time steps by the step number, not by the step time.
}

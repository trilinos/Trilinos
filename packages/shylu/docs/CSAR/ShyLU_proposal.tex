\documentclass[10pt]{amsart}

\usepackage{fullpage}
%\usepackage{amsmath}
%\usepackage{amsthm}
%\usepackage{graphicx}
%\usepackage{url}
%\usepackage{subfigure}
%\usepackage[foot]{amsaddr}

\date{}
\title{ShyLU: A Scalable Hybrid LU Solver for Sparse Linear Systems}

\pagestyle{empty}
\setlength{\topmargin}{-0.25in}
\setlength{\headheight}{0in}
\setlength{\headsep}{0in}
\advance\textheight by 1.0in 
%\advance\textwidth by 0.6in 
%\advance\oddsidemargin by -0.3in
%\advance\evensidemargin by -0.3in


\begin{document}

\maketitle

%\vspace{-8mm}
\section{Introduction}

As parallelism in a single node increases, NW applications
have to adapt to a hybrid system where each compute node by itself
is a shared memory system. 
In order to address the new challenges and opportunities 
for robust sparse linear solvers on the node,
we have developed a hybrid sparse solver ShyLU (initially funded by CSCAPES
and EASI). ShyLU is hybrid in the
mathematical sense (direct and iterative methods) making it more
robust than other preconditioners, while being less expensive
than a direct factorization.  ShyLU is also hybrid in the parallel
computing sense (MPI and threads). We envision ShyLU as a scalable
subdomain solver for large problems and/or as a standalone black-box solver
for medium sized problems.

ShyLU is based on a general Schur complement framework, with different options
for partitioning the matrices, block-diagonal solves and Schur complement
approximations.  Our hybrid implementation scales better than
a flat MPI implementation as the problem size per subdomain gets smaller, which
is important for some of the applications discussed below.
The MPI + Threads implementation helps ShyLU to scale well for up
to $384$ cores.

%\vspace{-2mm}
\section{Proposed Work}

\paragraph{\bf{ShyLU package}:}
ShyLU will be delivered as a package in Trilinos in the first year.
ShyLU will address the needs of applications that need a robust preconditioner
and/or a scalable MPI+threads subdomain solver, especially focusing on
strong scaling on the node.
We have implemented ShyLU using various packages in Trilinos and Trilinos will
be the ideal delivery vehicle. Our existing implementation is Epetra based,
as ShyLU uses lots of Trilinos packages.

TBD : Still needed why research code to production code will take time, tpetra
version in the long run, required by GDSW, MueLU etc.

\paragraph{\bf{Algorithms}:} ShyLU's robustness and scalability depends on good
approximations to the Schur complement. Currently, ShyLU uses 
structure-based probing
or a values-based dropping method. The dropping approximation, for
example, was added for circuit simulation problems. However, for problems
with a nice stucture to the Schur complement, from applications like Tramanto,
a probing based method will be more appropriate. We will develop better
approximation to Schur complement based on the needs of our applications, for
example we know indefinite matrices require a better approximation than either
of the two approaches in ShyLU.

\paragraph{\bf{Application Ties}:}

%\section{Application Ties}

\section{Personnel}
Erik Boman, Heidi Thornquist, Siva Rajamanickam.

\end{document}
